К командам побитовой обработки данных относятся логические команды, команды сдвига, установки, сброса и инверсии битов.

К логическим операциям относят команды and, or, xor, not. Для всех логических команд, кроме not операнды одновременно
не могут находится в памяти. OF = CF = 0, AF - не определен, SF, ZF, PF определяются результатом команды.

Второй операнд называют маской. Основным назначением команды and является установка в ноль с помощью маски
некоторых разрядов первого операнда. 

\subsubsection{Директивы внешних ссылок}

Директивы внешних ссылок позволяют организовать связь между различными модулями и файлами, расположенными на диске.

\begin{minted}{asm}
    Public <имя> [,<имя>, ..., <имя>]
\end{minted}

определяет указанные имена как глобальные аеличины, к которым можно обратиться из других модулей. Имена здесь "---
имена меток и переменных, определенных с помощью директивы '=' и EQU.

Если некоторое имя определено в модуле A как глобальное, и к нему можно обратиться из других модулей B и C, то в этих
модулях должна быть директива вида

\begin{minted}{asm}
    EXTRN <имя>:<тип> [,<имя>:<тип>]
\end{minted}
Здесь имя то же, что и в \textbf{Public}, а тип определяется
следующим образом: если <имя> "--- это имя переменной, то типом
может быть: BYTE, WORD, DWORD, FWORD, QWORD, TWORD.
Если <имя> "--- это имя метки, то типом может быть \textbf{NEAR} и \textbf{FAR}.

Директива \textbf{EXTRN} говорит о том, что перечисленные имена являются
внешними для данного модуля.

\subsubsection{Команды управления}

