\documentclass{article}
\usepackage[utf8]{inputenc}
\usepackage[russian]{babel}
\usepackage{hyperref}
\usepackage{underscore}
\usepackage{setspace}
\usepackage{indentfirst} 
\singlespacing

\begin{document}

\section{Фонетика}
Фонетика - раздел языкознания, в котором изучается звуковой
строй языка, т.е. звуки речи, слоги, ударения, интонацию.

\begin{itemize}
    \item \textbf{Акустика речи} изучает физические признаки речи.
    \item \textbf{Физиология речи} изучает биологические признаки речи, 
    т.е. работу, производимую человеком при произнесении (артикуляции) 
    или восприятии звуков речи.
    \item \textbf{Фонология} изучает звуки речи как средство общения, 
    т.е. функцию или роль языков...?
\end{itemize}

\subsection{Акустика речи}
\textbf{Звуки речи} -- это колебания воздушной среды, вызванные органами речи.

Звуки делятся на тоны (музыкальные звуки) и шумы (немузыкальные звуки).
\begin{itemize}
    \item \textbf{Тон} - это периодические ритмичные колебания голосовых связок.
    \item \textbf{Шум} - это периодические ритмичные колебания звучащего тела.
\end{itemize}

\subsection{Речевой аппарат человека}
\textbf{Подвижные органы речи}
\begin{itemize}
    \item Язык
    \item Губы
    \item Мягкое небо с маленьким язычком (увула)
    \item Задняя стенка фаринкса
    \item Голосовые связки
    \item Нижняя челюсть
\end{itemize}

\textbf{Неподвижные органы речи}
\begin{itemize}
    \item Верхняя челюсть
    \item Альвеолы
    \item Твёрдое небо с маленьким язычком (увула)
\end{itemize}

\subsection{Классификация звуков речи}
\textbf{Гласные} -- звуки, при артикуляции которых речевой канал в надгортанных
полостях свободен, открыт и ток воздуха не встречает препятствий.

У гласных нет определённого ``фокуса''.

\textbf{Согласные} -- при их произнесении обязательно возникает та или
иная преграда на пути воздушной струи.

\subsubsection{Классификация гласных}
\begin{itemize}
    \item по работе языка: учитывается ряд (передние, задние и смешанные) и степень
     подъёма языка (открытые, или ``широкие'', закрытые, или ``узкие'', гласные)
    \item по работе губ: различают огубленные и неогубленные гласные 
    (в некоторых случаях и разные степени огубленности)
    \item по работе нёбной занавески (неносовые и носовые)
    \item по долготе (долгие и краткие)
    \item по наличию дифтонгизации
\end{itemize}

\subsubsection{Классификация согласных}
\begin{itemize}
    \item по соотношению шума и голоса (сонаты, шумные)
    \item по характеру преграды (смычные, щелевые и дрожащие)
    \item по активному артикулирующему органу (губные, переднеязычные, 
    среднеязычные, заднеязычные, увулярные, фарингинальные и гортанные)
    \item по пассивному артикуляционному органу
\end{itemize}

\subsection{Изменения в потоке речи}
\begin{enumerate}
    \item \textbf{Комбинаторные изменения} -- вызванные взаимодействием между 
    артикуляциями звуков, находящимися в потоке речи в непосредственном или 
    близком соседстве.
    \item \textbf{Собственно позиционные изменения}, связанные с положением
    звука в безударном слоге либо на конце слова, и т.д.
\end{enumerate}

\subsubsection{Комбинаторные изменения}
\begin{itemize}
    \item \textbf{Аккомодация} -- 
    \item \textbf{Ассимиляция} -- качественное уподобление одного звука другому.
    \item \textbf{Диссимиляция} -- явления, противоположные ассимиляции.
\end{itemize}

\subsubsection{Собственно позиционные изменения}
\begin{itemize}
    \item \textbf{Редукция} (ослабление) -- гласных в безударных слогах и 
    оглушение согласных на...?
    \item Оглушение звонких согласных на конце слова перед паузой 
    наблюдается во многих языках.
\end{itemize}

\subsection{Теория фонем}
\textbf{Фонема} -- минимальная смыслоразличительная единица языка. 
Фонема не имеет самостоятельного лексического или грамматического значения,
но служит для различения и отождествления значимых единиц языка (морфем и слов):

\subsection{Слог}
\textbf{Слог} -- минимальная произносительная (артикуляционная) единица речи.

Чаще всего вершину или ядро слога образует тот или иной гласный звук, а на периферии
слога располагаются согласные.

\subsection{Ударение}
\begin{itemize}
    \item Ударные слоги может производиться с большей интенсивностью -- 
    так называемое \textbf{динамическое} или силовое, ударение.
    \item Он может удлиняться квантитативное
    \item музыкальное
    \item качественное
\end{itemize}

Словесное ударение может быть свободном (разноместным) или связанным 
(фиксированным, одноместном).

\subsection{Транскрипция}
\textbf{Транскрипция} -- передача на письме тем или иным набором 
письменных знаков (фонетическим алфавитом) элементов звучащей речи.

\subsection{Интонация}
\begin{itemize}
    \item Мелодика
    \item Интенсивность
    \item Темп речи
    \item Паузы
    \item Тембр
\end{itemize}

д.з.

Знать что такое: 
\begin{itemize}
    \item Генеалогическая классификация (индоевропейская семья, 
    германская группа языков, славянская группа языков)
    \item Лингвистическая типология (типы языков)
    \item Прочитать главу II из учебника (Баранникова Л.И. Введение 
    в языкознание. - Саратов, 1973). Особое внимание - стр. 53 и 55.
\end{itemize}

\section{Лексика}
-- это совокупность слов языка, его словарный запас.

\section{Семантика}
\textbf{Семантика} -- это содержание, информация, передаваемая языком или какой-либо его
отдельной единицей (морфемой, словом, словосочетанием, предложением),
а также наука, изучающая содержание языковых выражений.

\subsection{Лексема}
\textbf{Лексема} -- это слово, рассматриваемое как единица словарного состава языка
в совокупности его конкретных грамматических форм и выражающих их флексий.

\subsection{Словоформа}
\textbf{Словоформа} -- это слово в некоторой грамматической форме.

\textbf{Языковое значение} -- это закрепленное за данной единицей языка
относительно стабильное во времени и инвариантное содержание, знание которого
входит в зниние этого языка. Именно значения получают отражение в толковых словарях.

\textbf{Смысл} -- это связанная со словом или другой единицей языка информация, ...

\subsection{Значения единиц: лексическое и грамматическое}

\begin{itemize}
    \item Одно и то же значение в разных языках может быть выражено и лексически, и грамматически.
    \item Грамматические значения относятся к обязательным, лексические -- к необязательным.
    \item Грамматические значения имеют стандартный способ выражения.
\end{itemize}

\subsection{}
Лексико-семантическая информация, содержащаяся в лексеме, представляет собой совокупность нескольких компонентов (слоев)

\begin{itemize}
    \item \textbf{Денотативное значение} (денотат) -- это передаваемая словом информация о внеязыковой действительности.
    
    Слово дает имя какому-либо явлению в самом широком смысле (предмету, признаку, действию и т.д.).

    \textbf{Денотат} -- это обобщенный, целостный образ типичного, эталонного представителя
    класса объектов, который охватывает объем понятия (например, обобщенный образ яблока, стола).
    \item \textbf{Сигнификативное значение} (сигнификат) -- это информация о способе, посредством которого
    объект реальной действительности отражается в сознании говорящего.

    Между значением слова и научным понятием существует ряд принципиальных отличий. Понятие
    имеет общечеловеческий характер, отображает общие и существенные признаки явлений действительности
    и обязательно выражено в слове. Слова же имеют национальный характер, причем часть из них
    может не выражать понятия.

    \item \textbf{Прагматическое значение} -- это информация об условиях употребления слова,
    т.е. о многообразных аспектах коммуникативной ситуации, в которых они используются.

    Может непосредственно в лексическое значение не входить -- несущественные, но устойчивые признаки выражаемого словом понятия.
    \item \textbf{Синтаксическое значение} включает синтаксическую информацию (модель управления, сочетаемость и др.)
    
    Значение других слов может реализоваться только в определенном синтаксическом окружении
    или в определенной синтаксической функции. Так, значение 'часть предмета' у слов
    \textit{носик}, \textit{ручка}, \textit{ножка} вне контекста реализуется в сочетании
    с существительными.
\end{itemize}

\subsection{}
Для обозначения минимальной единицы значения используется целый ряд терминов:
\textit{сема}, \textit{семантический}, \textit{дифференциальный признак},
\textit{семантический множитель}, \textit{семантический примитив}, \textit{атом смысла},
\textit{фигура содержания}.

\subsection{Архисемы (гиперсемы, нитегральные семы)}
\textbf{Архисемы} -- это центральные, иерархически главные семы в структуре значения,
они являются общими (родовыми, интегрирующими) для целого класса единиц.

\subsection{Дифференциальные семы}
\textbf{Дифференциальные семы} -- это видовые, различительные семы, с их помощью описываются различия
единиц, имеющих одну архисему.

\subsection{Полисемия (многозначность)}
\textbf{Полисемия} -- это наличие у слова более одного значения, иными словами, наличие у лексемы
нескольких лексико-семантических вариантов.

\textbf{Омонимией} называется совпадение формы (устной или письменной) ...

\end{document}