\documentclass{article}
\usepackage[utf8]{inputenc}
\usepackage[russian]{babel}
\usepackage{hyperref}
\usepackage{underscore}
\usepackage{setspace}
\usepackage{indentfirst} 
\usepackage{mathtools}
\usepackage{amsfonts}
\usepackage{enumitem}
\usepackage{amsthm}
\usepackage{cancel}
\usepackage[left=7cm,right=7cm,
    top=2cm,bottom=2cm,bindingoffset=0cm]{geometry}
\singlespacing

\usepackage{graphicx}
\graphicspath{ {./images/matan/} }

\newcommand{\bydef}{\stackrel{\text{по опр.}}{\implies}} % by definition - по определению
\newcommand{\parspace}{\vspace{20pt}}

\newcommand{\imagin}{\mathrm{Im} \,}
\newcommand{\real}{\mathrm{Re} \,}

\newcommand{\dslim}{\displaystyle\lim}
\newcommand{\dslimn}{\dslim_{n \to \infty}}

\newcommand{\prop}[1]{#1^{\text{o}}}

\newcommand{\N}{\mathbb{N}}
\newcommand{\R}{\mathbb{R}}
\newcommand{\bb}[1]{\mathbb{#1}}

\newcommand{\approach}[1]{\underset{#1}{\longrightarrow}}

\newtheoremstyle{break}
  {\topsep}{\topsep}%
  {\itshape}{}%
  {\bfseries}{}%
  {\newline}{}%
\theoremstyle{break}
\newtheorem{theorem}{Теорема}[subsection]

\newtheorem{definition}{Определение}[subsection]

\begin{document}

\section{}

\[ l_1: \frac{x - x_1}{a^x} = \frac{y - y_1}{a^y} = \frac{z - z_1}{a^z} \]
\[ l_2: \frac{x - x_1}{b^x} = \frac{y - y_1}{b^y} = \frac{z - z_1}{b^z} \]

\parspace

Уравнение прямой, проходящей через данную точку $M_0(x_0, y_0, z_0)$ параллельно
данному вектору $a(a^x; a^y; a^z)$:

\[\frac{x - x_0}{a^x} = \frac{y - y_0}{a^y} = \frac{z - z_0}{a^z} \]

\parspace

Уравнение плоскости, проходящей через три данные точки
$M_1(x_1, y_1)$, $M_2(x_2, y_2)$, $M_3(x_3, y_3)$:

\[ \left| \begin{matrix}
  x - x_1 & y - y_1 & z - z_1 \\
  x - x_2 & y - y_2 & z - z_2 \\
  x - x_3 & y - y_3 & z - z_3 
\end{matrix} \right| = 0 \]

\parspace

Нахождение синуса угла $\varphi$ между прямой и плоскостью в пространстве:

\[ \sin \varphi = \pm
\frac{A a^x + B a^y + C a^z}
{\sqrt{(a^x)^2 + (a^y)^2 + (a^z)^2} \sqrt{A^2 + B^2 + C^2}} \]

\parspace

Две прямые лежат в одной плоскости тогда и только тогда, когда

\[ \left| \begin{matrix}
  x_2 - x_1 & y_2 - y_1 & z_2 - z_1 \\
  a^x & a^y & a^z \\
  b^x & b^y & b^z
\end{matrix} \right| = 0 \]

\parspace

Расстояние от точки $M_0(x_0, y_0)$ до прямой $Ax + By + C = 0$:

\[ \rho = \frac{| Ax_0 + By_0 + C |}{\sqrt{A^2 + B^2}} \]

\parspace

Уравнение прямой в пространстве, проходящей через две данные точки:

\[ \frac{x - x_1}{x_2 - x_1} = \frac{y - y_1}{y_2 - y_1} = \frac{z - z_1}{z_2 - z_1} \]

\end{document}