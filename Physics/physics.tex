\documentclass[a4paper,oneside]{article}
\usepackage[utf8]{inputenc}
\usepackage[russian]{babel}
\usepackage{hyperref}
\usepackage{underscore}
\usepackage{setspace}
\usepackage{indentfirst} 
\usepackage{mathtools}
\usepackage{amsfonts}
\usepackage{enumitem}
% \usepackage[standard]{ntheorem}
\usepackage{amsthm}
\usepackage{cancel}
\usepackage[left=1.4cm,right=1.4cm,
    top=2.3cm,bottom=2.3cm,bindingoffset=0cm]{geometry}
\singlespacing

\usepackage{graphicx}
\graphicspath{ {./images/} }

\usepackage{fancyhdr}
\pagestyle{fancy}

\usepackage{tikz}

% \theoremstyle{break}

% --- Теорема --- %
% \newtheoremstyle{break}% name
%   {}%         Space above, empty = `usual value'
%   {}%         Space below
%   {\itshape}% Body font
%   {}%         Indent amount (empty = no indent, \parindent = para indent)
%   {\bfseries}% Thm head font
%   {.}%        Punctuation after thm head
%   {\newline}% Space after thm head: \newline = linebreak
%   {}%         Thm head spec
% \theorembodyfont{\normalfont}
% \theoremstyle{break}
\newtheorem{theorem}{Теорема}[subsection]
% --------------- %

% --- Определение --- %
% \theorembodyfont{\normalfont}
% \newtheoremstyle{break}
%   {\topsep}{\topsep}%
%   {\itshape}{}%
%   {\bfseries}{}%
%   {\newline}{}%
% \theoremstyle{break}
\theoremstyle{definition}
\newtheorem{definition}{Определение}[subsection]
% ------------------- %

% --- Доказательство --- %
% \theoremheaderfont{\normalfont\itshape}
% \theorembodyfont{\normalfont}
% \newtheorem*{proof}{Доказательство.}
% ---------------------- %

% --- Пример --- %
\theoremstyle{definition}
\newtheorem*{example}{Пример}
% -------------- %

% --- Следствие --- %
% Corollary
% ----------------- %

% --- Замечание --- %
% Remark
\theoremstyle{definition}
\newtheorem*{remark}{Замечание}
% ----------------- %
\newcommand{\tbf}[1]{\textbf{#1}}
\newcommand{\dslim}{\displaystyle\lim}
\newcommand{\dslimn}{\dslim_{n \to \infty}}
\newcommand{\approach}[1]{\underset{#1}{\longrightarrow}}

\begin{document}
% \section{Кинематика}
% \subsection{Перемещение точки. Путь.}
% \begin{definition}
%     Материальная точка при движении описывает некоторую линию - \tbf{траекторию}.
    
%     \tbf{Путь} - длина траектории.

%     \tbf {Перемещение} - расстояние между началом и концом точек траектории.
% \end{definition}

% \subsection{Скорость.}
% $\vec{V} = \dslim_{\Delta \to 0} \frac{\Delta \vec{r}}{\Delta t} = \frac{d \vec{r}}{dt}$

% $\vec{V}$ совпадает с $\vec{r}$ в каждый момент времени.

\section{Механика.}
\subsection{Кинематика: криволинейное движение.}
    \tbf{Криволинейное движение} - движение вдоль кривой линии.
\tbf{*Рисунок*}

$R_C$ - радиус кривизны. $C = \frac{1}{R_C}$

\tbf{*Рисунок*}

1-2 $\rightarrow \Delta S$. $\dslim_{\Delta S \to 0} \frac{\Delta \varphi} {\Delta S} = C$. $\frac {d \varphi} {dS} = C$.

Для описания движения вводим криволинейную систему координат, то есть вводим координату, которая будет отмечать
пройденное расстояние - $\varepsilon$

\tbf{*Рисунок*}

Для описания движения необходимо в каждой точке ввести нормальную и тангенциальную компоненту скорости.

\tbf{*Рисунок*}

$| \vec{\tau} | = 1$, $| \vec{n} | = 1$
\[
\begin{cases}
    \vec{V_n} - \text{нормальная скорость} \\
    \vec{V_\tau} - \text{тангенциальная скорость}
\end{cases}
\]    
Отсюда следует, что существует нормальное и тангенциальное ускорение:
\[
\begin{cases}
    \vec{a_n} - \text{нормальное ускорение} \\
    \vec{a_\tau} - \text{тангенциальное ускорение}
\end{cases}
\]

Кинетическое уравнение криволинейного движения имеет вид: $\xi = \xi_0 + V_0t + \frac {at^2} {2}$

Нормальное ускорение всегда направлено к центру кривизны и определяется как $a_n = \frac {V^2} {R_C}$

\subsubsection{Криволинейное движение по окружности (вращательное движение)}

\tbf{*Рисунок*}

\tbf{Аксиальные} и \tbf{полярные} векторы. Полярные это те, направление которых определяется естественным образом
в данных условиях. Например: $\vec{V}, \vec{x}, \vec{a}$. Аксиальные вводятся для удобства
описания вращательного движения. Например: $\vec{\varphi}, \vec{\omega}, \vec{\varepsilon}$.

$[ \omega ]$ - угловая скорость $\frac{\text{рад}} {c}$

$[ \varphi ]$ - $\text{рад}$; $[ \varepsilon ]$ - $\frac{\text{рад}} {c^2}$

За положительное направление принято направление против часовой стрелки и $\varphi$ на оси 
располагается так (рисунок).

$\vec{\omega} = \frac {d\varphi} {dt}$, $\vec{\varepsilon} = \frac{d\vec{\omega}} {dt}$.

Кинематическое уравнение вращательного движения: $\varphi = \varphi_0 + \omega_0t + \frac {\varepsilon t^2} {2}$

$\vec{V} = [ \vec{\omega} * \vec{R} ]$, $a_n = \omega^2R$

\subsection{Динамика}
Динамика изучает закономерности движения с одновременным рассмотрением причины этого движения.

\tbf{Масса} - мера инертности. \tbf{Сила} - мера интенсивности взаимодействия.

Радиус-вектор задаёт положение в пространстве относительно фиксированного центра

\tbf{*Рисунок*}

Введём углы между каждой осью и радиус вектором.
\[
\begin{cases}
    \text{Проекция } \vec{r_x} = r \cos(\alpha) \\
    \text{Проекция } \vec{r_y} = r \cos(\beta) \\
    \text{Проекция } \vec{r_z} = r \cos(\gamma)
\end{cases}
\]

Эти косинусы называются направляющими. $\vec{r} = r (i\cos(\alpha) + j\cos(\beta) + k\cos(\gamma))$

\subsubsection{Законы классической механики}

\tbf{I Закон Ньютона}. Если на тело не действует силы или равнодействующая всех сил равна нулю,
тело находится в покое или движется прямолинейно и равномерно. Подобные системы называются \textit{инерциальными}.

\tbf{II Закон Ньютона}. Скорость изменения импульса определяется силой, действующей на него. $\frac{d \vec{P}} {dt} = \vec{F}$.
Если m = const, то $m\vec{a}=\vec{F} \implies \vec{a}=\frac{\vec{F_p}} {m}$

\tbf{III Закон Ньютона} В инерциальной системе отсчёта все тела взаимодействуют с одинаковыми по величине силам
и противоположными по знаку.

Следствия - закон сохранения механической энергии и сохранения импульса для замкнутых систем.


\subsubsection{Работа силы. Механическая энергия}
При движении под действием некоторой силы, этой силой совершается работа.

$dA = \vec{F} d \vec{S}$, $A = \int_1^2 \vec{F} d \vec{S}$, $d\vec{S}$ - дифференциально малое перемещение. 

Из \textit{II закона Ньютона} запишем силу и подставим: 
$A = \int_1^2 d \vec{P} * \frac{d \vec{S}} {dt} = \int_1^2 \vec{V} d \vec{P} = m \int_1^2 \vec{V} d \vec{V}$.

$A = \frac{m V_2^2}{2} - \frac{m V_1^2}{2} = K_2 - K_1 (\text{приращение кинетической энергии})$.

Рассмотрим работу, совершаемую силой тяжести:

\tbf{*Рисунок*}

$A = mg(h_1 - h_2) = U_1 - U_2$.

Потенциальная энергия определяет положение в пространстве. В отсутствие силы трения $K + U = const$. 
Работа силы трения всегда меньше нуля.

\subsubsection{Поле центральной силы}
Если в любой точке пространства сила направлена к одному и тому же центру или от него,
сила называется \tbf{центральной}.

\tbf{*Рисунок*}

$A = \int_1^2 \vec{F} d \vec{r}$

$\vec{r} \uparrow \uparrow \vec{F}$

$A = \int_1^2 Fdr$

В поле центральной силы работа определяется радиус-вектором начального и конечного положений.
Работа силы по замкнутому контуру равна 0.


\end{document}