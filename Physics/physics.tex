\documentclass[a4paper,oneside]{article}
\usepackage[utf8]{inputenc}
\usepackage[russian]{babel}
\usepackage{hyperref}
\usepackage{underscore}
\usepackage{setspace}
\usepackage{indentfirst} 
\usepackage{mathtools}
\usepackage{amsfonts}
\usepackage{enumitem}
% \usepackage[standard]{ntheorem}
\usepackage{amsthm}
\usepackage{cancel}
\usepackage[left=1.4cm,right=1.4cm,
    top=2.3cm,bottom=2.3cm,bindingoffset=0cm]{geometry}
\singlespacing

\usepackage{graphicx}
\graphicspath{ {./images/} }

\usepackage{fancyhdr}
\pagestyle{fancy}

\usepackage{tikz}

% \theoremstyle{break}

% --- Теорема --- %
% \newtheoremstyle{break}% name
%   {}%         Space above, empty = `usual value'
%   {}%         Space below
%   {\itshape}% Body font
%   {}%         Indent amount (empty = no indent, \parindent = para indent)
%   {\bfseries}% Thm head font
%   {.}%        Punctuation after thm head
%   {\newline}% Space after thm head: \newline = linebreak
%   {}%         Thm head spec
% \theorembodyfont{\normalfont}
% \theoremstyle{break}
\newtheorem{theorem}{Теорема}[subsection]
% --------------- %

% --- Определение --- %
% \theorembodyfont{\normalfont}
% \newtheoremstyle{break}
%   {\topsep}{\topsep}%
%   {\itshape}{}%
%   {\bfseries}{}%
%   {\newline}{}%
% \theoremstyle{break}
\theoremstyle{definition}
\newtheorem{definition}{Определение}[subsection]
% ------------------- %

% --- Доказательство --- %
% \theoremheaderfont{\normalfont\itshape}
% \theorembodyfont{\normalfont}
% \newtheorem*{proof}{Доказательство.}
% ---------------------- %

% --- Пример --- %
\theoremstyle{definition}
\newtheorem*{example}{Пример}
% -------------- %

% --- Следствие --- %
% Corollary
% ----------------- %

% --- Замечание --- %
% Remark
\theoremstyle{definition}
\newtheorem*{remark}{Замечание}
% ----------------- %
\newcommand{\tbf}[1]{\textbf{#1}}
\newcommand{\dslim}{\displaystyle\lim}
\newcommand{\dslimn}{\dslim_{n \to \infty}}
\newcommand{\approach}[1]{\underset{#1}{\longrightarrow}}
\newcommand{\dsint}{\displaystyle\int}

\begin{document}
% \section{Кинематика}
% \subsection{Перемещение точки. Путь.}
% \begin{definition}
%     Материальная точка при движении описывает некоторую линию - \tbf{траекторию}.
    
%     \tbf{Путь} - длина траектории.

%     \tbf {Перемещение} - расстояние между началом и концом точек траектории.
% \end{definition}

% \subsection{Скорость.}
% $\vec{V} = \dslim_{\Delta \to 0} \frac{\Delta \vec{r}}{\Delta t} = \frac{d \vec{r}}{dt}$

% $\vec{V}$ совпадает с $\vec{r}$ в каждый момент времени.

\section{Механика}
\subsection{Кинематика: криволинейное движение}
    \tbf{Криволинейное движение} - движение вдоль кривой линии.

\tbf{*Рисунок*}

\[R_C - \text{радиус кривизны. } C = \frac{1}{R_C}\]

\tbf{*Рисунок*}

1-2 $\rightarrow \Delta S$. $\dslim_{\Delta S \to 0} \frac{\Delta \varphi}{\Delta S} = C$. $\frac {d \varphi}{dS} = C$.

Для описания движения вводим криволинейную систему координат, то есть вводим координату, которая будет отмечать
пройденное расстояние -- $\varepsilon$

\tbf{*Рисунок*}

Для описания движения необходимо в каждой точке ввести нормальную и тангенциальную компоненту скорости.

\tbf{*Рисунок*}

$| \vec{\tau} | = 1$, $| \vec{n} | = 1$
\[
\begin{cases}
    \vec{V_n} - \text{нормальная скорость} \\
    \vec{V_\tau} - \text{тангенциальная скорость}
\end{cases}
\]    
Отсюда следует, что существует нормальное и тангенциальное ускорение:
\[
\begin{cases}
    \vec{a_n} - \text{нормальное ускорение} \\
    \vec{a_\tau} - \text{тангенциальное ускорение}
\end{cases}
\]

Кинетическое уравнение криволинейного движения имеет вид: $\xi = \xi_0 + V_0t + \frac {at^2}{2}$

Нормальное ускорение всегда направлено к центру кривизны и определяется как $a_n = \frac {V^2}{R_C}$

\subsubsection{Криволинейное движение по окружности (вращательное движение)}

\tbf{*Рисунок*}

\tbf{Аксиальные} и \tbf{полярные} векторы. Полярные это те, направление которых определяется естественным образом
в данных условиях. Например: $\vec{V}, \vec{x}, \vec{a}$. Аксиальные вводятся для удобства
описания вращательного движения. Например: $\vec{\varphi}, \vec{\omega}, \vec{\varepsilon}$.

$[ \omega ]$ -- угловая скорость $\frac{\text{рад}}{c}$

$[ \varphi ]$ -- $\text{рад}$; $[ \varepsilon ]$ -- $\frac{\text{рад}}{c^2}$

За положительное направление принято направление против часовой стрелки и $\varphi$ на оси 
располагается так (рисунок).

$\vec{\omega} = \frac {d\varphi}{dt}$, $\vec{\varepsilon} = \frac{d\vec{\omega}}{dt}$.

Кинематическое уравнение вращательного движения: $\varphi = \varphi_0 + \omega_0t + \frac {\varepsilon t^2}{2}$

$\vec{V} = [ \vec{\omega} * \vec{R} ]$, $a_n = \omega^2R$

\subsection{Динамика}
Динамика изучает закономерности движения с одновременным рассмотрением причины этого движения.

\tbf{Масса} -- мера инертности. \tbf{Сила} -- мера интенсивности взаимодействия.

Радиус-вектор задаёт положение в пространстве относительно фиксированного центра

\tbf{*Рисунок*}

Введём углы между каждой осью и радиус вектором.
\[
\begin{cases}
    {r_x} = r \cos(\alpha) \\
    {r_y} = r \cos(\beta) \\
    {r_z} = r \cos(\gamma)
\end{cases}
\]

Эти косинусы называются направляющими. $\vec{r} = r (\vec{i}\cos(\alpha) + \vec{j}\cos(\beta) + \vec{k}\cos(\gamma))$

\subsubsection{Законы классической механики}

\tbf{I Закон Ньютона}. Если на тело не действует силы или равнодействующая всех сил равна нулю,
тело находится в покое или движется прямолинейно и равномерно. Подобные системы называются \textit{инерциальными}.

\tbf{II Закон Ньютона}. Скорость изменения импульса определяется силой, действующей на него. $\frac{d \vec{P}}{dt} = \vec{F}$.
Если m = const, то $m\vec{a}=\vec{F} \implies \vec{a}=\frac{\vec{F_p}}{m}$

\tbf{III Закон Ньютона} В инерциальной системе отсчёта все тела взаимодействуют с одинаковыми по величине силам
и противоположными по знаку.

Следствия -- закон сохранения механической энергии и сохранения импульса для замкнутых систем.


\subsubsection{Работа силы. Механическая энергия}
При движении под действием некоторой силы, этой силой совершается работа.

$dA = \vec{F} d \vec{S}$, $A = \dsint_1^2 \vec{F} d \vec{S}$, $d\vec{S}$ -- дифференциально малое перемещение. 

Из \textit{II закона Ньютона} запишем силу и подставим: 
$A = \dsint_1^2 d \vec{P} * \frac{d \vec{S}}{dt} = \dsint_1^2 \vec{V} d \vec{P} = m \dsint_1^2 \vec{V} d \vec{V}$.

$A = \frac{m V_2^2}{2} - \frac{m V_1^2}{2} = K_2 - K_1 (\text{приращение кинетической энергии})$.

Рассмотрим работу, совершаемую силой тяжести:

\tbf{*Рисунок*}

$A = mg(h_1 - h_2) = U_1 - U_2$.

Потенциальная энергия определяет положение в пространстве. В отсутствие силы трения $K + U = const$. 
Работа силы трения всегда меньше нуля.

\subsubsection{Поле центральной силы}
Если в любой точке пространства сила направлена к одному и тому же центру или от него,
сила называется \tbf{центральной}.

\tbf{*Рисунок*}

$A = \dsint_1^2 \vec{F} d \vec{r}$

$\vec{r} \uparrow \uparrow \vec{F}$

$A = \dsint_1^2 Fdr$

В поле центральной силы работа определяется радиус-вектором начального и конечного положений.
Работа силы по замкнутому контуру равна 0.

\subsubsection{Профиль функции потенциальной энергии}
Профиль -- график изменения функции с изменением координаты.

\tbf{*Рисунок*}

Если объект находится на промежутке $[a;b]$, то для данного случая энергия не превышает $U_1$, и, таким образом, объект находится
в потенциальной яме, так как при движении по $x$ или в противоположную сторону энергия должна увеличиваться,
а её значение не больше $U_1$. Движения являтся финитным, то есть ограниченным.

\begin{enumerate}
    \item $x \in (a;b)$ -- потенциальная яма. Для этого объекта $x < a$ и $x > b$ - потенциальные барьеры.
    \item $x \geq c$ движение называется инфинитным

    $x < c$ -- потенциальный барьер
    \item $x \in (b;c)$, $x < a$ -- потенциальный барьер
\end{enumerate}
\subsubsection{Взаимосвязь силы и потенциальной энергии}
\tbf{*Рисунок*}

Направление максимального изменения функции задаётся её градиентом $\text{grad } U(x,y,z)$.
Градиент -- оператор, который скалярную функцию переводит в векторную, и этот вектор показывает направление максимального изменения.

\[\text{grad } U(x,y,z) = \vec{i} \frac{dU}{dx} + \vec{j} \frac{dU}{dy} + \vec{k} \frac{dU}{dz}\].

\[\vec{F_{\text{т}}} = -\frac{dU} {dz} \vec{k} \text{; } \vec{F} = -\text{grad } U(x,y,z)\].

\subsubsection{Движение в неинерциальных системах отсчёта}
Неинерциальные системы отсчёта двигаются произвольным образом. Произвольное движение можно представить как совокупность
вращательного движения относительно мгновенной оси и линейного.

\tbf{*Рисунок*}

$\vec{R} = \vec{R_0} + \vec{r}$

\tbf{Задача}: составить уравнение относительного движения для объекта с массой $m$.

$\vec{R_0}$ радиус-вектор, задающий положение точки $O$, и скорость её движения не учитывает возможный 
поворот неинерциальной системы. В каждый момент времени необходимо вводить и учитывать мгновенную ось поворота.
\begin{enumerate} 
    \item Найдём относительную скорость движения.

Продифференцируем $\vec{R} = \vec{R_0} + \vec{r}$:

\[\frac{dR}{dt} = \frac{dR_0}{dt} + \frac{d}{dt} (\vec{i}x + \vec{j}y + \vec{k}z)\]

Из векторного анализа известно, что при уголовой скорости $\omega$ производная вектора есть векторное произведение
на соответсвующий вектор координаты:

\[\vec{V_0} +([ \vec{\omega} * \vec{x} ] + [ \vec{\omega} * \vec{y} ] + [ \vec{\omega} * \vec{z} ])
 + (\vec{i} \frac {dx}{dt} + \vec{j} \frac {dy}{dt} + \vec{k} \frac {dz}{dt})\]

 \[\vec{V_\text{абс}} = \vec{V_0} + [\vec{\omega} * \vec{r}] + \vec{V_\text{отн}}\]

 Движение относительно инерциальной системы отсчёта будем называть \textit{абсолютным}, и это движение складывается
 из относительного и переносного.

 
    \item Поиск ускорения абсолютного движения: $\vec{a_{\text{абс}}}$.

 \[\frac{d \vec{r}}{dt} = [\vec{\omega} * \vec{r}] + \vec{V_{\text{отн}}}\]

 \[ \frac{d \vec{V_\text{абс}}}{dt} = \frac{d \vec{V_0}}{dt} + 
 ([\dot{\vec{\omega}} * \vec{r}] + [\vec{\omega} + \dot{\vec{r}}]) + \frac{d}{dt} (\vec{i} V_x + \vec{j} V_y + \vec{k} V_z) = \] 
\[= \vec{a_0} + ([\dot{\vec{\omega}} * \vec{r}] + ([\vec{\omega} * ([\vec{\omega} * \vec{r}] + \vec{V_\text{отн}})]) + 
([\vec{\omega} * \vec{V_\text{отн}}] + \vec{a_\text{отн}}) = \] 
\[ = \vec{V_0} + [\dot{\vec{\omega}} * \vec{r}] + [\vec{\omega} * [\vec{\omega} * \vec{r}]] + [\vec{\omega} * \vec{V_\text{отн}}] +
[\vec{\omega} * \vec{V_\text{отн}}] + \vec{a_\text{отн}} \]
\[ \vec{a_\text{абс}} = \vec{a_\text{отн}} + [\dot{\vec{\omega}} * \vec{r}] + [\vec{\omega} * [\vec{\omega} * \vec{r}]] +
\dot{\vec{V_0}} + 2[\vec{\omega} * \vec{V_\text{отн}}] \]
\end{enumerate}

\tbf{Теорема Кориолиса}. 

Абсолютное ускорение есть сумма относительного ускорения, переносного и кориолисова.

Переносное ускорение включает в себя:
\begin{itemize}
    \item Линейное ускорение ($\dot{\vec{V_0}}$)
    \item Угловое ускорение ($\dot{\vec{\omega}}$)
    \item Центробежное ускорение ($[\vec{\omega} * [\vec{\omega} * \vec{r}]] = -\vec{\omega^2} \vec{r_\perp}$,
     где $\vec{r_\perp}$ -- радиус-вектор объекта относительно оси поворота)
\end{itemize}

\tbf{*Рисунок*}

Составим уравнение движения в неинерциальной системе отсчёта:

\[ \vec{a_\text{отн}}m = \vec{F} + \vec{\omega^2} \vec{r_\perp}m - m[\dot{\vec{\omega}} * \vec{r}] - m\dot{\vec{V_0}} + 2m[\vec{V_\text{отн}} * \vec{\omega}] \]
Все остальные слагаемые называются силами инерции:
\begin{itemize}
    \item $\vec{\omega^2} \vec{r_\perp}m$ -- центробежная сила;
    \item $ -m[\dot{\vec{\omega}} * \vec{r}]$ -- сила инерции при ускоренном повороте;
    \item $ -m\dot{\vec{V_0}}$ -- сила инерции из-за ускорения центра неинерциальной системы отсчёта;
    \item $ \vec{F_\text{Кор}} = 2m[\vec{V_\text{отн}} * \vec{\omega}]$.
\end{itemize}

Рассмотрим движение объекта во вращательной системе. В качестве вращательной системы --
вращение Земли, а движущийся объект -- река. 

\tbf{*Рисунок*}

Сила Кориолиса не может быть скомпенсирована и играет формирующую роль.
\end{document}