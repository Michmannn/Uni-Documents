\section{Теория множеств}

\subsection{Множества и отношения между ними}

Множество "--- совокупность различаемых объектов.
Объекты называются его элементами.

Два множества равны ($A = B$), если:

\begin{equation*}
     \forall x: (x \in A \Leftrightarrow x \in B)
\end{equation*}

Множество $A$ включается в множество $B$ ($A \subseteq B$), если:

\begin{equation*}
    \forall x: (x\in A \Rightarrow x \in B)
\end{equation*}

Множество, $A$ строго включается в множество $B$ ($A \subset B$), если:

\begin{equation*}
    \forall x: (A \subseteq B ~ \& ~ A \neq B)
\end{equation*}

Множество называется пустым, если оно не содержит элементов.
Такое множество обозначается как $\varnothing$.

\begin{equation*}
    \forall A \neq \varnothing: ~ \varnothing \subseteq A
\end{equation*}

Некоторое множество $\Omega \neq \varnothing$ назовем универсальным множеством
и скажем, что

\begin{equation*}
    \forall A \neq \varnothing: ~ A \subseteq \Omega
\end{equation*}

\subsection{Операции над множествами}

Пусть $A, B \subseteq \Omega$:

\begin{equation*}
    A \cup B = \{ \forall~ x\in \Omega: ~ x \in A \logor x \in B \} \quad \text{(объединение)}
\end{equation*}

\begin{equation*}
    A\cap B = \{ \forall ~ x \in \Omega: ~ x \in A \logand x \in B \} \quad \text{(пересечение)}
\end{equation*}

\begin{equation*}
    \overline{A} = \{ \forall ~ x \in \Omega: ~ x \notin A \} \quad \text{(дополнение)}
\end{equation*}

\begin{equation*}
    A \setminus B = \{ \forall ~ x \in \Omega: ~ x\in A \logand x \notin B \} = 
    A \cup \overline{B}  \quad \text{(разность)} 
\end{equation*}

\begin{equation*}
    A \triangle B = \{ \forall ~ x \in \Omega: ~ x\in A \logand x\notin B \logor x \notin A 
    \logand x \in B \} \quad \text{(симметричная разность)}
\end{equation*}

Множество, содержащее конечное число элементов называется конечным.

Пусть $ A \neq \varnothing$ и $A$ состоит из $n$ элементов, тогда $|A| = n$ "---
мощность этого множества.

Множество при этом также обозначается как $A = \{ a_1, a_2, \dots, a_n\}$ или $A = \{x ~|~ P(x) \}$

Примеры:

\begin{equation*}
    A = \{ x \in \R ~|~ x\geq 0 \logand x \leq 1 \}
\end{equation*}

\subsection{Характеристические векторы множеств}

Пусть есть некоторое конечное множество $\Omega \neq \varnothing$ из $n$ элементов, 
$\Omega = \{ x_1, x_2, \dots, x_n\}$

Для удобства будем считать, что порядок перечисления множеств зафиксирован.

Теперь пусть $A \subseteq \Omega$.

Характеристическим вектором $A$ назовем булев вектор из $n$ компонентов 

\begin{equation*}
    \chi_A = (\chi_1^A, \chi_2^A, \dots, \chi_n^A), \text{~где } \chi_i^A = 
    \begin{cases}
        1, & x_i \in A \\
        0, & x_i \notin A 
    \end{cases}
\end{equation*}

Пусть $P(A)$ "--- множество всех подмножеств непустого множества $A$. 

Отметим, что $A \in P(\Omega)$

Пример:

$\Omega = \{a, b, c\}$

$P(\Omega) = \{ \varnothing, \{a\}, \{b\}, \{c\}, \{a, b\}, \{b, c\}, \{a, c\}, \{a, b, c\} \}$
\begin{center}
    \begin{tabularx}{0.2\textwidth}{c|c}
        $A$ & $\chi_A$ \\ \hline
        $\varnothing$ & $(0, 0, 0)$ \\
        $\{ a \}$ & $(1, 0, 0)$ \\  
        $\{ b \}$ & $(0, 1, 0)$ \\  
        $\{ c \}$ & $(0, 0, 1)$ \\  
        $\{ a,b \}$ & $(1, 1, 0)$ \\  
        $\{ b,c \}$ & $(0, 1, 1)$ \\  
        $\{ a,c \}$ & $(1, 0, 1)$ \\  
        $\{ a,b,c \}$ & $(1, 1, 1)$ \\  
    \end{tabularx}    
\end{center}

Стоит отметить, что данное отображение является биективным. 

\begin{theorem}[о числе подмножеств конечного множества]
    Число подмножеств $n$"=элементного множества равно $2^n$:

    \begin{equation*}
        |P(\Omega)| = 2^n
    \end{equation*}
    
\end{theorem}

Двоичной алгеброй называется $B_2 = \{ 0, 1\}$ с операциями $\{', +, \cdot\}$, где


\begin{itemize}
    \item <<$'$>> "--- логическое <<НЕ>>
    \item <<$+$>> "--- логическое <<ИЛИ>>
    \item <<$\cdot$>> "--- логическое <<И>>
\end{itemize}


Обозначим $B_2^n$ множество всех двоичных векторов длины $n$.

Пусть $\vec{a} = (a_1, a_2, \dots, a_n), \vec{b} = (b_1, b_2, \dots, b_n) 
\in B_2^n$

\begin{enumerate}
    \item $\vec{a}' = (a_1', a_2', \dots, a_n')$
    \item $\vec{a} + \vec{b} = (a_1 + b_1, a_2 + b_2, \dots, a_n + b_n )$
    \item $\vec{a} \cdot \vec{b} = (a_1 \cdot b_1, a_2 \cdot b_2, \dots, a_n \cdot b_n)$
\end{enumerate}

\begin{theorem}[О характеристических векторах]
    Пусть $\Omega \neq \varnothing$ (универсальное множество).
    Для любых непустых множеств $A, B \subseteq \Omega$ справедливо следующее:
    \begin{enumerate}
        \item $A = B \iff \chi_A = \chi_B$
        \item $\chi_{A\cup B} = \chi_A + \chi_B$
        \item $\chi_{A\cap B} = \chi_A \cdot \chi_B$
        \item $\chi_{\overline{A}} = (\chi_A)'$
        \item $\chi_\varnothing = (0, 0, \dots, 0) = \vec{0}$
        \item $\chi_\Omega = (1, 1, \dots, 1) = \vec{1}$
    \end{enumerate} 
\end{theorem}

Пример:

\begin{example}
Пусть $\Omega = \{ 1, 2, 3, 4, 5\}$, $A = \{1, 3, 4\}$, $B = \{ 1, 4, 5\}$.

Тогда $\chi_A = (1, 0, 1, 1, 0)$, $\chi_B = (1, 0, 0, 1, 1)$.

Получим:

$\chi_{\overline{A}} = (0, 1, 0, 0, 1)$

$\chi_{A \cup B} = (1, 0, 0, 1, 0)$

$\chi_{A \cap B} = (1, 0, 1, 1, 1)$
\end{example}

\subsection{Декартово произведение и отношения между множествами}

Пусть $A, B \neq \varnothing$. Декартовым произведением множеств $A$ и $B$
назовем множество упорядоченных пар 

\begin{equation*}
    A \times B = \{ ~(a, b) ~|~ a \in A \logand b \in B\}
\end{equation*}

$(a_1, b_1) = (a_2, b_2) \iff a_1 = a_2 \logand b_1 = b_2$

Очевидно, что если $|A| = n$, $|B| = m$, то $|A\times B| = n \times m$. Кроме того, $|P(A\times B)| = 2^{nm}$

Бинарным отношением между множествами $A$ и $B$ назовем любое подмножество
декартового произведения $\rho \subseteq |A \times B|$.

Все теоретико"=множественные операции также справедливы для бинарных отношений.








    
