\section{Теория множеств}

\subsection{Множества и отношения между ними}

Множество "--- совокупность различаемых объектов.
Объекты называются его элементами.

Два множества равны ($A = B$), если:

\begin{equation*}
     \forall x: (x \in A \Leftrightarrow x \in B)
\end{equation*}

Множество $A$ включается в множество $B$ ($A \subseteq B$), если:

\begin{equation*}
    \forall x: (x\in A \Rightarrow x \in B)
\end{equation*}

Множество, $A$ строго включается в множество $B$ ($A \subset B$), если:

\begin{equation*}
    \forall x: (A \subseteq B ~ \& ~ A \neq B)
\end{equation*}

Множество называется пустым, если оно не содержит элементов.
Такое множество обозначается как $\varnothing$.

\begin{equation*}
    \forall A \neq \varnothing: ~ \varnothing \subseteq A
\end{equation*}

Некоторое множество $\Omega \neq \varnothing$ назовем универсальным множеством
и скажем, что

\begin{equation*}
    \forall A \neq \varnothing: ~ A \subseteq \Omega
\end{equation*}

\subsection{Операции над множествами}

Пусть $A, B \subseteq \Omega$:

\begin{equation*}
    A \cup B = \{ \forall~ x\in \Omega: ~ x \in A \logor x \in B \} \quad \text{(объединение)}
\end{equation*}

\begin{equation*}
    A\cap B = \{ \forall ~ x \in \Omega: ~ x \in A \logand x \in B \} \quad \text{(пересечение)}
\end{equation*}

\begin{equation*}
    \overline{A} = \{ \forall ~ x \in \Omega: ~ x \notin A \} \quad \text{(дополнение)}
\end{equation*}

\begin{equation*}
    A \setminus B = \{ \forall ~ x \in \Omega: ~ x\in A \logand x \notin B \} = 
    A \cup \overline{B}  \quad \text{(разность)} 
\end{equation*}

\begin{equation*}
    A \triangle B = \{ \forall ~ x \in \Omega: ~ x\in A \logand x\notin B \logor x \notin A 
    \logand x \in B \} \quad \text{(симметричная разность)}
\end{equation*}

Множество, содержащее конечное число элементов называется конечным.

Пусть $ A \neq \varnothing$ и $A$ состоит из $n$ элементов, тогда $|A| = n$ "---
мощность этого множества.

Множество при этом также обозначается как $A = \{ a_1, a_2, \dots, a_n\}$ или $A = \{x ~|~ P(x) \}$

Примеры:

\begin{equation*}
    A = \{ x \in \R ~|~ x\geq 0 \logand x \leq 1 \}
\end{equation*}

\subsection{Характеристические векторы множеств}

Пусть есть некоторое конечное множество $\Omega \neq \varnothing$ из $n$ элементов, 
$\Omega = \{ x_1, x_2, \dots, x_n\}$

Для удобства будем считать, что порядок перечисления множеств зафиксирован.

Теперь пусть $A \subseteq \Omega$.

Характеристическим вектором $A$ назовем булев вектор из $n$ компонентов 

\begin{equation*}
    \chi_A = (\chi_1^A, \chi_2^A, \dots, \chi_n^A), \text{~где } \chi_i^A = 
    \begin{cases}
        1, & x_i \in A \\
        0, & x_i \notin A 
    \end{cases}
\end{equation*}

Пусть $P(A)$ "--- множество всех подмножеств непустого множества $A$. 

Отметим, что $A \in P(\Omega)$

Пример:

$\Omega = \{a, b, c\}$

$P(\Omega) = 
\{ \varnothing, \{a\}, \{b\}, \{c\}, \{a, b\}, \{b, c\}, \{a, c\}, \{a, b, c\} \}$
\begin{center}
    \begin{tabularx}{0.2\textwidth}{c|c}
        $A$ & $\chi_A$ \\ \hline
        $\varnothing$ & $(0, 0, 0)$ \\
        $\{ a \}$ & $(1, 0, 0)$ \\  
        $\{ b \}$ & $(0, 1, 0)$ \\  
        $\{ c \}$ & $(0, 0, 1)$ \\  
        $\{ a,b \}$ & $(1, 1, 0)$ \\  
        $\{ b,c \}$ & $(0, 1, 1)$ \\  
        $\{ a,c \}$ & $(1, 0, 1)$ \\  
        $\{ a,b,c \}$ & $(1, 1, 1)$ \\  
    \end{tabularx}    
\end{center}

Стоит отметить, что данное отображение является биективным. 

\begin{theorem}[о числе подмножеств конечного множества]
    Число подмножеств $n$"=элементного множества равно $2^n$:

    \begin{equation*}
        |P(\Omega)| = 2^n
    \end{equation*}
    
\end{theorem}

Двоичной алгеброй называется $B_2 = \{ 0, 1\}$ с операциями $\{', +, \cdot\}$, где


\begin{itemize}
    \item <<$'$>> "--- логическое <<НЕ>>
    \item <<$+$>> "--- логическое <<ИЛИ>>
    \item <<$\cdot$>> "--- логическое <<И>>
\end{itemize}


Обозначим $B_2^n$ множество всех двоичных векторов длины $n$.

Пусть $\vec{a} = (a_1, a_2, \dots, a_n), \vec{b} = (b_1, b_2, \dots, b_n) 
\in B_2^n$

\begin{enumerate}
    \item $\vec{a}' = (a_1', a_2', \dots, a_n')$
    \item $\vec{a} + \vec{b} = (a_1 + b_1, a_2 + b_2, \dots, a_n + b_n )$
    \item $\vec{a} \cdot \vec{b} = (a_1 \cdot b_1, a_2 \cdot b_2, \dots, a_n \cdot b_n)$
\end{enumerate}

\begin{theorem}[О характеристических векторах]
    Пусть $\Omega \neq \varnothing$ (универсальное множество).
    Для любых непустых множеств $A, B \subseteq \Omega$ справедливо следующее:
    \begin{enumerate}
        \item $A = B \iff \chi_A = \chi_B$
        \item $\chi_{A\cup B} = \chi_A + \chi_B$
        \item $\chi_{A\cap B} = \chi_A \cdot \chi_B$
        \item $\chi_{\overline{A}} = (\chi_A)'$
        \item $\chi_\varnothing = (0, 0, \dots, 0) = \vec{0}$
        \item $\chi_\Omega = (1, 1, \dots, 1) = \vec{1}$
    \end{enumerate} 
\end{theorem}

Пример:

\begin{example}
Пусть $\Omega = \{ 1, 2, 3, 4, 5\}$, $A = \{1, 3, 4\}$, $B = \{ 1, 4, 5\}$.

Тогда $\chi_A = (1, 0, 1, 1, 0)$, $\chi_B = (1, 0, 0, 1, 1)$.

Получим:

$\chi_{\overline{A}} = (0, 1, 0, 0, 1)$

$\chi_{A \cup B} = (1, 0, 0, 1, 0)$

$\chi_{A \cap B} = (1, 0, 1, 1, 1)$
\end{example}

\subsection{Декартово произведение и отношения между множествами}

Пусть $A, B \neq \varnothing$. Декартовым произведением множеств $A$ и $B$
назовем множество упорядоченных пар 

\begin{equation*}
    A \times B = \{ ~(a, b) ~|~ a \in A \logand b \in B\}
\end{equation*}

$(a_1, b_1) = (a_2, b_2) \iff a_1 = a_2 \logand b_1 = b_2$

Очевидно, что если $|A| = n$, $|B| = m$, то $|A\times B| = n \times m$. 
Кроме того, $|P(A\times B)| = 2^{nm}$

Бинарным отношением между множествами $A$ и $B$ назовем любое подмножество
декартового произведения $\rho \subseteq |A \times B|$.

\subsection{Операции над бинарными отношениями}

Все теоретико"=множественные операции также справедливы для бинарных отношений.

\paragraph{Обращение бинарного отношения}

Пусть $\rho \subseteq A \times B$. Обратным отношением назовем 
$\rho^{-1} \subseteq B \times A~:~ \rho^{-1} = \{(b, a) \in B \times A ~|~ (a, b) \in \rho \}$

\begin{example}
    $A = \{1, 2, 3, 4\}$, $B = \{a, b , c\}$
    
    $\rho = \{(1, a), (1, b), (2, c), (4, b), (4, a)\}$
    
    $\rho = \{(a, 1), (b, 1), (c, 2), (b, 4), (a, 4)\}$


\end{example}

\paragraph{Операция умножения бинарных отношений}

Пусть $\rho \subseteq A \times B, \sigma \subseteq B \times C$. 
Определим операцию умножения бинарных отношений $\rho \cdot \sigma = 
\{(a, c) \in A \times C ~|~ \exists b ~:~ (a, b) \in \rho \logand (b, c) \in \sigma\}$

\subsection{Способы задания бинарных отношений}

\subsubsection{С помощью графов}

Пусть $\rho \subseteq A \times B$. $\vec{G}(\rho)$ "--- граф бинарного отношения.

Ребра такого графа "--- элементы множества $A$ и $B$, а ребра "--- ориентированные дуги 
из вершин из множества $A$ в вершины множества $B$.

\subsubsection{Двоичная булева матрица}

Двоичная матрица "--- матрица из $n$ строк и $m$ стобцов.

Определим следующую операцию: $(M)_{il}$ "--- элемент, находящийся на пересечении $i$ строки и $j$ столбца.

Пусть имеются матрицы $N$ и $M$. Определим следующие операции:

\begin{itemize}
    \item Дополнением матрицы называется $M'$, у которой $(M')_{ij} = (M)'_{i, j}$
    \item Суммой двух матриц называется $(M + N)_{A + B}$ такая, что $M_{ij} = N_{ij} + M_{ij}$
    \item Операциией пересечения матриц называется $(M \logand N)_{ij} = M_{ij} \cdot N_{ij}$.
    \item Транспонированием матриц называется $(M^T)_{ij} = M_{ji}$
    \item Пусть существуют матрицы $M_{n \times m}$ и $N_{m \times p}$. Определим произведение
    матриц как $\displaystyle M \cdot N = (M \cdot N)_{ij} = \sum_{k = 1}^m(M_{ik}) * N_{kj}$
\end{itemize}

Пусть множества $A$ и $B$ непустые и $\rho \subseteq A \times B$. 
Матрицей бинарного отношения назовем двоичную матрицу $M(\rho)$ и
определим её следующим образом 
\begin{equation}
    (M(\rho))_{ij} = \begin{cases}
        1, & (a_i, b_j) \in \rho \\
        0, & (a_i, b_j) \notin \rho
    \end{cases}
\end{equation}

Пусть $A$ "--- множество мощности $n$, $B$ "--- множество мощности m.
$A = \{a_1, \dots, a_n\}, B = \{b_1, \dots, b_m\}$.

$M(\rho)$ "--- матрица размерности $n \times m$

\begin{example}
    $A = \{1, 2, 3, 4\}$, $B = \{a, b, c\}$

    $\rho = \{(1, a), (1, b), (3, b), (4, a), (4, c)\}$
\end{example}


\begin{equation*}
    M(\rho) = \left( \begin{matrix}
        1 & 1 & 0 \\
        0 & 0 & 0 \\
        0 & 1 & 0 \\
        1 & 0 & 1 
    \end{matrix} \right)
\end{equation*}

\begin{theorem}{О матрицах бинарных отношений}
    Пусть $A$ и $B$ непустые множества и $\rho, \sigma \subseteq A \times B$,
    $\tau \subseteq B \times C$
    \begin{itemize}
        \item $\rho = \sigma \Leftrightarrow M(\rho) = M(\sigma)$
        \item Матрица дополнения бинарного отношения $M(\rho')
        = (M(\rho))'$
        \item $M(\rho \cap \sigma = M(\rho) \logand M(\sigma)$
        \item $(M\rho \cup \sigma = M(\rho) \logor M(\sigma))$
        \item $M(\rho^{-1}) = (M(\rho))^T$
        \item $M(\rho \cdot \tau) = M(\rho) \cdot M(\sigma)$
        \item $M(A \times B) = 1$
        \item $M(\varnothing) = 0$
    \end{itemize}

\end{theorem}

\subsection{Виды бинарных отношений}

\begin{definition}
    Пусть $A$ и $B$ непустые множества $\rho \subseteq A \times B$.
    Первую(вторую) проекцию $\rho$ определим следующим образом:
    \begin{equation*}
        {pr_1}\rho =
         \{a(b) \in A(B) ~|~ \exists b(a) \in B(A) ~:~ (a, b) \in \rho\}
    \end{equation*}
\end{definition}

\begin{definition}
    $\rho$ называется 1(2)"=полным отношением, если $pr_1(2)\rho = A(B)$
\end{definition}

\begin{definition}
    Пусть $a \in A$. Срезом бинарного отношения $\rho \subseteq A \times B$ назовем
    множество, определяемое следующим образом:

    \begin{equation*}
        \rho(a) = \{b \in B ~|~ (a, b) \in \rho\}
    \end{equation*}
\end{definition}

\begin{definition}
    Бинарное отношение $\rho$ называется однозначным, если 
    $\forall a \in A ~| \rho(a)| \leq 1$
\end{definition}

Легко заметить, что если в матрице бинарного отношения не больше чем 1 единица,
то бинарное отношение является однозначным.

\begin{definition}
    Бинарное отношение $\rho$ называется обратнооднозначным, если обратное ему
    отношение является однозначным.
\end{definition}

Аналогичные рассуждения о матрицах можно привести и в этом случае.

\begin{definition}
    Бинарное отношение $\rho$ является взаимнооднозначным, если оно однозначно
    и обратнооднозначно.
\end{definition}

\begin{definition}
    Бинарное отношение называется отображением, если оно 1"=полно и однозначно.
\end{definition}

\begin{definition}
    Бинарное отношение $\rho$ называется инъективным (инъекцией), если $\rho$ является
    взаимнооднозначным отображением.
\end{definition}

$\rho$ называется сюръекцией, если оно 2"=полное + отображение.

Отображение, которое является одновременно инъекцией и 
сюръекцией (взаимооднозначным соответствием).

Из определения инъекцией $\forall a, b \in A (a \neq b) \rightarrow \rho(a) \neq \rho(b)$.

\paragraph{Отношение на множестве}

Пусть $A$ "--- непустое множество. Любое подмножество $A \times A$
называется отношением на множество $A$.

$|P(A \times A)| = 2^{n^2}$

$\delta_A = \{(a, a) ~|~ a \in A\}$ называется отношением тождества.

Приведем некоторые примеры отношений:

\begin{definition}
    Пусть $\rho \subseteq A \times B$.
    Тогда элемент $a$ находится в отношении $\rho$ с элементом $b$. Записывают это как $a \rho b$.
\end{definition}

\begin{example}
    Рассмотрим множество $\mathbb{N}$. Определим на нем:
    \begin{itemize}
        \item Отношение равенства: $a ~ \rho ~ b \Leftrightarrow a = b$
        \item Отношение $\leq$. Задается аналогично классическому варианту из школьной математики.
        \item Отношение делимости: $a ~\rho~ b \Leftrightarrow \exists k \in \mathbb{N} ~|~ k \cdot b = a$
        \item Отношение параллельности прямых. Задается на множестве прямых.
        \item Отношение $\tau$ на $\mathbb{R}$: $a ~ \rho ~ b \Leftrightarrow |a|  =  |b|$
    \end{itemize}
\end{example}

\subsection{Свойства отношений на множестве}

Пусть $A \neq \varnothing$, $\rho \subseteq A \times A$.

\begin{enumerate}
    \item $\rho$ называется рефлексивным, если $\forall a \in A, (a, a) \in \rho$
    \item $\rho$ называется антирефлексивным, если $\forall a \in A, (a, a) \notin \rho$
    \item $\rho$ называется симметричным, если 
    $\forall a, b \in A: (a, b) \in \rho \Leftrightarrow (b, a) \in \rho$
    \item $\rho$ называется антисимметричным, если
    $\forall a, b \in A: ((a, b) \in \rho) \logand ((b, a) \in \rho) \Leftrightarrow a = b$
    \item $\rho$ называется транзитивным, если 
    $\forall a, b, c \in A: (a, b) \in \rho \logand (b, c) \in \rho \Rightarrow (a, c) \in \rho$
    \item $\rho$ называется полным, если $\forall a,b \in A: (a, b) \in \rho \logor (b, a) \in \rho$
\end{enumerate}

\paragraph{Утверждение.} Пусть $A \neq \varnothing$, $\rho \subseteq A \times A$. $\rho$ транзитивно
тогда и только тогда, когда $\rho \times \rho \subseteq \rho$.

\subsection{Отношения эквивалетности}

Пусть $A \neq \varnothing$. Бинарное отношение на $A$ назовем отношением эквивалетности, если
оно рефлексивно, симметрично и транзитивно.

$\varepsilon \subseteq A \times A$

Назовем $\sqcap$ разбиением множества $A$, если

\begin{enumerate}
    \item $\bigcup\limits_{i \in I} \sqcap_i = A$ 
    \item $\forall i \in I ~:~ \forall j \in I (i \neq j \Rightarrow \sqcap_i \cap \sqcap_j = \varnothing)$
    , где $\sqcap_i$ "--- блоки разбиение $\sqcap$
\end{enumerate}

\begin{definition}
    Пусть $A \neq \varnothing$, а $\varepsilon$ на $A$.
    Подмножество $M \subseteq A$ называется классом
    эквивалентности $\varepsilon$ ($\varepsilon$"=классом) если
    выполняются два условия:
    \begin{itemize}
        \item $\forall a, b \in M:~ (a, b) \in \varepsilon$
        \item Любое добавление к $M$ другого элемента из множества $A$
        приведет к нарушению первого условия.
    \end{itemize}
\end{definition}

\paragraph{Утверждение} Классами эквивалетности $\varepsilon$ является
в точности срезом $\varepsilon$ по элементу множества $A$.

\begin{equation*}
    a \in A: ~  \varepsilon(a) = \{b \in A ~|~ (a, b) \in \varepsilon\}
\end{equation*}

\begin{theorem}{основная теорема об эквивалентности}
    Пусть $A \neq \varnothing$, тогда справедливо 3 Утверждения:
    \begin{enumerate}
        \item Каждому отношению эквивалетности $\varepsilon$ соотвествует
        разбиение $\sqcap(\varepsilon)$, которое состоит из $\varepsilon$"=классов.
        \item Для каждого разбиения $\sqcap$ множество $A$ множно построить эквивалетность
        $\varepsilon(\sqcap)$ на множестве $A$, которая определяется следующим образом:
        
        $\varepsilon(\sqcap) = \{(a, b) ~|~ a, b ~ \text{лежат в одном блоке разбиения }\sqcap \}$
        \item $\varepsilon(\sqcap(\varepsilon)) = \varepsilon$, $\sqcap(\varepsilon(\sqcap)) = \sqcap$
    \end{enumerate}

\end{theorem}

\subsection{Отношение порядка}

\begin{definition}
    Пусть $A \neq \varnothing$, $\rho \subseteq A \times A$ называется отношением 
порядка, если оно рефлексивно, антисимметрично и транзитивно.
\end{definition}

\begin{definition}
    Пусть $\rho$ явяляется отношением порядка на множестве $A$. Упорядоченным множеством назовем пару $(A, \rho)$.
\end{definition}

Примерами упорядоченного множества являются множество целых и рациональных
чисел с отношением $\leq$, множество целых чисел с отношением делимости,
множество множеств с отношением $\subseteq$.

В дальнейшем определим и будем обозначать отношение порядка $\leq$ и обратный
порядок $\geq$

\begin{definition}
    Упорядоченное множество $(A, \leq)$ называются линейно упорядоченным (цепью),
если порядок обладает свойством полноты, т.е $\forall a, b \in A : ~ a \leq b \logor b \leq a$
\end{definition}

\begin{definition}
    Пусть имеется упорядоченное множество с заданным порядком $(A , \leq)$. Убывающие
цепью назовем любую последовательность начинающуюся с $a$.

$a \ge a_1 \ge a_2 \ge \dots$
\end{definition}


\begin{definition}
    Пусть $(A, \leq)$ "--- конечное упорядоченное множество $a \in A$. Длиной убывающей 
цепи начинающейся с элемента $a$ называется количество элементов в ней, уменьшенное на 1.
\end{definition}

\begin{definition}
    Высотой $h(a)$ называется длина максимальной убывающей последовательности,
начинающейся с $a$.
\end{definition}

\begin{definition}
    Пусть $(A, \leq)$ "--- порядоченное множество. Будем говорить, что $a$ является
    нижним соседом $b$, если $a < b$ и $!\exists x \in A: ~ a < x < b$.
\end{definition}

\begin{definition}
    Пусть $(A, \leq)$ "--- конечное убывающее множество, $k$ "--- максимальной
    высоты. Проведем $k + 1$ горизонталь и пронумируем снизу вверх. На $i$ диагонали
    расположим элементы, длина цепи которых равна $i$. Получим диаграмму упорядоченного множества
\end{definition}

%Добавить примеры














































    
