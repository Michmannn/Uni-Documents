\documentclass[a4paper,oneside]{article}
\usepackage[utf8]{inputenc}
\usepackage[russian]{babel}
\usepackage{hyperref}
\usepackage{underscore}
\usepackage{setspace}
\usepackage{indentfirst} 
\usepackage{mathtools}
\usepackage{amsfonts}
\usepackage{enumitem}
% \usepackage[standard]{ntheorem}
\usepackage{amsthm}
\usepackage{cancel}
\usepackage{amssymb}
\usepackage[left=1.4cm,right=1.4cm,
    top=2.3cm,bottom=2.3cm,bindingoffset=0cm]{geometry}
\singlespacing

\usepackage{graphicx}
\graphicspath{ {./images/} }

\usepackage{fancyhdr}
\pagestyle{fancy}

\usepackage{tikz}

\usepackage{tabularx}
\newcommand{\bydef}{\stackrel{\text{по опр.}}{\implies}} % by definition - по определению
\newcommand{\parspace}{\vspace{10pt}}

\newcommand{\logor}{\vee}
\newcommand{\logand}{\wedge}

\newcommand{\imagin}{\mathrm{Im} \,}
\newcommand{\real}{\mathrm{Re} \,}

\newcommand{\dslim}{\displaystyle\lim}
\newcommand{\dslimn}{\dslim_{n \to \infty}}

\newcommand{\prop}[1]{#1^{\text{o}}}

\newcommand{\N}{\mathbb{N}}
\newcommand{\R}{\mathbb{R}}
\newcommand{\bb}[1]{\mathbb{#1}}

\newcommand{\eps}{\varepsilon}

\newcommand{\approach}[1]{\underset{#1}{\longrightarrow}}

% \theoremstyle{break}

% --- Теорема --- %
% \newtheoremstyle{break}% name
%   {}%         Space above, empty = `usual value'
%   {}%         Space below
%   {\itshape}% Body font
%   {}%         Indent amount (empty = no indent, \parindent = para indent)
%   {\bfseries}% Thm head font
%   {.}%        Punctuation after thm head
%   {\newline}% Space after thm head: \newline = linebreak
%   {}%         Thm head spec
% \theorembodyfont{\normalfont}
% \theoremstyle{break}
\newtheorem{theorem}{Теорема}[subsection]
% --------------- %

% --- Определение --- %
% \theorembodyfont{\normalfont}
\theoremstyle{definition}
\newtheorem{definition}{Определение}[subsection]
% ------------------- %

% --- Доказательство --- %
% \theoremheaderfont{\normalfont\itshape}
% \theorembodyfont{\normalfont}
% \newtheorem*{proof}{Доказательство.}
% ---------------------- %

% --- Пример --- %
\theoremstyle{definition}
\newtheorem*{example}{Пример}
% -------------- %

% --- Следствие --- %
% Corollary
% ----------------- %

% --- Замечание --- %
% Remark
\theoremstyle{definition}
\newtheorem*{remark}{Замечание}
% ----------------- %


\begin{document}

%----------------------------------------------------------------------------------------
%	TITLE PAGE
%----------------------------------------------------------------------------------------

\begin{titlepage} % Suppresses displaying the page number on the title page and the subsequent page counts as page 1
	\newcommand{\HRule}{\rule{\linewidth}{0.5mm}} % Defines a new command for horizontal lines, change thickness here
	
	\center % Centre everything on the page
	
	%------------------------------------------------
	%	Headings
	%------------------------------------------------
	
	\textsc{\LARGE Издательство Пупы и Лупы}\\[1.5cm] % Main heading such as the name of your university/college
	
	\textsc{\Large }\\[0.5cm] % Major heading such as course name
	
	\textsc{\large }\\[0.5cm] % Minor heading such as course title
	
	%------------------------------------------------
	%	Title
	%------------------------------------------------
	
	\HRule\\[0.4cm]
	
	{\huge\bfseries Лекции по дискретной математике}\\[0.4cm] % Title of your document
	
	\HRule\\[1.5cm]
	
	%------------------------------------------------
	%	Author(s)
	%------------------------------------------------
	
	\begin{minipage}{0.4\textwidth}
		\begin{flushleft}
			\large
			\textit{Автор}\\
			\textsc{Пупа} % Your name
		\end{flushleft}
	\end{minipage}
	~
	\begin{minipage}{0.4\textwidth}
		\begin{flushright}
			\large
			\textit{Редактор}\\
			\textsc{Лупа} % Supervisor's name
		\end{flushright}
	\end{minipage}
	
	% If you don't want a supervisor, uncomment the two lines below and comment the code above
	%{\large\textit{Author}}\\
	%John \textsc{Smith} % Your name
	
	%------------------------------------------------
	%	Date
	%------------------------------------------------
	
	\vfill\vfill\vfill % Position the date 3/4 down the remaining page
	
	{\large\today} % Date, change the \today to a set date if you want to be precise
	
	%------------------------------------------------
	%	Logo
	%------------------------------------------------
	
	%\vfill\vfill
	%\includegraphics[width=0.2\textwidth]{placeholder.jpg}\\[1cm] % Include a department/university logo - this will require the graphicx package
	 
	%----------------------------------------------------------------------------------------
	
	\vfill % Push the date up 1/4 of the remaining page
	
\end{titlepage}

%----------------------------------------------------------------------------------------

\section{Теория множеств}

\subsection{Множества и отношения между ними}

Множество "--- совокупность различаемых объектов.
Объекты называются его элементами.

Два множества равны ($A = B$), если:

\begin{equation*}
     \forall x: (x \in A \Leftrightarrow x \in B)
\end{equation*}

Множество $A$ включается в множество $B$ ($A \subseteq B$), если:

\begin{equation*}
    \forall x: (x\in A \Rightarrow x \in B)
\end{equation*}

Множество, $A$ строго включается в множество $B$ ($A \subset B$), если:

\begin{equation*}
    \forall x: (A \subseteq B ~ \& ~ A \neq B)
\end{equation*}

Множество называется пустым, если оно не содержит элементов.
Такое множество обозначается как $\varnothing$.

\begin{equation*}
    \forall A \neq \varnothing: ~ \varnothing \subseteq A
\end{equation*}

Некоторое множество $\Omega \neq \varnothing$ назовем универсальным множеством
и скажем, что

\begin{equation*}
    \forall A \neq \varnothing: ~ A \subseteq \Omega
\end{equation*}

\subsection{Операции над множествами}

Пусть $A, B \subseteq \Omega$:

\begin{equation*}
    A \cup B = \{ \forall~ x\in \Omega: ~ x \in A \logor x \in B \} \quad \text{(объединение)}
\end{equation*}

\begin{equation*}
    A\cap B = \{ \forall ~ x \in \Omega: ~ x \in A \logand x \in B \} \quad \text{(пересечение)}
\end{equation*}

\begin{equation*}
    \overline{A} = \{ \forall ~ x \in \Omega: ~ x \notin A \} \quad \text{(дополнение)}
\end{equation*}

\begin{equation*}
    A \setminus B = \{ \forall ~ x \in \Omega: ~ x\in A \logand x \notin B \} = 
    A \cup \overline{B}  \quad \text{(разность)} 
\end{equation*}

\begin{equation*}
    A \triangle B = \{ \forall ~ x \in \Omega: ~ x\in A \logand x\notin B \logor x \notin A 
    \logand x \in B \} \quad \text{(симметричная разность)}
\end{equation*}

Множество, содержащее конечное число элементов называется конечным.

Пусть $ A \neq \varnothing$ и $A$ состоит из $n$ элементов, тогда $|A| = n$ "---
мощность этого множества.

Множество при этом также обозначается как $A = \{ a_1, a_2, \dots, a_n\}$ или $A = \{x ~|~ P(x) \}$

Примеры:

\begin{equation*}
    A = \{ x \in \R ~|~ x\geq 0 \logand x \leq 1 \}
\end{equation*}

\subsection{Характеристические векторы множеств}

Пусть есть некоторое конечное множество $\Omega \neq \varnothing$ из $n$ элементов, 
$\Omega = \{ x_1, x_2, \dots, x_n\}$

Для удобства будем считать, что порядок перечисления множеств зафиксирован.

Теперь пусть $A \subseteq \Omega$.

Характеристическим вектором $A$ назовем булев вектор из $n$ компонентов 

\begin{equation*}
    \chi_A = (\chi_1^A, \chi_2^A, \dots, \chi_n^A), \text{~где } \chi_i^A = 
    \begin{cases}
        1, & x_i \in A \\
        0, & x_i \notin A 
    \end{cases}
\end{equation*}

Пусть $P(A)$ "--- множество всех подмножеств непустого множества $A$. 

Отметим, что $A \in P(\Omega)$

Пример:

$\Omega = \{a, b, c\}$

$P(\Omega) = \{ \varnothing, \{a\}, \{b\}, \{c\}, \{a, b\}, \{b, c\}, \{a, c\}, \{a, b, c\} \}$
\begin{center}
    \begin{tabularx}{0.2\textwidth}{c|c}
        $A$ & $\chi_A$ \\ \hline
        $\varnothing$ & $(0, 0, 0)$ \\
        $\{ a \}$ & $(1, 0, 0)$ \\  
        $\{ b \}$ & $(0, 1, 0)$ \\  
        $\{ c \}$ & $(0, 0, 1)$ \\  
        $\{ a,b \}$ & $(1, 1, 0)$ \\  
        $\{ b,c \}$ & $(0, 1, 1)$ \\  
        $\{ a,c \}$ & $(1, 0, 1)$ \\  
        $\{ a,b,c \}$ & $(1, 1, 1)$ \\  
    \end{tabularx}    
\end{center}

Стоит отметить, что данное отображение является биективным. 

\begin{theorem}[о числе подмножеств конечного множества]
    Число подмножеств $n$"=элементного множества равно $2^n$:

    \begin{equation*}
        |P(\Omega)| = 2^n
    \end{equation*}
    
\end{theorem}

Двоичной алгеброй называется $B_2 = \{ 0, 1\}$ с операциями $\{', +, \cdot\}$, где


\begin{itemize}
    \item <<$'$>> "--- логическое <<НЕ>>
    \item <<$+$>> "--- логическое <<ИЛИ>>
    \item <<$\cdot$>> "--- логическое <<И>>
\end{itemize}


Обозначим $B_2^n$ множество всех двоичных векторов длины $n$.

Пусть $\vec{a} = (a_1, a_2, \dots, a_n), \vec{b} = (b_1, b_2, \dots, b_n) 
\in B_2^n$

\begin{enumerate}
    \item $\vec{a}' = (a_1', a_2', \dots, a_n')$
    \item $\vec{a} + \vec{b} = (a_1 + b_1, a_2 + b_2, \dots, a_n + b_n )$
    \item $\vec{a} \cdot \vec{b} = (a_1 \cdot b_1, a_2 \cdot b_2, \dots, a_n \cdot b_n)$
\end{enumerate}

\begin{theorem}[О характеристических векторах]
    Пусть $\Omega \neq \varnothing$ (универсальное множество).
    Для любых непустых множеств $A, B \subseteq \Omega$ справедливо следующее:
    \begin{enumerate}
        \item $A = B \iff \chi_A = \chi_B$
        \item $\chi_{A\cup B} = \chi_A + \chi_B$
        \item $\chi_{A\cap B} = \chi_A \cdot \chi_B$
        \item $\chi_{\overline{A}} = (\chi_A)'$
        \item $\chi_\varnothing = (0, 0, \dots, 0) = \vec{0}$
        \item $\chi_\Omega = (1, 1, \dots, 1) = \vec{1}$
    \end{enumerate} 
\end{theorem}

Пример:

\begin{example}
Пусть $\Omega = \{ 1, 2, 3, 4, 5\}$, $A = \{1, 3, 4\}$, $B = \{ 1, 4, 5\}$.

Тогда $\chi_A = (1, 0, 1, 1, 0)$, $\chi_B = (1, 0, 0, 1, 1)$.

Получим:

$\chi_{\overline{A}} = (0, 1, 0, 0, 1)$

$\chi_{A \cup B} = (1, 0, 0, 1, 0)$

$\chi_{A \cap B} = (1, 0, 1, 1, 1)$
\end{example}

\subsection{Декартово произведение и отношения между множествами}

Пусть $A, B \neq \varnothing$. Декартовым произведением множеств $A$ и $B$
назовем множество упорядоченных пар 

\begin{equation*}
    A \times B = \{ ~(a, b) ~|~ a \in A \logand b \in B\}
\end{equation*}

$(a_1, b_1) = (a_2, b_2) \iff a_1 = a_2 \logand b_1 = b_2$

Очевидно, что если $|A| = n$, $|B| = m$, то $|A\times B| = n \times m$. Кроме того, $|P(A\times B)| = 2^{nm}$

Бинарным отношением между множествами $A$ и $B$ назовем любое подмножество
декартового произведения $\rho \subseteq |A \times B|$.

Все теоретико"=множественные операции также справедливы для бинарных отношений.








    


\end{document}