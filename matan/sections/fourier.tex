\section{Ряд и Интеграл Фурье}

\subsection{Задача, приводящая к ряду Фурье}

Пусть задана $f(x)$ и функциональная последовательность ${\varphi_n(x)}$ на $[a ; b]$.

\textbf{Задача}

Найти такую линейную комбинацию $\varphi_1(x), \varphi_2(x),\dots, \varphi_n(x)$, 
которая лучше всего приближает в среднем функцию $f(x)$ на $[a ; b]$. Т.е. задача 
нахождения таких коэффициентов \[ a_1,\dots,a_n : I = I(a_1,\dots,a_n) = \ds\int_a^b \left[ f(x) 
- \dssum_{k=1}^n a_k\varphi_k(x)\right] ^2 dx \approach{} \text{min.} \]


\begin{definition}
    Функции $\varphi_1(x), \varphi_2(x),\dots$ называются ортогональными на $[a ; b]$,
    если $\ds\int_a^b \varphi_k(x)*\varphi_l(x) dx = 0$
    $\forall k \neq l$.
    
    Тогда мы будем рассматривать ортогональную систему $\varphi_1(x),\dots,\varphi_n(x)$

    \[ I = I(a_1,\dots,a_n) = \ds\int_a^b \left[f(x) - \dssum_{k=1}^n a_k\varphi_k(x)\right]^2 dx 
    = \ds\int_a^b \left(f^2(x) - 2f(x)\dssum_{k=1}^n a_k\varphi_k(x) + \left(\dssum_{k=1}^n a_k\varphi_k(x)\right)^2 \right) dx = \]
    
    \[ = \ds\int_a^b f^2(x) dx - 2\dssum_{k=1}^n a_k \cdot \ds\int_a^b f(x) \varphi_k(x) dx 
    + \ds\int_a^b \left(\dssum_{k=1}^n a_k^2 \varphi_k^2(x) + 2 a_1 a_2 \varphi_1(x) \varphi_2(x)
    +\dots+ 2a_{n-1}a_n\varphi_{n-1}(x)\varphi_n(x) \right) dx = \]
    \[ = \left( \text{в силу ортогональности}: \ds\int_a^b\varphi_k(x)\varphi_l(x) dx = 0 (k \neq l)
    \right) \ds\int_a^b f^2(x)dx - 2 \dssum_{k=1}^n a_k\ds\int_a^b f(x)\varphi_k(x) dx 
    + \dssum_{k=1}^n a_k^2\ds\int_a^b\varphi_k^2(x) dx = \]
    \[ = \ds\int_a^b f^2(x)dx - 2 \cdot \dssum_{k=1}^n a_k\frac{\ds\int_a^b f(x)\varphi_k(x)dx}{\ds\int_a^b \varphi_k^2(x)dx}
    \ds\int_a^b \varphi_k^2(x)dx + \dssum_{k=1}^n a_k^2\ds\int_a^b \varphi_k^2(x) dx = \]
    
    % обозначим
    \[ \left(\dn \frac{\ds\int_a^b f(x)\varphi_k(x)dx}{\ds\int_a^b \varphi_k^2(x)dx} = c_k;
    \ds\int_a^b \varphi_k^2(x) dx = d_k^2 \right) \]
    
    \[ = \ds\int_a^b f^2(x)dx - 2\dssum_{k=1}^n a_k c_k d_k^2 + \dssum_{k=1}^n a_k^2 d_k^2 =
    \ds\int_a^b f^2(x)dx + \dssum_{k=1}^n (a_k - c_k)^2 d_k^2 - \dssum_{k=1}^n c_k^2
    d_k^2 \geq \ds\int_a^b f^2(x)dx - \dssum_{k=1}^n c_k^2 d_k^2 \dn \delta_n \]

    \[ \implies I(a_1,\dots,a_n) \geq \delta_n \text{ и } I(a_1,\dots,a_n) = \delta_n \text{ при } a_k = c_k (k=\overline{1, n}) \]

    т. е. $ \min I(a_1,\dots,a_n) = \delta_n$.

    $\{ \delta_n \}$ --- не возрастает и ограничена снизу $(\delta_n \geq 0)$
    $\iff$ по теореме о сходимости монотонной последовательности
    $\{ \delta_n \}$ сходится, т. е. $\exists \ds\limn \delta_n \dn \delta_0$

    Т. е. $\dslimn \left( \ds\int_a^b f^2(x)dx - \dssum_{k=1}^n c_k^2d_k^2 \right) = \delta_0 $

    \[ \ds\int_a^b f^2(x)dx - \dssum_{k=1}^n c_k^2d_k^2 = \delta_0 \]

    \[ \implies \dssum_{k=1}^n c_k^2d_k^2 = \ds\int_a^b f^2(x)dx - \delta_0 \leq \ds\int_a^bf^2(x)dx. \]

    Это неравенство носит название \textbf{неравенство Бесселя}.

    Оно имеет вид: $\dssum_{k=1}^\infty \frac{\left(\ds\int_a^b f(x)\varphi_k(x)dx\right)^2}{\ds\int_a^b \varphi_k^2(x)dx} \leq \ds\int_a^b f^2(x) dx$.

    Таким образом задача решается для ортогональной системы функций $\{ \varphi_k \}$
    и минимум указанной величине доставляют коэффициенты $a_k$  следующего вида:

    \[ a_k = c_k = \frac{\ds\int_a^b f(x) \varphi_k(x)dx}{\ds\int_a^b \varphi_k^2(x) dx}. \]

    Если $\delta_0 = 0 \implies f_n(x) = \dssum_{k=1}^n c_k\varphi_k(x)$ сходится
    в среднем к $f(x)$ на $[a ; b]$ и в этом случае неравенство Бесселя
    превращается в равенство Парсеваля: 
    \[ \dssum_{k=1}^\infty c_k^2 d_k^2 = \ds\int_a^b f^2(x)dx \]
    
\end{definition}


\begin{definition}
    Если $\delta_0 = 0$ для произвольной функции $f(x)$ из некоторого класса функций,
    то система $\{ \varphi_n(x) \}$ называется полной на отрезке $[a ; b]$ для
    рассматриваемого класса функций.
\end{definition}


\subsection{Тригонометрический ряд Фурье}

Рассмотрим наилучшее приближение на $[a ; b]$ системой тригонометрических полиномов
степени $n$ с периодом $T=b-a$. Т.к. при линейной замене переменных степень полинома
не меняется, то вместо $[a ; b]$ будем рассматривать $[-\pi ; \pi]$ и тригонометрические
полиномы вида:

\[T_n(x) = a_0 + \dssum_{k=1}^n a_k \cos kx + b_k \sin kx \]

\[\ds\int_{-\pi}^\pi (\cos kx \cdot \cos lx) dx =
 \begin{cases}
    0, &k \neq l \\
    \pi, &k = l \neq 0 \\
    2\pi, &k = l = 0   
  \end{cases}
\]

\[ \ds\int_{-\pi}^\pi (\sin kx \cdot \sin lx) dx =
 \begin{cases}
    0, &k \neq l \\
    \pi, &k = l \neq 0
 \end{cases}   
\]

\[\ds\int_{-\pi}^\pi (\sin lx \cdot \cos kx) dx = 0 \]

Таким образом последовательность $1, \cos x, \sin x, \cos 2x, \sin 2x,\dots$ --- 
взаимоортогональные функции на отрезке $[-\pi ; \pi]$, поэтому по выше приведённому
рассуждению, наилучшее приближение в среднем для заданной функции $f(x)$ дают тригонометрические
полиномы вида:

\[f_n(x) = \frac{a_0}{2} + \dssum_{k=1}^n a_k \cos kx + b_k \sin kx \text{, где } \]

\[a_k = \frac{1}{\pi} \ds\int_{-\pi}^\pi f(x) \cos kx \; dx \]

\[b_k = \frac{1}{\pi} \ds\int_{-\pi}^\pi f(x) \sin kx \; dx \]

Тогда эти функции $f_n(x)$ можно рассматривать как частичные суммы ряда:

\[\frac{a_0}{2} + \dssum_{k=1}^\infty a_k \cos kx + b_k \sin kx \] который
называется тригонометрическим рядом Фурье функции $f(x)$. Для такой системы неравенство Бесселя
примет следующий вид:

\[\frac{a_0^2}{2} + \dssum_{k=1}^\infty \left(a_k^2 + b_k^2 \right) \leq \frac{1}{\pi} \ds\int_{-\pi}^\pi f^2(x)dx \]
Покажем, что наша система функций $\frac{1}{2}, \cos x, \sin x, \cos 2x, \sin 2x,\dots$
является полной для интегрируемой на отрезке $[-\pi ; \pi]$ функции $f(x)$

\begin{remark}
    Таким образом, в рассматриваемой задаче ряд 
    $\ds\sum_{k = 1}^\infty c_k \varphi_k(x)$, где
    $c_k = \frac{\ds\int_a^b f(x) \varphi_k(x) dx}{\ds\int_a^b \varphi_k^2(x) dx}$
    является рядом Фурье для функции $f(x)$ по ортогональной системе $\{ \varphi_k(x) \}$.
\end{remark}

\begin{theorem}[Теорема о сходимости в среднем частичных сумм ряда Фурье]
    Частичные суммы ряда Фурье интегригрирумой на $[-\pi ; \pi]$ функции $f(x)$,
    сходятся в среднем на этом отрезке к этой функции.    
\end{theorem}
\begin{proof}
    $f(x)$ интегрируема на $[-\pi ; \pi]$, тогда по теореме о приближении интегрируемой
    функции $\forall \eps_1 > 0 \exists g_{\eps_1}(x)$ --- непрерывная на $[-\pi ; \pi]$

    \[g_{\eps_1}(-\pi) = g_{\eps_1}(\pi) \text{ и } \ds\int_{-\pi}^\pi \left(f(x) - g_{\eps_1}(x) \right)^2dx < \eps_1 \]

    По второй теореме Вейерштрасса существует равномерно сходящаяся
    к $g_{\varepsilon_1}(x)$ последовательность тригонометрических 
    полиномов $\{ T_n(x) \}$ с периодом $2\pi$. Она сходится к 
    $g_{\varepsilon_1}(x)$ и в среднем, то есть $\forall \varepsilon_1 > 0 \quad
    \exists n_1 \: : \quad \forall n \geq n_1 \quad
    \ds\int_{-\pi}^\pi (T_n(x) - g_{\varepsilon_1}(x))^2 dx < \varepsilon_1$.

    Частичная сумма $\{ f_n(x) \}$ даёт для $f(x)$ наилучшее приближение в 
    среднем на $[-\pi; \pi]$ среди всех полиномов $T_n(x) \implies
    \ds\int_{-\pi}^\pi (f(x) - f_n(x))^2 dx \leq 
    \ds\int_{-\pi}^\pi (f(x) - T_n(x))^2 dx \implies$ при $n \geq n_1$:
    
    \begin{align*}
        \ds\int_{-\pi}^\pi (f(x) - f_n(x))^2 dx &\leq 
        \ds\int_{-\pi}^\pi (f(x) - T_n(x))^2 dx
        = \ds\int_{-\pi}^\pi [ (f(x) - g_{\varepsilon_1}(x)) + (g_{\varepsilon_1}(x) - T_n(x)) ]^2 dx \leq \\
        &\leq 2 \ds\int_{-\pi}^\pi (f(x) - g_{\varepsilon_1}(x))^2 dx +
        2 \ds\int_{-\pi}^\pi (g_{\varepsilon_1}(x) - T_n(x))^2 dx < 4 \varepsilon_1 \dn \varepsilon
    \end{align*}

    Получаем, что $\forall \varepsilon > 0 \quad \exists n_\varepsilon = n_1 :
    \forall n \geq n_\varepsilon \quad \ds\int_{-\pi}^\pi (f(x) - f_n(x))^2 dx < \varepsilon$
    то есть частичная сумма ряда Фурье $\{ f_n(x) \}$ сходится в среднем к $f(x)$.
\end{proof}

\begin{remark}
    \begin{enumerate}
        \item
            Система $1, \cos x, \sin x, \cos 2x, \dots$ является полной на
            классе интегрируемых на $[-\pi; \pi]$ функций.
        \item
            Так как $f_n(x)$ имеют на $[-\pi; \pi]$ непрерывную производную
            любого порядка, получаем, что интегрируемую на $[-\pi; \pi]$
            функцию можно приблизить в среднем на нём с помощью функции,
            имеющей непрервыную производную любого порядка.
    \end{enumerate}
\end{remark}

\begin{example}
    \begin{align*}
        f(x) &= q \quad [-\pi;\pi] \\
        a_0 &= \frac{1}{\pi} \ds\int_{-\pi}^\pi x \; dx = 
        \frac{1}{\pi} \cdot \frac{x^2}{2} \Bigg|_{-\pi}^\pi = 0, \quad
        a_k = \frac{1}{\pi} \ds\int_{-\pi}^\pi x \cdot \cos x \; dx = 0. \\
        b_k &= \frac{1}{\pi} \ds\int_{-\pi}^\pi x \sin kx \; dx = \left[ \begin{array}{lr}
            u = x, \, &u' = 1 \\
            v' = \sin kx, \, &v = \frac{-\cos kx}{k}
        \end{array} \right] = \\
        &= \frac{1}{\pi} \left( -\frac{x \cos x}{k} \Bigg|_{-\pi}^\pi + 
        \frac{1}{k} \ds\int_{-\pi}^\pi \cos kx \; dx \right) =
        \frac{1}{\pi} \left( -\frac{2 \pi \cos \pi k}{k} + \frac{\sin kx}{k^2} \Bigg|_{-\pi}^\pi \right) =
        \frac{2 (-1)^{k + 1}}{k} \\
        &\implies f_n(x) = \ds\sum_{k = 1}^n \frac{2 (-1)^{k + 1}}{k} \sin kx
    \end{align*}

    Преобразуем выражение для частичных сумм ряда Фурье. Так как члены ряда
    Фурье --- периодичные функции, то будем считать $f(x)$ периодичной с
    периодом $2\pi$ (это всегда можно сделать, продолжив её периодически
    вне $[-\pi; \pi]$).

    \begin{align*}
        f_n(x) &= \frac{a_0}{2} + \sum_{k = 1}^n a_k \cos kx + b_k \sin kx = \\
        &= \frac{1}{2\pi} \int_{-\pi}^\pi f(S) dS + \sum_{k = 1}^n \left(
        \cos kx \cdot \frac{1}{\pi} \int_{-\pi}^\pi f(S) \cos kS \; dS +
        \sin kx \cdot \frac{1}{\pi} \int_{-\pi}^\pi f(S) \sin kS \; dS \right) = \\
        &= \frac{1}{\pi} \int_{-\pi}^\pi f(S) \left( \frac{1}{2} \sum{k = 1}^n
        \left( \cos kx \cdot \cos kS + \sin kx \cdot \sin kS \right) \right) dS = \\
        &= \frac{1}{\pi} \int_{-\pi}^\pi f(S) \left( \frac{1}{2} + \sum{k = 1}^n
        \cos k (S - x) \right) ds
    \end{align*}

    Произведём преобразования над выражением внутри скобок:

    \begin{align*}
        \frac{1}{2} + \sum_{k = 1}^n \cos k \underbrace{(S - x)}_t &=
        \frac{1}{2} + \cos t + \cos 2t + \dots + \cos nt = \frac{1}{2} +
        \frac{2 \sin \frac{t}{2} (\cos t + \dots + \cos nt)}{2 \sin \frac{t}{2}} = \\
        &= \frac{1}{2} + \frac{ 
            \sin \frac{3t}{2} - 
            \sin \frac{t}{2} + 
            \sin \frac{5t}{2} - 
            \sin \frac{3t}{2} + \dots +
            \sin \left(n + \frac{1}{2}\right)t -
            \sin \left(n 1 \frac{1}{2}\right)t }{2 \sin \frac{t}{2}} = \\
        &= \frac{1}{2} + \frac{\sin \left(n + \frac{1}{2}\right)t - \sin \frac{t}{2}}{2 \sin \frac{t}{2}} =
        \frac{\sin \left( n + \frac{1}{2} \right)t}{2 \sin \frac{t}{2}}
    \end{align*}

    Получаем:

    \begin{align*}
        \frac{1}{\pi} \int_{-\pi}^\pi f(S) \frac{\sin \left( n + \frac{1}{2} \right) (S - x)}{2 \sin \frac{S - x}{2}} \; dS =
        \int_{-\pi}^\pi f(S) D_n(S - x) \; dS
    \end{align*}

    Где $D_n(t) = \frac{\sin \left( n + \frac{1}{2} \right)t}{2 \pi \sin \frac{t}{2}}$ ---
    Ядро Дирихле.

    Отметим, что $D_n(t) = D_n(-t)$ и $D_n(t + 2\pi) = D_n(t)$. Тогда 
    \begin{align*}
        f_n(x) &= \int_{-\pi}^\pi f(S) D_n(S - x) \; dS \over{s = t + x}{=} \\
        &= \int_{-\pi - x}^{\pi - x} f(x + t) D_n(t) \; dt +
        \int_{-\pi - x}^{-\pi} f(x + t) D_n(t) \; dt +
        \int_{\pi}^{\pi - x} f(x + t) D_n(t) \; dt \over{t = y - 2\pi}{=} \\
        &= \int_{-\pi}^\pi f(x + t) D_n(t) \; dt +
        \int_{\pi - x}^\pi f(x + y - 2\pi) D_n(y - 2\pi) \; dy -
        \int_{\pi - x}^\pi f(x + t) D_n(t) \; dt = \\
        &= \int_{-\pi}^\pi f(x + t) D_n(t) dt
    \end{align*}

    \textbf{???} $\implies D_n(t) \; dt = 1$
\end{example}


\begin{theorem}[Лемма 1]
    Если $f(x)$ интегрируема на $[a - \omega; b + \omega] \implies 
    \forall \varepsilon > 0 \exists \delta_\varepsilon > 0 : \forall |x| <
    \delta_\varepsilon < \omega$ выполняется 
    \[ \int_a^b |f(x + t) - f(t)| \; dt < \varepsilon \]
\end{theorem}
\begin{proof}
    Если $f(x)$ интегрируема на $[a - \omega; b + \omega] \implies
    \forall \varepsilon_1 > 0 \quad \exists f_{\varepsilon_1}(x)$ ---
    непрерывная: $\ds\int_{a - \omega}^{b + \omega} |f(t) - f_{\varepsilon_1}(t)| dt < \varepsilon_1$.
    
    Но \begin{align*}
        \int_a^b |f(x + t) - f(t)| dt &\leq 
        \int_a^b |f(x + t) - f_{\varepsilon_1}(x + t)| dt + \\
        &+ \int_a^b |f_{\varepsilon_1}(x + t) - \\
        &- f_{\varepsilon_1} (t)| dt + \\
        &+ \int_a^b |f(t) - f_{\varepsilon_1}(t)| dt < \\
        &< 2 \varepsilon_1 + \int_a^b |f_{\varepsilon_1} (x + t) - f_{\varepsilon_1} (t)|
    \end{align*}

    Так как \begin{align*}
        \int_a^b |f(x + t) - f_{\varepsilon_1}(x + t)| dt =
        \int_{a + x}^{b + x} |f(S) - f_{\varepsilon_1}(S)| dS \leq
        \int_{a - \omega}^{b + \omega} |f(S) - f_{\varepsilon_1}(S)| dS < \varepsilon_1
    \end{align*}

    и \begin{align*}
        \int_a^b |f(t) - f_{\varepsilon_1}(t)| dt \leq 
        \int_{a - \omega}^{b + \omega} |f(t) - f_{\varepsilon_1}(t)| dt
    \end{align*}

    По Теореме о непрерывности интеграла по параметру для $\varepsilon_1 > 0 \quad
    \exists \delta_{\varepsilon_1} > 0 : \forall |x| < \delta_{\varepsilon_1}$
    выполняется \begin{align*}
        \int_a^b |f_{\varepsilon_1}(x + t) - f_{\varepsilon_1}(t)| dt < \varepsilon_1 \\
        \implies \int_a^b |f(x + t) - f(t)| dt < 3 \varepsilon_1 \dn \varepsilon
    \end{align*}

    То есть $\forall \varepsilon > 0 \quad \exists \delta_\varepsilon =
    \delta_{\varepsilon_1} > 0 : \forall |x| < \delta_\varepsilon < \omega$
    выполняется \begin{align*}
        \int_a^b |f(x + t) - f(t)| dt < \varepsilon
    \end{align*}
\end{proof}


\begin{theorem}[Лемма 2]
    Пусть $f(x)$ и $g(x)$ интегрируемы на любом конечном отрезке и $g(x)$ ---
    ограничена, а $\ds\int_{-\infty}^\infty |f(S)| dS$ сходится. Тогда 
    \[ I(x, \lambda) = \int_{-\infty}^\infty f(x + t) g(t) \cos \lambda t \; dt \]
    и
    \[ \overline{I}(x, \lambda) = \int_{-\infty}^\infty f(x + t) g(t) \sin \lambda t \; dt \]
    сходятся к 0 при $\lambda \approach{} +\infty$ равномерно на любом отрезке
    $[-a; a]$.
\end{theorem}
\begin{proof}
    Заменим в этих интегралах $t$ на $t + \frac{\pi}{\lambda}$.

    \begin{align*}
        I(x, \lambda) &= -\int_{-\infty}^\infty f(x + t + \frac{\pi}{\lambda}) g(t + \frac{\pi}{\lambda}) \cos \lambda t \; dt \\
        \overline{I}(x, \lambda) &= \int_{-\infty}^\infty f(x + t + \frac{\pi}{\lambda}) g(t + \frac{\pi}{\lambda}) \sin \lambda t \; dt
    \end{align*}

    Возьмём полусумму исходного и нового выражения для каждого из них, то
    получим:

    \begin{align*}
        |I(x, \lambda)| &\leq \frac{1}{2} \int_{-\infty}^{+\infty} \left| f(S + \frac{\pi}{\lambda}) g(S) - f(S)g(t) \right| dt \\
        |\overline{I}(x, \lambda)| &\leq \frac{1}{2} \int_{-\infty}^{+\infty} \left| f(S + \frac{\pi}{\lambda}) g(S) - f(S)g(t) \right| dt
    \end{align*} где $S = x + t$.

    Отсюда видно, что доказательство достаточно провести для $I(x, \lambda)$.
    Имеем \begin{align*}
        |I(x, \lambda)| &\leq I_N^{(1)} + I_N^{(2)}, \text{где} \\
        I_N^{(1)} &= \frac{1}{2} \int_{|t| \geq N} \left|f(S + \frac{\pi}{\lambda})g(S) - f(S)g(t)\right| dt, \\
        I_N^{(2)} &= \frac{1}{2} \int_{|t| \leq N} \left|f(S + \frac{\pi}{\lambda})g(S) - f(S)g(t)\right| dt, \\
    \end{align*}

    Пусть $|x| \leq a, \, \lambda \geq \pi \implies$ при $|t| \geq N \implies
    |S| \geq N - a$ 
    \begin{align*}
        \left| S + \frac{\pi}{\lambda} \right| &\geq N - a - 1 \\
        \implies I_N^{(1)} &\leq \frac{1}{2} \int_{|t| \geq N} 
        \left[ \left| f(S + \frac{\pi}{\lambda}) \right|
               \left| g(t + \frac{\pi}{\lambda}) \right| +
               |f(S)| |g(t)| \right] dt \leq \\
        &\leq M \cdot \int_{|t| \geq N} \left[ \left| f(S + \frac{\pi}{\lambda}) \right| + |f(S)| \right] dt \leq \\
        &\leq 2 M \cdot \int_{|S| \geq N - a - 1} |f(S)| dS
    \end{align*}

    Так как $\ds\int_{-\infty}^{+\infty} |f(S)| dS$ сходится $\implies
    \forall \varepsilon > 0 \quad \exists n_\varepsilon : \forall N > 
    n_\varepsilon$ выполняется $2 M \cdot \ds\int_{|S| \geq N - a - 1} |f(S)| dS < \frac{\varepsilon}{2}$,
    то есть $I_N^{(1)} < \frac{\varepsilon}{2}$.

    Теперь при $|t| \leq N, |x| \leq a, \lambda \geq \pi$ имеем $|S| \leq N + a, \,
    |S + \frac{\pi}{\lambda} \leq N + a + 1$. И пусть $M_1 = \ds\sup_{|S| \leq N + a} |f(S)| \implies$

    \begin{align*}
        I_N^{(2)} &\leq \int_{|t| \leq N} 
        \left| \left( f\left( S + \frac{\pi}{\lambda} \right) - f(S) \right)
               g \left( t + \frac{\pi}{\lambda} \right) +
               \left( g\left( t + \frac{\pi}{\lambda} \right) - g(t) \right)
               f(S) \right| dt \leq \\
        &\leq M \int_{|S| \leq N + a} \left| 
            f \left( S + \frac{\pi}{\lambda} \right) - f(S) \right| dS +
        M_1 \int_{|t| \leq N} \left| 
            g \left( t + \frac{\pi}{\lambda} \right) - g(t) \right| dt
    \end{align*}

    Тогда по Лемме 1: $\exists \lambda(\varepsilon) : \forall \lambda > 
    \lambda(\varepsilon)$ выполняется 
    \begin{align*}
        M \int_{|S| \leq N + a} \left| 
        f \left( S + \frac{\pi}{\lambda} \right) - f(S) \right| dS < \frac{\varepsilon}{4} \; \text{ и } \;
        M_1 \int_{|t| \leq N} \left| 
        g \left( t + \frac{\pi}{\lambda} \right) - g(t) \right| dt < \frac{\varepsilon}{4}
    \end{align*}

    $\implies I_N^{(2)} < \frac{\varepsilon}{2}$, то есть $\forall \varepsilon > 0 \quad
    \exists \lambda(\varepsilon) : \forall \lambda > \lambda(\varepsilon)$ 
    выполняется $|I(x, \lambda) < \varepsilon|$
\end{proof}


\subsection{Частичная сумма ряда Фурье}

\begin{theorem}
    Пусть $f(x)$ интегрируема на $[-\pi; \pi]$ и периодична с периодом 
    $T = 2\pi$. Если в точке $x_0$ $\exists \ds\lim_{x \to x_0 - 0} f(x) \dn
    f(x_0 - 0)$ и $\exists \ds\lim_{x \to x_0 + 0} f(x) \dn f(x_0 + 0)$ и $f(x)$
    имеет ограниченную производную в $(x_0 - \delta_0; x_0)$ 
    $(x_0; x_0 + \delta_0) \quad (\delta > 0)$ (то есть $\exists M > 0 : 
    |f'(x)| \leq M \quad \forall x \in (x_0 - \delta_0; x_0) \cup (x_0; x_0 + \delta_0))
    \implies$ ряд Фурье функции $f(x)$ сходится в точке $x_0$ к
    $\frac{f(x_0 - 0) + f(x_0 + 0)}{2}$.

    Если же $f(x)$ имеет ограниченную производную на $[a; b] \implies$ ряд
    Фурье функции $f(x)$ сходится к $f(x)$ на $\forall [a_1; b_1] \subset [a; b]
    \quad (a < a_1 < b_1 < b)$ \underline{равномерно}.
\end{theorem}
\begin{proof}
    \begin{align*}
        f_n(x_0) - \frac{f(x_0) + f(x_0 + 0)}{2} &= 
        \int_{-\pi}^\pi f(x_0 + t) D_n(t) dt - 
        f(x_0 - 0) \int_{-\pi}^0 D_n(t) dt - f(x_0 + 0) \int_{-0}^\pi D_n(t) dt = \\
        &= \int_{-\pi}^0 (f(x_0 + t) - f(x_0 - 0)) D_n(t) dt +
        \int_0^\pi (f(x_0 + t) - f(x_0 + 0)) \cdot D_n(t) dt = \\
        &= \left( 
            \int_{-\delta}^0 (f(x_0 + t) - f(x_0 - 0)) D_n(t) dt +
            \int_0^\delta (f(x_0 + t) - f(x_0 + 0)) D_n(t) dt
        \right) + \\ &+ \left( 
            \int_{-\pi}^{-\delta} (f(x_0 + t) - f(x_0 - 0)) D_n(t) dt +
            \int_\delta^\pi (f(x_0 + t) - f(x_0 + 0)) D_n(t) dt
        \right) = \\
        &= I_n(\delta, x_0) + \overline{I_n}(\delta, x_0)
    \end{align*}
    
    Второе слагаемое преобразуем следующим образом:
    \begin{align*}
        \overline{I_n}(\delta, x_0) = \int_{\delta \leq |t| \leq \pi}
            f(x_0 + t) D_n(t) dt - (f(x_0 - 0) + f(x_0 + 0)) \cdot
            \int_{\delta \leq |t| \leq \pi} D_n(t) dt
    \end{align*}

    Оценим $I_n(\delta; x_0)$:

    $f(x)$ и $f_n(x)$ периодичны с $T = 2\pi \implies$ оценим пров. при 
    $|x| \leq \pi$. Если $\delta_0 : \delta < \delta_0 < \pi \implies$ по
    Теореме Лагранжа, $|f(x_0 + t) - f(x_0 - 0)| = |f'(\xi) \cdot t| 
    \leq M \cdot |t|$ если $t \in [-\delta; 0]$, то есть $x_0 - \delta < \xi < x_0$
    и $|f(x_0 + t) - f(x_0 + 0)| = |f'(\xi_1) \cdot t| \leq M \cdot |t|$, где
    $t \in [0; \delta], \, x_0 < \varepsilon_1 < x_0 + \delta \implies$ при
    $\delta < \delta_0$ $|I_n(\delta, x_0)| \leq M \ds\int_{-\delta}^\delta
    |t \cdot D_n(t)| dt$.

    Но $|t \cdot D_n(t)| = \left| 
        \frac{t \sin \left( n + \frac{1}{2} \right) t}{2 \pi \sin \frac{t}{2}} \right|
    \leq \left| \frac{\frac{t}{2}}{\pi \sin \frac{t}{2}} \right| \implies
    \forall \varepsilon > 0 \quad \exists \delta > 0 : |I_n(\delta, x_0)| \leq
    M \ds\int_{-\delta}^\delta |t \cdot D_n(t)| dt \leq M_\delta 
    \leq \frac{\varepsilon}{2}$ (при $\delta < \ds\frac{\varepsilon}{2M}$)

    Рассмотрим оценку $\overline{I_n}(\delta, x_0)$.

    Имеем $\ds\int_{\delta \leq |t| \leq \pi} f(x_0 + t) D_n(t) dt =
    \ds\int_{-\infty}^{+\infty} \overline{f}(x_0 + t) g(t) \sin(\lambda_n t) dt$

    \[ 
        \int_{\delta \leq |t| \leq \pi} D_n(t) dt =
        \int_{-\infty}^{+\infty} \varphi(x_0 + t) g(t) \sin (\lambda_n t) dt
    \]

    где
    \begin{align*}
        \lambda_n &= n + \frac{1}{2}, \quad \overline{f}(S) = \begin{cases}
            f(S), &\text{при } |S| \leq 2\pi \\
            0, &\text{при } |S| > 2\pi
        \end{cases} \\
        g(t) &= \begin{cases}
            \frac{1}{2\pi \sin \frac{t}{2}}, &\text{при } \delta \leq |t| \leq \pi \\
            0, &\text{при } |t| < \delta
        \end{cases} \\
        \varphi(S) &= \begin{cases}
            1, &\text{при } |S| \leq 2\pi \\
            0, &\text{при } |S| > 2\pi
        \end{cases}
    \end{align*}

    Тогда по Лемме 2 оба этих интеграла стремятся к 0 при $n \approach{} \infty$
    равномерно на любом отрезке вида $[-a; a]$, то есть для нашего $\varepsilon > 0$:

    \begin{align*}
        \exists N : \forall n \geq N \quad |\overline{I_n}(\delta, x_0)| &\leq
        \left| \ds\int_{-\infty}^{+\infty} \overline{f}(x_0 + t) g(t) \sin(\lambda_n t) dt \right| + \\
        &+ \frac{|f(x_0 + 0) + f(x_0 - 0)|}{2} \cdot \\ &\cdot \left|
            \int_{-\infty}^{+\infty} \varphi(x_0 + t) g(t) \sin (\lambda_n t) dt
        \right| < \frac{\varepsilon}{2}
    \end{align*}

    \begin{align*}
        \implies \forall \varepsilon > 0 \quad \exists N : \forall n \geq N \quad
        \left| f_n(x_0) - \frac{f(x_0 + 0) + f(x_0 - 0)}{2} \right| <
        |I_n(\delta; x_0)| + |\overline{I_n}(\delta; x_0)| <
        \frac{\varepsilon}{2} + \frac{\varepsilon}{2} = \varepsilon
    \end{align*}

    Первая часть теоремы доказана.

    Пусть $\forall x \in [a_1; b_1] \subset [a; b] \quad |f'(x)| \leq M$. Так
    как $\exists f'(x) \implies$ функция непрерывна на $[a_1; b_1] \implies
    \forall x \in [a_1; b_1] \quad \frac{f(x_0 + 0) + f(x_0 - 0)}{2} = f(x)$
    и $|f'(x + t)| \leq M$, если $|t| \leq \delta \leq \delta_0$, а
    $\delta_0 = \min \{ b - b_1; a_1 - a \} \implies \delta$ можно выбрать
    не зависящим от $x$.

    Так как $\left| \ds\frac{f(x + 0) + f(x - 0)}{2} \right| = |f(x)| \leq 
    \ds\sup_{x \in [a; b]} |f(x)| \implies$ номер $N$ можно выбрать в оценке
    $\overline{I_n}(\delta; x_0)$ общим для $\forall x \in [a_1; b_1] \implies
    f_n(x)$ на $[a_1; b_1]$ сходится к $f(x)$ равномерно.
\end{proof}


\subsection{Оценка остатка ряда Фурье}

Рассмотрим вопрос о скорости сходимости ряда Фурье в случае, если функция $f(x)$
имеет производную порядка $m + 1$.

\begin{theorem}
    Пусть функция $f(x)$ и все её производные до некоторого порядка $m$ 
    непрерывны на сегменте $[-\pi; \pi]$, причём $f^{(l)}(-\pi) = f^{(l)}(\pi)$
    при $l \leq m$, а производная $f^{m + l}(x)$ кусочно-непрерывная на
    сегменте $[-\pi; \pi]$. Тогда частичная сумма порядка $N$ ряда Фурье
    для $f(x)$ отличается от $f(x)$ на величину 
    $\overline{\beta}\left( \ds\frac{1}{N^{m + 1/2}} \right)$ при
    $N \approach{} \infty$.
\end{theorem}
\begin{proof}
    Проведём доказательство при $m = 0$.

    Преобразуем выражение для коэффициентов Фурье с помощью интегрирования
    по частям. При $k > 0$ имеем

    \begin{align*}
        a_k = \frac{1}{\pi} \int_{-\pi}^\pi f(t) \cos kt \; dt =
        \frac{1}{\pi} \left[ f(t) \cdot \frac{\sin kt}{k} \Bigg|_{-\pi}^\pi -
        \frac{1}{k} \int_{-\pi}^\pi f'(t) \sin kt \; dt \right] =
        -\frac{b_k^{(1)}}{k}, \quad b_k = \frac{a_k^{(1)}}{k}
    \end{align*}

    Где $a_k^{(1)}, b_k^{(1)}$ --- коэффициенты Фурье функции $f'(x)$. Если
    $S_N$ --- частичная сумма порядка $N$ для ряда Фурье функции $f(x)$, то
    \begin{align*}
        |f(x) - S_N(x)| = \left| \sum_{k = N + 1}^\infty [a_k \cos kx +
        b_k \sin kx] \right| \leq \sum_{k = N + 1}^\infty [|a_k| + |b_k|] = 
        \sum_{k = N + 1}^\infty \frac{|a_k^{(1)}| + |b_k^{(1)}|}{k}, \\
        \text{т.к.} |\cos kx| \leq 1, \, |\sin kx| \leq 1
    \end{align*}

    С помощью неравенства Коши-Буняковского и неравенства
    \[ 
        \left[ |a_k^{(1)}| + |b_k^{(1)} \right]^2 \leq 
        2 \left[ |a_k^{(1)}|^2 + |b_k^{(1)}|^2 \right] 
    \]

    Получем
    \[ 
        \sum_{k = N + 1}^\infty \frac{|a_k^{(1)}| + |b_k^{(1)}|}{k} \leq
        \sqrt{2 \sum_{k = N + 1}^\infty \frac{1}{k^2} \cdot 
        \sum_{k = N + 1}^\infty \left[ |a_k^{(1)}|^2 + |b_k^{(1)}|^2 \right]}
    \]

    На основании неравенства Беселя для функции $f'(x)$ ряд 
    \[ \sum_{k = N + 1}^\infty \left[ |a_k^{(1)}|^2 + |b_k^{(1)}|^2 \right] \]
    сходится. Поэтому
    \[ 
        \sum_{k = N + 1}^\infty \left[ |a_k^{(1)}|^2 + |b_k^{(1)}|^2 \right] =
        \overline{\overline{o}}(1) \quad (N \approach{} \infty)
    \]

    С другой стороны, $\frac{1}{k^2} \leq \ds\int_{k - 1}^k \frac{dx}{x^2}$ и,
    следовательно,
    \[ 
        \sum_{k = N + 1}^\infty \frac{1}{k^2} \leq \int_N^\infty \frac{dx}{x^2} 
        = \frac{1}{N}
    \]

    Поэтому \[
        \sum_{k = N + 1}^\infty \frac{|a_k^{(1)}| + |b_k^{(1)}|}{k} =
        \overline{\overline{o}} \left( \frac{1}{\sqrt{N}} \right), \quad
        N \approach{} \infty
    \]

    То есть $f(x) - S_N(x) = \overline{\overline{o}} \left( \frac{1}{\sqrt{N}} \right),
    N \approach{} \infty$, что и требовалось доказать при $m = 0$.

    Аналогично проводится доказательство и при $m > 0$. Только в этом случае
    для коэффициентов Фурье необходимо воспользоваться интегрированием по
    частям $m + 1$ раз. При этом получаем оценку $|f(x) - S_N(x)| \leq
    \ds\frac{|a_k^{(m + 1)}| + |b_k^{(m + 1)}|}{k^{m + 1}}$, где 
    $a_k^{(m + 1)}, b_k^{(m + 1)}$ --- коэффициенты Фурье функции
    $f^{(m + 1)}(x)$.

    Так как \[
        \sum_{k = N + 1}^\infty \frac{|a_k^{(m + 1)}| + |b_k^{(m + 1)}|}{k^{m + 1}} \leq
        \sqrt{
            2 \sum_{k = N + 1}^\infty \frac{1}{k^{2(m + 1)}} \cdot
            \sum_{k = N + 1}^\infty \left[ |a_k^{(m + 1)}|^2 + |b_k^{(m + 1)}|^2 \right]
        }
    \]

    то из сходимости ряда $\ds\sum_{k = N + 1}^\infty \left[ |a_k^{(m + 1)}|^2 + |b_k^{(m + 1)}|^2 \right]$
    и оценки \[
        \sum_{k = N + 1}^\infty \frac{1}{k^{2(m + 1)}} \leq
        \int_N^\infty \frac{dx}{x^{2(m + 1)}} =
        \underline{\underline{O}} \left( \frac{1}{N^{2 (m + 1)}} \right)
    \]

    следует, что $f(x) - S_N(x) = \overline{\overline{o}} \left(
        \ds\frac{1}{N^{m + 1/2}}
    \right), N \approach{} \infty$
\end{proof}


\subsection{Интеграл Фурье}

В доказанном соотношении
\[
    \lim_{n \to \infty} \int_{-\pi}^\pi f(t + x) \frac{\sin \lambda_n t}{2\pi \sin \frac{t}{2}} dt =
    \frac{f(x + 0) + f(x - 0)}{2}, quad \lambda = n + \frac{1}{2}
\]

Вклад в интеграл при достаточно больших значениях $\lambda_n$ даёт только
область малых значений $t$, в которой $\sin \frac{t}{2}$ мало отличается от
$\frac{t}{2}$. Поэтому при определённых требованиях, наложенных на функцию
$f(x)$, это соотношение можно заменить более общим

\[ 
    \lim_{\lambda \to +\infty} \frac{1}{\pi} \int_{-\infty}^\infty 
        f(t + x) \frac{\sin \lambda t}{t} dt =
    \frac{f(x + 0) + f(x - 0)}{2}
\]

Левая часть последнего соотношения называется \underline{Интегралом Фурье} для
функции $f(x)$.

\begin{remark}
    Чаще всего используют другую запись интеграла Фурье. Так как
    \begin{align*}
        \int_{-\infty}^\infty f(t + x) \frac{\sin \lambda t}{t} dt &=
        \int_{-\infty}^\infty f(S) \frac{\sin \lambda (S - x)}{S - x} dS, \text{и} \\
        \frac{\sin \lambda (S - x)}{S - x} &= \frac{1}{2}
        \int_{-\lambda}^\lambda \cos \mu (S - x) d\mu
    \end{align*}

    то интеграл Фурье можно записать в виде
    \begin{align*}
        \lim_{\lambda \to +\infty} \frac{1}{2\pi} \int_{-\infty}^\infty f(S) dS
        \int_{-\lambda}^\lambda \cos \mu (S - x) d\mu =
        \frac{f(x + 0) + f(x - 0)}{2}
    \end{align*}

    Но $|f(S) \cos \mu (S - x)| \leq |f(S)|$ и по теореме об интегрировании по
    параметру несобственных интегралов, зависящих от параметров, интегрирование
    по переменным $S$ и $\mu$ для функций, удовлетворяющих условию
    \[ \ds\int_{-\infty}^\infty |f(S)| dS < \infty \] можно менять местами, откуда
    $\ds\frac{f(x - 0) + f(x + 0)}{2} = 
    \ds\lim_{\lambda \to +\infty} \frac{1}{2\pi} \ds\int_{-\lambda}^\lambda
    d\mu [ \cos \mu x \cdot F_1(\mu) + \sin \mu x \cdot F_2(\mu)]$, где
    $F_1(\mu) = \ds\int_{-\infty}^\infty f(S) \cos \mu S \; dS$,
    $F_2(\mu) = \ds\int_{-\infty}^\infty f(S) \sin \mu S \; dS$.

    С помощью формул $\cos x = \ds\frac{e^{ix} + e^{-ix}}{2}$,
    $\sin x = \ds\frac{e^{ix} - e^{-ix}}{2i}$ ряд Фурье и интеграл Фурье можно
    представить в симметричной форме:

    \[ 
        \ds\frac{f(x - 0) + f(x + 0)}{2} = \ds\sum_{k = -\infty}^\infty C_k \cdot e^{ikx},
        \text{где } C_k = \ds\frac{1}{2\pi} \ds\int_{-\pi}^\pi f(x) e^{-ikx} dx
    \]

    \[
        \frac{f(x - 0) + f(x + 0)}{2} = \lim_{\lambda \to +\infty} 
        \int_{-\lambda}^\lambda F(\mu) e^{i \mu x} d\mu,
        \text{где } F(\mu) = \frac{1}{2\pi} \int_{-\infty}^\infty 
        f(x) e^{-i \mu x} dx
    \]
\end{remark}

В самом деле:

a)

\begin{align*}
    \frac{a_0}{2} + \sum_{k = 1}^\infty (a_k \cos kx + b_k \sin kx) &=
    \frac{a_0}{2} + \sum_{k = 1}^\infty \left(
        a_k \frac{e^{ikx} + e^{-ikx}}{2} +
        b_k \frac{e^{ikx} - e^{-ikx}}{2i}
    \right) = \\
    &= \frac{a_0}{2} + \sum_{k = 1}^\infty \left(
        a_k \frac{e^{ikx} + e^{-ikx}}{2} -
        i b_k \frac{e^{ikx} - e^{-ikx}}{2}
    \right) = \\
    &= \frac{a_0}{2} + \sum_{k = 1}^\infty \frac{1}{2} \left[
        (a_k - i b_k) e^{ikx} + (a_k + i b_k) e^{-ikx}
    \right] = \\
    &= \frac{a_0}{2} + \sum_{k = 1}^\infty \frac{1}{2} \Bigg[
        \left( 
            \frac{1}{\pi} \int_{-\pi}^\pi f(x) \cos kx dx -
            \frac{i}{\pi} \int_{-\pi}^\pi f(x) \sin kx dx 
        \right) e^{ikx} + \\
        &+ \left(
            \frac{1}{\pi} \int_{-\pi}^\pi f(x) \cos kx dx +
            \frac{i}{\pi} \int_{-\pi}^\pi f(x) \sin kx dx
        \right) e^{-ikx}
    \Bigg] = \\
    &= \frac{1}{2\pi} \int_{-\pi}^\pi f(x) dx +
        \sum_{k = 1}^\infty \frac{1}{2\pi} \int_{-\pi}^\pi 
        f(x) (\cos kx - i \sin kx) dx \cdot e^{ikx} + \\ &+
        \sum_{k = 1}^\infty \frac{1}{2\pi} \int_{-\pi}^\pi
        f(x) [\cos kx + i \sin kx] dx \cdot e^{-ikx} = \\
    &= \frac{1}{2\pi} \int_{-\pi}^\pi f(x) dx +
        \sum_{k = 1}^\infty \frac{e^{ikx}}{2\pi}
        \int_{-\pi}^\pi f(x) \left(
            \frac{e^{ikx} + e^{-ikx}}{2} -
            \frac{e^{ikx} - e^{-ikx}}{2}
        \right) dx + \\ &+
        \sum_{k = 1}^\infty \frac{e^{-ikx}}{2\pi}
        \int_{-\pi}^\pi f(x) \left(
            \frac{e^{ikx} + e^{-ikx}}{2} +
            \frac{e^{ikx} - e^{-ikx}}{2}
        \right) dx = \\
    &= \frac{1}{2\pi} \int_{-\pi}^\pi f(x) dx +
        \sum_{k = 1}^\infty \frac{e^{ikx}}{2\pi}
        \int_{-\pi}^\pi f(x) \cdot e^{-ikx} dx + \\ &+
        \sum_{k = 1}^\infty \frac{e^{-ikx}}{2\pi}
        \int_{-\pi}^\pi f(x) \cdot e^{ikx} dx
\end{align*}

Заменим порядок суммирования в последней сумме $k$ на $-k$. Получим:

\begin{align*}
    \frac{1}{2\pi} \int_{-\pi}^\pi f(x) dx &+
    \sum_{k = 1}^\infty \frac{1}{2\pi}
    \int_{-\pi}^\pi f(x) e^{-ikx} dx \cdot e^{ikx} + \\ 
    &+ \sum_{k = -\infty}^1 \frac{1}{2\pi}
    \int_{-\pi}^\pi f(x) e^{-ikx} dx \cdot e^{ikx} = \\
    &= \sum_{k = -\infty}^\infty C_k \cdot e^{ikx},
    \text{где  } C_k = \frac{1}{2\pi} \int_{-\pi}^\pi f(x) e^{-ikx} dx
\end{align*}


b) Преобразуем интеграл Фурье:

\begin{align*}
    &\frac{1}{2\pi} \int_{-\lambda}^\lambda d\mu \left[
        \frac{e^{i\mu x} + e^{-i\mu x}}{2} \cdot
        \int_{-\infty}^\infty f(S) \cos \mu S \; dS +
        \frac{e^{i\mu x} - e^{-i\mu x}}{2i} \cdot
        \int_{-\infty}^\infty f(S) \sin \mu S \; dS
    \right] = \\
    &= \frac{1}{2\pi} \int_{-\lambda}^\lambda d\mu \left[
        \frac{e^{i\mu x} + e^{-i\mu x}}{2} \cdot
        \int_{-\infty}^\infty f(S) \frac{e^{i\mu S} + e^{-i\mu S}}{2} \; dS -
        \frac{e^{i\mu x} - e^{-i\mu x}}{2} \cdot
        \int_{-\infty}^\infty f(S) \frac{e^{i\mu S} - e^{-i\mu S}}{2} \; dS
    \right] = \\
    &= \frac{1}{2\pi} \int_{-\lambda}^\lambda d\mu \Bigg[
        \frac{e^{i\mu x}}{2} \cdot
        \int_{-\infty}^\infty f(S) \left(
            \frac{e^{i\mu S} + e^{-i \mu S}}{2} -
            \frac{e^{i\mu S} - e^{-i \mu S}}{2}
        \right) dS + \\ &+
        \frac{e^{-i\mu x}}{2} \cdot
        \int_{-\infty}^\infty f(S) \left(
            \frac{e^{i\mu S} + e^{-i \mu S}}{2} +
            \frac{e^{i\mu S} - e^{-i \mu S}}{2}
        \right) dS
    \Bigg] = \\
    &= \frac{1}{2\pi} \int_{-\lambda}^\lambda d\mu \Bigg[
        \frac{e^{i\mu x}}{2} \cdot
        \int_{-\infty}^\infty f(S) e^{-i\mu S} dS +
        \frac{e^{-i\mu x}}{2} \cdot
        \int_{-\infty}^\infty f(S) e^{i\mu S} dS
    \Bigg] = \\
    &= 
    \frac{1}{2\pi} \int_{-\lambda}^\lambda d\mu \Bigg[
    \frac{e^{i\mu x}}{2} \cdot
    \underbrace{\int_{-\infty}^\infty f(S) e^{-i\mu S} dS}_{F(\mu)} \Bigg] +
    \frac{1}{2\pi} \int_{-\lambda}^\lambda d\mu \Bigg[
    \frac{e^{-i\mu x}}{2} \cdot
    \underbrace{\int_{-\infty}^\infty f(S) e^{i\mu S} dS}_{F(\mu)} \Bigg]
\end{align*}

Сделаем замену $\mu$ на $-\mu$:

\begin{align*}
    \frac{1}{2\pi} \int_{-\lambda}^\lambda \underbrace{\left(
        \int_{-\infty}^\infty f(S) e^{-i\mu S} dS
    \right)}_{F(\mu)} e^{i\mu x} d\mu = 
    \int_{-\lambda}^\lambda F(\mu) e^{i\mu x} \; d\mu, \,
    \text{где  } F(\mu) = \frac{1}{2\pi} 
    \int_{-\infty}^\infty f(S) e^{-i\mu S} dS
\end{align*}

Предел $\ds\lim_{\lambda \to +\infty} \ds\int_{-\lambda}^\lambda F(\mu) e^{i\mu x} d\mu$
часто кратко обозначают $\ds\int_{-\infty}^\infty F(\mu) e^{i\mu x} d\mu$, так
как он является главным значением этого интеграла.


Рассмотрим условия, при которых интеграл Фурье сходится к
$\frac{f(x + 0) + f(x - 0)}{2}$. При этом для интеграла Фурье будем
использовать выражение
\[ \lim_{\lambda \to +\infty} \frac{1}{\pi} \int_{-\infty}^\infty f(t + x) \frac{\sin \lambda t}{t} dt \]


\begin{theorem}[О сходимости интеграла Фурье в точке]
    Пусть функция $f(x)$ задана на интервале $(-\infty; +\infty)$ и интеграл
    $\ds\int_{-\infty}^\infty |f(x)| dx$ сходится. Если в рассматриваемой
    точке $x$ существуют пределы $f(x - 0)$ и $f(x + 0)$ и функция имеет
    ограниченную производную на интервале $(x - \delta_0, x)$, 
    $(x, x + \delta_0)$, $\delta_0 > 0$, то интеграл Фурье в данной точке
    сходится и $\ds\frac{f(x - 0) + f(x + 0)}{2}$.

    Если же функция $f(x)$ имеет ограниченную производную на некотором
    сегменте $[a; b]$ при $b - a < 2\pi$, то интеграл Фурье сходится к функции
    $f(x)$ равномерно на любом сегменте $[a_1; b_1]$ при $0 < a_1 < b_1 < b$.
\end{theorem}
\begin{proof}
    Пусть $x \in (c, c + 2\pi)$ и $\varphi(x) = f(x)$ при $x \in [c, c + 2\pi)$.
    Продолжим функцию $\varphi(x)$ вне отрезка $[c, c + 2\pi)$ периодически с
    периодом $2\pi$. Пусть $\lambda = \lambda_n + \alpha$, где
    $\lambda_n = n + \frac{1}{2}$, $n$ --- целое положительное число,
    $0 \leq \alpha < 1$, а $\varphi_n(x)$ --- частичная сумма ряда Фурье
    функции $\varphi(x)$.

    Утверждение данной теоремы следует из предыдущей теоремы, если показать, что
    при $\lambda \approach{} +\infty$ разность 
    \[ \frac{1}{\pi} \int_{-\infty}^\infty f(t + x) \frac{\sin \lambda t}{t} dt - \varphi_n(x) \]

    сходится к нулю равномерно на любом сегменте $[a; b]$ при
    $c < a < b < c + 2\pi$. Так как 
    \begin{align*}
        \varphi_n(x) = \frac{1}{\pi} \int_{-\pi}^\pi \varphi(t + x) 
            \frac{\sin \lambda_n t}{2 \sin \frac{t}{2}} dt
    \end{align*}

    то

    \begin{align*}
        &\frac{1}{\pi} \int_{-\infty}^\infty f(t + x)
            \frac{\sin \lambda t}{t} dt - \varphi_n(x) = \\
        &= \frac{1}{\pi} \int_{|t| \leq \delta} \left[
            f(t + x) \frac{\sin \lambda t}{t} -
            \varphi(t + x) \frac{\sin \lambda_n t}{2 \sin \frac{t}{2}}
        \right] dt + \\
        &+ \frac{1}{\pi} \int_{|t| \geq \delta}
            f(t + x) \frac{\sin \lambda t}{t} dt -
        \frac{1}{\pi} \int_{\delta \leq |t| \leq \pi}
            \varphi(t + x) \frac{\sin \lambda_n t}{2 \sin \frac{t}{2}} dt = \\
        &= I_1(x, \lambda) + I_2(x, \lambda) + I_3(x, \lambda)
    \end{align*}

    Если выбрать $\delta$ из условия, чтобы $[x - \delta, x + \delta] \subset
    [c, c + 2\pi]$ при $x \in [a, b]$, то при $|t| \leq \delta$ имеем
    $\varphi(t + x) = f(t + x)$.

    Поэтому $I_1(x, \lambda) = \frac{1}{\pi} \int_{|t| \leq \delta}
    f(t + x) \psi \psi(t, \lambda) dt$, где

    \begin{align*}
        \psi(t, \lambda) &= \frac{\sin \lambda t}{t} - 
        \frac{\sin \lambda_n t}{2 \sin \frac{t}{2}} =
        \frac{\sin \lambda t - \sin \lambda_n t}{t} + 
        \sin \lambda_n t \left( \frac{1}{t} - \frac{1}{2\sin \frac{t}{2}} \right) = \\
        &= \alpha \sin \lambda_n t \cdot \frac{\cos \alpha t - 1}{\alpha t} +
        \alpha \cdot \cos \lambda_n t \frac{\sin \alpha t}{\alpha t} +
        \sin \lambda_n t \left( \frac{1}{t} - \frac{1}{2 \sin \frac{t}{2}} \right)
    \end{align*}

    Отсюда $|\psi(t, \lambda)| \leq \left| \frac{\cos \alpha t - 1}{\alpha t} \right| +
    \left| \frac{\sin \alpha t}{\alpha t} \right| + 
    \left| \frac{1}{t} - \frac{1}{2 \sin \frac{t}{2}} \right|$.

    Посколько функции $\frac{\cos t - 1}{t}$, $\frac{\sin t}{t}$,
    $\frac{1}{t} - \frac{1}{2 \sin \frac{t}{2}}$ имеют конечный при 
    $t \approach{} 0$ и непрерывны при $0 < |t| \leq \pi$, то
    функция $\psi(t, \lambda)$ удовлетворяет неравенству $|\psi(t, \lambda)| \leq A$
    при $|t| \leq \pi$ ($A$ --- постоянная).

    Поэтому
    \[ 
        |I_1(x, \lambda)| \leq \frac{1}{\pi} \int_{|t| \leq \delta}
            |f(t + x)| \cdot |\psi(t, \lambda)| dt \leq
        A \sup_{x \in [c, c + 2 \pi]} |f(x)| \frac{2 \delta}{\pi}
    \]

    Следовательно, при любом $\epsilon > 0$ можно выбрать $\delta$ настолько
    малым, чтобы имело место быть неравенство $I_1(x, \lambda) < \frac{\epsilon}{3}$
    при $x \in [a, b]$. При выбранном значении $\delta$ согласно лемме 2,
    интегралы
    \begin{align*}
        I_2(x, \lambda) = \frac{2}{\pi} \int_{|t| \geq \delta}
            f(t + x) \cdot \frac{\sin \lambda t}{t} dt, \\
        I_3(x, \lambda) = \frac{1}{\pi} \int_{\delta \leq |t| \leq \pi}
            \varphi(t + x) \cdot \frac{\sin \lambda_n t}{2 \sin \frac{t}{2}} dt
    \end{align*}

    равномерно сходятся к нулю при $\lambda \approach{} +\infty, \,
    x \in [a, b]$. Поэтому найдётся такое число $\overline{\lambda} =
    \overline{\lambda}(\varepsilon)$, что
    \[ 
        |I_2(x, \lambda)| < \frac{\varepsilon}{3},
        |I_3(x, \lambda)| < \frac{\varepsilon}{3},
    \]

    при $\lambda \geq \overline{\lambda}, \, x \in [a, b]$.

    Таким образом, при $\lambda \geq \overline{\lambda}(\varepsilon), \, 
    x \in [a, b]$ имеем
    \[ 
        \left| \frac{1}{\pi} \int_{-\infty}^\infty f(t + x) \cdot
            \frac{\sin \lambda t}{t} dt - \varphi_n (x) \right| \leq
        |I_1(x, \lambda)| + |I_2(x, \lambda)| + |I_3(x, \lambda)| < \varepsilon
    \]
\end{proof}
