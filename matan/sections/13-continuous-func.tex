\section{Непрерывность функций}

\subsection{Точки непрерывности}

Пусть функция $f(x)$ задана на множестве $A$.

\begin{definition}[Гейне]
    $f(x)$ называют \textbf{непрерывной} в точке $a$, если
    $\forall \{ x_n \}: x_n \in A,\, x_n \approach{n \to \infty} a$
    выполняется $f(x_n) \approach{n \to \infty} f(a)$
\end{definition}

\begin{definition}[Коши]
    $f(x)$ называют \textbf{непрерывной} в точке $a$, если
    $\forall \eps > 0 \quad \exists \delta_\eps > 0 \quad
    \forall x \in A: \: |x - a| < \delta_\eps$ выполняется
    $|f(x) - f(a)| < \eps$
\end{definition}

\begin{definition}
    Если $a$ -- предельная точка множества $A$, то $f(x)$ называют \textbf{непрерывной}
    в точке $a$, если $\exists \dslim_{x \to \infty} f(x) = f(a)$
\end{definition}

\textbf{Замечание.} Во всех определениях тогда точка $a$ называется \textbf{точкой непрерывности}
функции $f(x)$.

Если в определении Коши заменить окрестность точки $a$ на левую или правую полуокрестность,
то получим определение функции, непрерывной в точке $a$ слева или справа соответственно.
(Для фукнций одной переменной).

\begin{definition}
    Если функция непрерывна в каждой точке множества $B$, 
    то говорят, что функция непрерывна на множестве $B$.
\end{definition}

\begin{definition}
    Точка $a$ называется \textbf{точкой разрыва} $f(x)$, если или $f(x)$ определена
    в точке $a$, но точка $a$ не является точкой непрерывности $f(x)$ или точка $a$ --
    предельная точка $A$ и $\exists \dslim_{x \to a} f(x)$, но $a \notin A = D(f)$
\end{definition}

\subsection{Классификация точек разрыва (для функции одной переменной)}

\begin{enumerate}
    \item
        Если $\exists \dslim_{x \to a} f(x)$, но он либо не равен $f(a)$, либо
        $a \notin A = D(f) \implies a$ называется точкой устранимого разрыва.

        *рисунок*

        $f(x) = \frac{\sin x}{x} \implies x = 0$ -- точка устранимого разрыва.

    \item 
        Если $\exists \dslim_{x \to a - 0} f(x) \ne \dslim_{x \to a + 0} f(x)$, то
        точку $a$ называют точкой разрыва I рода.

        *рисунок*

        $sgn x = \begin{cases}
            1  &\implies x > 0 \\
            0  &\implies x = 0 \\
            -1 &\implies x < 0
        \end{cases}$

        $x = 0$ -- точка разрыва I рода.
    
    \item
        В остальных случаях точка $a$ называется точкой разрыва II-го рода, в частности,
        если в этой точке не существует хотя бы одного одностороннего предела.

        *рисунок*

        $y = \frac{1}{x} \implies x = 0$ -- точка разрыва II-го рода.
\end{enumerate}

\begin{theorem}[Свойства непрерывности функций]
    \begin{enumerate}
        \item 
            Если фукнция $f(x)$ задана в окрестности точки $a$ и непрерывна в точке $a$,
            то существует окрестность точки $a$ в которой $f(x)$ ограничена и существует
            окрестность точки $a$ в которой $f(x)$ сохраняеи знак числа $f(a) \ne 0$.
        
        \item 
            Если $f_1(x)$ и $f_2(x)$ непрерывны в точке $a$ $\implies$
            $c_1 f_1(x) + c_2 f_2(x), \, (c_1, c_2 \textnormal{ -- const})$,
            $f_1(x) * f_2(x)$, $\frac{f_1(x)}{f_2(x)}, \, (f_x(x) \ne 0)$ -- непрерывны в точке $a$.
        
        \item
            Если $t = \varphi(x), \, y = f(t)$ заданы так, что $E(\varphi) = D(f)$,
            $\varphi(x)$ непрерывна в точке $a$, $f(t)$ непрерывна в $b = \varphi(a)$
            $\implies f(\varphi(x))$ непрерывна в точке $a$.
    \end{enumerate}
\end{theorem}

\subsection{Некоторые свойства функций непрерывных на замкнутом ограниченном\\ множестве}

\begin{definition}
    Непрерывную на множестве $A$ функцию $f(x)$, ($A \subset \R$ или $A \subset \R^m$)
    называют \textbf{равномерно непрерывной} на $A$, если
    \[
        \forall \eps > 0 \quad \exists \delta_\eps > 0 \quad \forall x', x'' \in A:
        |x' - x''| < \delta_\eps
    \]
    \[
        \textnormal{выполняется } |f(x') - f(x'')| < \eps
    \]
\end{definition}

\textbf{Пример}

$y = \sin \frac{1}{x}, \, A = (0; 1)$

$\sin \frac{1}{x}$ -- непрерывна на $A$, но не является равномерно непрерывной, так как

$x_n = \frac{1}{\frac{\pi}{2} + \pi n} 
\implies \begin{cases}
    f(x_{2n}) = \sin (\frac{\pi}{2} + 2 \pi n) = 1 \\
    f(x_{2n+1} = \sin (-\frac{\pi}{2} + 2 \pi n) = -1
\end{cases}$

$x_{2n} - x_{2n-1} = \frac{1}{\frac{\pi}{2} + 2 \pi n} - \frac{1}{-\frac{\pi}{2} + 2 \pi n} =
-\frac{\pi}{(\frac{\pi}{2} + 2 \pi n)(-\frac{\pi}{2} + 2 \pi n)} \approach{n \to \infty} 0$

$\implies \exists \eps_0 > 0$ (К примеру, $\eps_0$ любое из $0 < \eps_0 < 2$. Возьмём $\eps_0 = \frac{1}{10}$)
$\implies \forall \delta > 0 \quad \exists x'=2n, x''=2n-1: \: |x' - x''| < \delta$, а
$\underbrace{|f(x') - f(x'')|}_2 \ge \underbrace{\eps_0}_\frac{1}{10} \implies \sin \frac{1}{x}$ 
не является равномерно непрерывной на $A$.

\begin{theorem}[О непрерывных функциях на замкнутом непрерывном множестве]
    Пусть $f(x)$ задано и непрерывно на заданном ограниченном множестве $A$. Тогда она:

    \begin{enumerate}
        \item Ограничена на $A$ (первая теорема Вейерштрасса)
        \item Достигает на $A$ своего наибольшего и наименьшего значения (вторая теорема Вейерштрасса)
        \item Равномерно непрерывна на $A$ (теорема Кантора)
    \end{enumerate}
    
\end{theorem}

\begin{proof}
    \begin{enumerate}
        \item 
            Допустим \textbf{от противного}, что $f(x)$ не ограничена на $A$, то есть
            $\forall C > 0 \quad \exists x_C \in A: \: |f(x_C)| > C$
            Возьмём $C = n, \, n \in \N \implies \exists x_n \in A: \: f(x_n) > n$
            (Точки $x_n$ всегда можно считать различными, достаточно их выбирать из условия
            $|f(x_{n+1})| > |f(x_n)|$)

            Получили $\{ x_n \}$ -- ограниченная, так как $x_n \in A$ -- ограниченная
            $\implies$ по теореме Больцано-Вейерштрасса для последовательностей

            $\exists \{ x_{K_n} \} \subset \{ x_n \}: \: x_{K_n} \approach{n \to \infty} a$
            $\implies a$ -- предельная точка $A$, $A$ -- замкнуто $\implies a \in A$.

            $f(x)$ непрерывна на $A \implies f(x)$ непрерывна в точке $a$ $\implies$
            по определению Гейне $f(x_{K_n}) \approach{} f(a)$, но это невозможно, так как

            \[ f(x_{K_n}) > n, \, n \in \N, \, \textnormal{то есть неограничена} \]

            Получили противоречие $\implies$ $f(x)$ -- ограничена на множестве $A$.
        
        \item
            Надо доказать, что $\exists x_1, x_2 \in A: \begin{cases}
                f(x_1) = \sup f(x) = M \\
                f(x_2) = \inf f(x) = m
            \end{cases}; x \in A$

            Допустим \textbf{противное}, что
            \[ \forall x \in A \quad f(x) \ne M \]
            
            Рассмотрим $\varphi(x) = \frac{1}{M - f(x)}$ -- непрерывную на $A$, $\varphi(x) > 0$ на $A$.
            $\implies$ по первой теореме Вейерштрасса $\varphi$ -- ограничена на $A$ 
            $\implies \exists \mu > 0: \: \forall x \in A \quad \varphi(x) \le \mu$,
            то есть $\frac{1}{M - f(x)} \le \mu \quad \forall x \in A$
            $\implies M - f(x) \ge \frac{1}{\mu} ~ f(x) \le M - \frac{1}{\mu} \quad
            \forall x \in A$, то есть мы нашли мажоранту $< M$, а это противоречит определению
            верхней грани, значит
            \[ \exists x_1 \in A \quad f(x_1) = M \]

            Для $m$ аналогично.

        \item
            \textbf{От противного}: $f(x)$ не является равномерно непрерывной на $A$, то есть
            \[ 
                \exists \eps_0 > 0 \quad \forall \delta > 0 \quad \exists x_0', x_0'' \in A: \: | x_0' - x_0'' | < \delta,
                \, | f(x_0') - f(x_0'') | \ge \eps_0
            \]
            Будем брать $\delta = \delta_n$, где $\{ \delta_n \} \approach{n \to 0} 0$

            Тогда $\forall \delta_n \quad \exists x_n', x_n'': \: |x_n' - x_n''| < \delta_n, \, |f(x_n') - f(x_n'')| \ge \eps_0$

            $\{x_n'\}, \{x_n''\} \in A$, $A$ -- ограничено $\implies$ $\{x_n'\}, \{x_n''\}$ -- ограничены
            $\implies$ по теореме Больцано-Вейерштрасса $\exists \{x_{K_n}'\}$ -- подпоследовательность
            $\{x_n'\}: \: x_{K_n}' \approach{n \to \infty} a$, $A$ -- замкнуто $\implies a \in A$.

            \[ 0 < |x_{K_n}' - x_{K_n}''| < \delta_{K_n} \]
            \[ 0 \approach{} 0, \, \delta_{K_n} \approach{} 0 \implies |x_{K_n}' - x_{K_n}''| \approach{} 0 \]
            
            $\implies \{x_{K_n}''\} \approach{n \to \infty} a$, тогда по определению непрерывности по Гейне
            $f(x_{K_n}') \approach{} f(a)$ и $f(x_{K_n}'') \approach{} f(a)$, а это противоречит тому, что
            $|f(x_{K_n}') - f(x_{K_n}'')| \ge \eps_0$, значит $f(x)$ равномерно непрерывна на множестве $A$.
    \end{enumerate}
\end{proof}

\begin{theorem}[Коши о промежуточном значении]
    Пусть $f(x)$ -- непрерывна на $[a; b]$ (на концах отрезка будем понимать как одностороннюю непрерывность).
    Если $f(a) \ne f(b)$ $\implies$ $f(x)$ принимает на $(a; b)$ любое значение между $f(a)$ и $f(b)$.
\end{theorem}
\begin{proof}
    Допустим $f(a) < f(b)$ и пусть $\gamma: \: f(a) < \gamma < f(b)$

    Рассмотрим $\varphi(x) = f(x) - \gamma$. Очевидно $\varphi(a) < 0, \varphi(b) > 0$,
    $\varphi(x)$ непрерывна на $[a; b]$. $\frac{a + b}{2}$ -- середина отрезка $[a; b]$

    Возможны три случая:
    \begin{enumerate}
        \item $\varphi(\frac{a + b}{2}) = 0 \implies x_0 = \frac{a + b}{2} \implies f(x_0) = \gamma$
        \item $\varphi(\frac{a + b}{2}) > 0$, то вместо отрезка $[a; b]$ возьмём $[a_1; b_1] = [a; \frac{a + b}{2}]$
        \item $\varphi(\frac{a + b}{2}) < 0$, то вместо отрезка $[a; b]$ возьмём $[a_1; b_1] = [\frac{a + b}{2}; b]$
    \end{enumerate}
    и $b_1 - a_1 = \frac{b - a}{2}$

    Разделим отрезок $[a_1; b_1]$ пополам и так далее. В результате будем получать отрезки 
    $[a_n; b_n]: \: \varphi(a_n) < 0, \, \varphi(b_n) > 0, \, b_n - a_n = \frac{b - a}{2^n}$

    Если при $k$: $\varphi(\frac{a_k - b_k}{2}) = 0: \: x_0 = \frac{a_k - b_k}{2} \quad f(x0) = \gamma$

    Если процесс бесконечен, то простроим две последовательности:
    \[ \{a_n\} \textnormal{ -- монотонно неубывающую, ограниченную } (a_n < b) \]
    \[ \{b_n\} \textnormal{ -- монотонно невозрастающую, ограниченную } (b_n < a) \]

    $\implies \exists \dslim_{x \to \infty} a_n$ обозначим $x_0$ и 
    $\exists \dslim_{x \to \infty} b_n = x_0$, так как

    \[ 0 < b_n - a_n < \frac{b - a}{2^n} \]
    \[ 0 \approach{} 0, \, \frac{b - a}{2^n} \approach{} 0 \implies b_n - a_n \approach{} 0 \]

    Так как $\varphi(a_n) < 0 \implies \varphi(x_0) \le 0; \; \varphi(b_n) > 0 \implies \varphi(x_0) \ge 0$
    $\implies \varphi(x_0) = 0$, то есть $f(x_0) = \gamma$
\end{proof}

\textbf{Следствие}

Функция $f(x)$ непрерывная на отрезке $[a; b]$, а также принимает на нём любое значение между $\inf f(x)$ и $\sup f(x)$.

%%%%%%%%%%%%%

\subsection{Монотонные функции}

\begin{definition}
    Фукнция $f(x)$ называется \textbf{монотонно неубывающей (возрастающей; невозрастающей; убывающей)} на $A$, если

    $\forall x_1, x_2 \in A: \: x_1 < x_2$ выполняется $f(x_1) \le f(x_2)$ 
    $(
        f(x_1) < f(x_2); \;
        f(x_1) \ge f(x_2); \;
        f(x_1) > f(x_2) 
    )$

    Неубывающие (возрастающие; невозрастающие; убывающие) функции называют монотонными. Иногда \textbf{возрастающие и убывающие}
    называют \textbf{строго монотонными}.
\end{definition}

\begin{theorem}[T1. О монотонной функции]
    Пусть $f(x)$ монотонна на $(a; b) \implies \forall \xi \in (a; b)$ 
    $\quad \exists \dslim_{x \to \xi - 0} f(x)$, обозначенное как $f(\xi - 0)$,
    $\quad \exists \dslim_{x \to \xi + 0} f(x)$, обозначенное как $f(\xi + 0)$,
    а $f(\xi)$ заключено между ними.
\end{theorem}
\begin{proof}
    Пусть для определённости $f(x)$ монотонно неубывает на $(a; b)$. Возьмём $x < \xi$
    $\implies f(x) \le f(\xi) \implies \sup f(x) \le f(\xi) \; (x < \xi)$
    $\implies$ по 2 признаку sup $\forall \eps > 0 \quad \exists x_0:\: x_0 < \xi$
    и $f(x_0) > \sup f(x) - \eps \quad (x < \xi)$

    Тогда $\forall x: x_0 < x < \xi$ имеем 
    $\sup f(x) - \eps < f(x_0) \le f(x) \le \sup f(x) < \sup f(x) + \eps \quad (x < \xi)$

    Получаем $|f(x) - \sup f(x)| < \eps \quad (x < \xi) \implies$
    \[ \forall \eps > 0 \quad \exists \delta_\eps = \xi - x_0 > 0 \]
    \[ \forall \xi - \delta_\eps < x < \xi \textnormal{ выполняется } |f(x) - \sup f(x)| < \eps \quad (x < \xi)\]

    $\bydef \sup f(x) = f(\xi - 0) \quad (x < \xi)$

    $\exists f(\xi + 0) = \inf f(x) \quad (x > \xi)$ доказывается аналогично

    $f(\xi) \ge \inf f(x) \quad (x > \xi)$
\end{proof}

\begin{theorem}[Т2. О монотонной непрерывной функции]
    Если $f(x)$ монотонна и непрерывна на интервале $(a; b)$, то она принимает на $(a; b)$
    все значения между $f(a + 0)$ и $f(b - 0)$
\end{theorem}
\begin{proof}
    Пусть $f(x)$ монотонно неубывает на $(a; b)$ $\implies$
    по Т1 $\exists f(a + 0) = \inf f(x), \, \exists f(b - 0) = \sup f(x) \quad (x \in (a; b))$

    Пусть число $\gamma:\: f(a + 0) < \gamma < f(b - 0) \implies$ по признаку sup и inf ($\prop{2}$)
    $\implies \forall \eps > 0$
    \[ \exists x_1 \in (a; b): \: f(a + 0) < f(x_1) < f(a + 0) + \eps \]
    \[ \exists x_2 \in (a; b): \: f(b - 0) - \eps < f(x_2) < f(b - 0) \]

    Возьмём $\eps > 0$ таким, чтобы $f(a + 0) + \eps < \gamma < f(b - 0) - \eps$
    $\implies f(x_1) < \gamma < f(x_2)$

    Но $f(x)$ непрерывна на $[x_1; x_2] \implies$ по Теореме Коши о промежуточном значении
    \[ \exists x_0 \in [x_1; x_2] \subset (a; b) \quad f(x_0) = \gamma \]
\end{proof}

\begin{theorem}[Т3. О существовании обратной функции]
    Если $f(x)$ непрерывна и строго монотонна на $[a; b]$, 
    $A = \inf f(x), B = \sup f(x) \quad (x \in [a; b])$,
    тогда на $[A; B]$ существует непрерывная строго монотонная функция
    $\varphi(x)$, обратная функции $f(x)$.
\end{theorem}
\begin{proof}
    \begin{enumerate}
        \item 
            Существование
        
            $f(x)$ непрерывна и строго монотонна на $[a; b]$ $\implies$ по Т2 она может
            принимать любое значение $\gamma$ между $A$ и $B$ и только один раз, то есть каждому
            значению можно сопоставить только один аргумент $\implies$ на $[A; B]$ существует
            обратная $y = f(x)$ функция $\varphi(x)$.

        \item
            Монотонность

            Пусть $f(x)$ возрастающая $\implies \forall x_1 < x_2 \quad f(x_1) < f(x_2)$ и наоборот.

            Обозначим $y_1 = f(x_1) \implies x_1 = \varphi(y_1); \; y_2 = f(x_2) \implies x_2 = \varphi(y_2)$
            $\implies \forall y_1 < y_2 \quad \varphi(y_1) < \varphi(y_2) \implies x = \varphi(y)$ -- возрастающая.
        
        \item
            Непрерывность

            $x_0 \in (a; b)$

            Возьмём $\eps > 0$, что $(x_0 - \eps; x_0 + \eps) \subset (a; b)$

            Обозначим $y_0 = f(x_0), \, y_1 = f(x_0 - \eps), \, y_2 = f(x_0 + \eps)$
            и выберем $\delta = \min (y_2 - y_0; y_0 - y_1)$

            *рисунок*

            Тогда, если $|y - y_0| < \delta \implies 
            y_1 = y_0 - (y_0 - y_1) \le y_0 - \delta < y_0 + \delta \le y_0 + (y_2 - y_0) = y_2$
            $\implies (y_0 - \delta; y_0 + \delta) \subset (y_1; y_2) \implies$ при
            $|y - y_0| < \delta$ получим $x_0 - \eps = \varphi(y_1) < \varphi(y) < \varphi(y_2) = x_0 + \eps$
            $\implies |\varphi(y) - \varphi(y_0)| < \eps \implies$ по определению $\varphi(y)$
            непрерывна в точке $y_0$; $y_0 \in (A; B)$, а в силу того, что $x_0$ была выбрана
            произвольно $\implies y_0$ произвольна $\implies$ получаем $\varphi(y)$ непрерывна на $(A; B)$.

            В концах $A$, $B$ непрерывность доказывается аналогично.
        \end{enumerate}
\end{proof}

\textbf{Замечание}

В Теоремах отрезок $[a; b]$ можно заменить на $(a; b)$ или $[a; b)$ или $(a; b]$. Это же касается $[A; B]$.

%%%%%%%%%%%%%%%%%%
