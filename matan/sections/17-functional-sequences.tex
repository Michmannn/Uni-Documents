\section{Функциональные последовательности и ряды}

\begin{definition}
    Если любому $n \in \N$ поставить в соответствие функцию $f_n(x)$
    определённую на $A$ ($A \subset \R$ или $A \subset \R^m$), то говорится,
    что на множестве $A$ задана функциональная последовательность $\{ f_n(x) \}$
\end{definition}

\begin{definition}
    Зафиксируем $x_0 \in A$. Если сходится числовая последовательность
    $\{ f_n(x_0) \}$, то говорят, что функциональная последовательность
    $\{ f_n(x) \}$ сходится в точке $x_0$.

    А если функциональная последовательность $\{ f_n(x) \}$ сходится в
    каждой точке множества $A$, то говорят, что $\{ f_n(x) \}$ сходится на
    множестве $A$ (поточечно). И это значит, что существует предельная
    функция $f(x) = \dslimn f_n(x)$

    (То есть $\forall x \in A \quad \forall \varepsilon > 0 \quad
    \exists n_{\varepsilon,x} \quad \forall n \geq n_{\varepsilon,x}$
    выполняется $|f_n(x) - f(x)| < \varepsilon$)
\end{definition}

\begin{definition}
    Говорят, что функциональная последовательность $\{ f_n(x) \}$ сходится
    равномерно на множестве $A$ к $f(x)$, если
    $\forall \varepsilon > 0 \quad \exists n_\varepsilon \quad
    \forall n \geq n_\varepsilon \quad \forall x \in A$ выполняется
    $|f_n(x) - f(x)| < \varepsilon$
\end{definition}

\begin{example}
    Равномерная сходимость -- условие более жёсткое, чем поточечная. Из того,
    что функциональная последовательность сходится равномерно $\implies$
    она сходится и поточечно. Обратное неверно.

    \[ f_n(x) = x^n \quad A = (0; 1) \]
    Имеем $\exists \dslimn x^n = 0 \implies f(x) = 0 \quad (x \in A)$.
    Но $x^n$ сходится к 0 на интервале $(0; 1)$ только поточечно, но не
    равномерно.

    $\exists \varepsilon_0 = \frac{1}{4}$ ($\varepsilon_0$ достаточно взять
    конкретным положительным $< \frac{1}{2}$).
    $\forall n \quad \exists x_n = \sqrt[n]{0.5} \in (0; 1): \: f_n(x_n) - f(x) =
    \sqrt[n]{0.5} - 0 = 0.5 > \varepsilon_0 = 0.25$
\end{example}

\begin{theorem}[Критерий Коши для равномерной сходимости функциональной последовательности]
    Для того, чтобы функциональная последовательность $\{ f_n(x) \}$ равномерно
    сходилась на множестве $A$ $\iff$ 
    $\forall \varepsilon > 0 \quad \exists n_\varepsilon: \:
    \forall m \geq n_\varepsilon \quad \forall n \geq n_\varepsilon \quad
    \forall x \in a$ выполняется $|f_n(x) - f_m(x)| < \varepsilon$
\end{theorem}
\begin{proof}
    \begin{enumerate}
        \item 
            Необходимость

            Пусть $\{ f_n(x) \}$ равномерно сходится на множестве $A$ к $f(x)$
            $\bydef \forall \varepsilon_1 > 0 \quad \exists n_{\varepsilon_1}: \:
            \forall n \geq n_{\varepsilon_1} \quad \forall x \in A$ выполняется
            $|f_n(x) - f(x)| < \varepsilon_1$

            Возьмём $n \geq n_{\varepsilon_1}, \, m \geq n_{\varepsilon_1}, 
            x \in A \implies \begin{cases}
                |f_n(x) - f(x)| < \varepsilon_1 \\
                |f_m(x) - f(x)| < \varepsilon_1 \\
            \end{cases} \implies$ 
            рассмотрим $|f_n(x) - f(x) + f(x) - f_m(x)| \leq 
            |f_n(x) - f(x)| + |f_m(x) - f(x)| < 2 \varepsilon_1 \implies
            \forall \varepsilon > 0 \quad 
            \exists n_\varepsilon = n_{\varepsilon_1}: \:
            \forall n \geq n_\varepsilon \quad \forall m \geq n_\varepsilon
            \quad \forall x \in A$ выполняется $|f_n(x) - f_m(x)| < \varepsilon$

        \item 
            Достаточность

            Пусть $\forall \varepsilon_1 > 0 \quad \exists n_{\varepsilon_1}: \:
            \forall n \geq n_{\varepsilon_1} \quad
            \forall m \geq n_{\varepsilon_1} \quad \forall x \in A$ выполняется
            $|f_n(x) - f_m(x)| < \varepsilon_1 \quad (*)$

            Зафиксируем $x \in A$. По критерию Коши сходимости числовой 
            последовательности $\implies \{ f_n(x) \}$ в фиксированной точке
            $x$ сходится. И это справедливо $\forall x \in A$
            $\implies \{ f_n(x) \}$ сходится поточечно на $A$ к 
            некоторой $f(x)$, то есть $\exists \dslimn f_n(x) = f(x)$

            Перейдём в $(*)$ к пределу при $m \approach{} \infty$.
            Получим $|f_n(x) - f(x)| \leq \varepsilon_1$. Тогда
            $\forall \varepsilon > 0$ возьмём $0 < \varepsilon_1 < \varepsilon$.
            Получим $\exists n_\varepsilon = n_{\varepsilon_1}: \:
            \forall n \geq n_\varepsilon \quad \forall x \in A$ выполняется
            $|f_n(x) - f(x)| < \varepsilon$ $\bydef \{ f_n(x) \}$ равномерно
            сходится на $A$ к $f(x)$.
    \end{enumerate}
\end{proof}

\begin{theorem}[об изменении порядка предельного перехода в функциональной последовательности]
    Пусть функциональная последовательность $f_n(x)$ сходится равномерно
    на $A$ к $f(x)$ и пусть $\exists \dslim_{x \to a} f_n (x) = b_n$.
    Тогда $\exists \dslim_{x \to a} f(x) = b$, где $b = \dslimn b_n$
    (то есть $\dslim_{x \to a} (\dslimn f_n(x)) = \dslimn (\dslim_{x \to a} f_n(x))$)
\end{theorem}
\begin{proof}
    $\{ f_n(x) \}$ равномерно сходится к $f(x)$ на $A$ $\implies$ по критерию
    Коши равномерной сходимости функциональной последовательности
    $\forall \varepsilon_1 > 0 \quad \exists n_{\varepsilon_1} \quad
    \forall n \geq n_{\varepsilon_1} \quad \forall m \geq n_{\varepsilon_1} \quad
    \forall x \in A$ выполняется $|f_n(x) - f_m(x)| < \varepsilon_1 \quad (*)$

    Перейдём в $(*)$ к пределу при $x \approach{} a$ 
    $\implies |b_n - b_m| \leq \varepsilon_1 \quad (**)$. 
    Тогда $\forall \varepsilon > 0$ выбрав $0 < \varepsilon_1 < \varepsilon$
    $\implies \exists n_\varepsilon = n_{\varepsilon_1}: \:
    \forall n \geq n_\varepsilon \quad \forall m \geq n_\varepsilon$ выполняется
    $|b_n - b_m| < \varepsilon \implies$ по критерию Коши сходимости
    числовой последовательности $\{ b_n \}$ сходится 
    $\implies \exists \dslimn b_n \dn b$

    Перейдём в $(**)$ к пределу при $m \approach{} \infty$
    $\implies |b_n - b| \leq \varepsilon_1 \quad \forall n \geq n_{\varepsilon_1}$.
    И перейдём в $(*)$ к пределу при $m \approach{} \infty$
    $\implies |f_n(x) - f(x)| \leq \varepsilon_1 \quad 
    \forall n \geq n_\varepsilon \quad \forall x \in A$

    Рассмотрим $|f(x) - b| 
    = |f(x) - f_n(x) + f_n(x) - b_n + b_n - b| 
    \leq |f(x) - f_n(x)| + |f_n(x) - b_n| + |b_n - b|
    \leq 2 \varepsilon_1 + |f_n(x) - b_n| 
    \quad \forall x \in a \quad \forall n \geq n_\varepsilon$

    Зафиксируем $n \geq n_{\varepsilon_1}$. Так как по условию
    $\dslim_{x \to a} f_n(x) = b_n \bydef$ для нашего 
    $\varepsilon_1 > 0 \quad \exists \delta_{\varepsilon_1} > 0: \:
    \forall x \in a: \: 0 < |x - a| < \delta_{\varepsilon_1}$ выполняется
    $|f_n(x) - b_n| < \varepsilon_1$.

    Тогда получаем $|f(x) - b| < 3 \varepsilon_1 \dn \varepsilon
    \implies \forall \varepsilon > 0 \quad 
    \exists \delta_\varepsilon = \delta_{\varepsilon_1} > 0: \:
    \forall x \in A: \: 0 < |x - a| < \delta_\varepsilon$ выполняется
    $|f(x) - b| < \varepsilon$ $\implies$ по определению Коши предела
    функции в точке $\exists \dslim_{x \to a} f(x) = b$
\end{proof}

\begin{corollary}
    Если $\forall n \in \N \quad f_n(x)$ непрерывна в $x_0 \in A$ и
    функциональная последовательность $\{ f_n(x) \}$ сходится равномерно на $A$
    к $f(x)$ $\implies$ $f(x)$ непрерывна в точке $x_0$
\end{corollary}
\begin{proof}
    Если $x_0$ -- предельная точка $A$, то по предыдущей теореме $\implies$
    $\dslim_{x \to x_0} f(x) 
    = \dslim_{x \to x_0} (\dslimn f_n(x))
    = \dslimn (\dslim_{x \to x_0} f_n(x))
    = \dslimn f_n (x_0) = f(x_0)$ 
    $\bydef$ функция $f(x)$ непрерывна в точке $x_0$

    Если $x_0$ -- изолированная точка множества $A$, то $f(x)$ непрерывна в
    точке $x_0$ по определению.
\end{proof}

\begin{theorem}[о почленном дифференцировании функциональной последовательности]
    Пусть $\forall n \in \N$ функция $f_n(x)$ дифференцируема на $[a; b]$,
    причём последовательность $\{ f_n'(x) \}$ равномерно сходится на $[a; b]$,
    а функциональная последовательность $\{ f_n(x) \}$ сходится в точке
    $x_0 \in [a; b]$. Тогда функциональная последовательность $\{ f_n(x) \}$
    сходится на $[a; b]$ к некоторой дифференцируемой на $[a; b]$ 
    функции $f(x)$, причём $f'(x) = \dslimn f_n' (x)$
\end{theorem}
\begin{proof}
    Рассмотрим
    \[
        \frac{\Delta f_n(x_0)}{\Delta x}
        = \frac{f_n(x) - f_n(x_0)}{x - x_0}, 
        \quad \Delta x = x - x_0, \quad x \ne x_0
    \]

    Докажем, что функциональная последовательность 
    $\{ \frac{\Delta f_n(x_0)}{\Delta x} \}$ сходится равномерно на $[a; b]$.
    По критерию Коши
    \[
        \frac{\Delta f_n(x_0)}{\Delta x} - \frac{\Delta f_m(x_0)}{\Delta x}
        = \frac{\Delta (f_n - f_m)(x_0)}{\Delta x}
        = \frac{(f_n(x) - f_m(x)) - (f_n(x_0) - f_m(x_0))}{\Delta x}
    \]

    По теореме Лагранжа $\exists \xi$ между $x$ и $x_0$ $\implies$
    \[
        \frac{1}{\Delta x} \cdot 
        \frac{d}{dx} (f_n (x) - f_m(x)) \Big|_{x=\varepsilon} \cdot (x - x_0)
        = f_n'(\xi) - f_m'(\xi)
    \]

    Так как $\{ f_n'(x) \}$ равномерно сходится на $[a; b]$ $\implies$
    по критерию Коши равномерной сходимости функциональной последовательности
    $\forall \varepsilon > 0 \quad \exists n_\varepsilon \quad
    \forall n \geq n_\varepsilon \quad \forall m \geq n_\varepsilon \quad
    \forall x \in [a; b]$ выполняется $|f_n'(x) - f_m'(x)| < \varepsilon$
    $\implies \forall \varepsilon > 0 \quad \exists n_\varepsilon \quad
    \forall n \geq n_\varepsilon \quad \forall m \geq n_\varepsilon \quad
    \forall x \in [a; b]$ выполняется
    $\left| \frac{\Delta f_n(x_0)}{\Delta x} - \frac{\Delta f_m(x_0)}{\Delta x} \right| < \varepsilon$
    $\implies$ по критерию Коши равномерной сходимости функциональной
    последовательности $\left\{ \frac{\Delta f_n(x_0)}{\Delta x} \right\}$
    равномерно сходится на $[a; b]$.

    Рассмотрим $f_n(x) = f_n(x_0) + \frac{\Delta f_n(x_0)}{\Delta x} \cdot \Delta x$.
    $f_n(x_0)$ сходится при $n \approach{} \infty$. 
    $\frac{\Delta f_n(x_0)}{\Delta x}$ сходится при $n \approach{} \infty$ в
    $\forall x \in [a; b]$.
    Следовательно, $\{ f_n(x) \}$ сходится на $[a; b]$.

    Обозначим $\dslimn f_n(x) = f(x)$

    Тогда
    \[
        \dslimn \frac{\Delta f_n(x)}{\Delta x}
        = \dslimn \frac{f_n(x) - f_n(x_0)}{x - x_0}
        = \frac{f(x) - f(x_0)}{x - x_0} = \frac{\Delta f(x_0)}{\Delta x}
    \]

    Рассмотрим 
    \[
        \forall x_0 \in [a; b] \quad \exists \dslimn f_n'(x)
        = \dslimn \left( \dslim_{x \to x_0} \frac{f_n(x) - f_n(x_0)}{x - x_0} \right)
        = \dslimn \left( \dslim_{x \to x_0} \frac{\Delta f_n(x_0)}{\Delta x} \right)
    \]

    По теореме об изменении порядка предельного перехода:
    \[
        \dslim_{x \to x_0} \left( \dslimn \frac{\Delta f_n(x_0)}{\Delta x} \right)
        = \dslim_{x \to x_0} \frac{\Delta f(x_0)}{\Delta x} = f'(x_0)
    \]
    $\implies$ $f(x)$ дифференцируема на $[a; b]$ и $f'(x) = \dslimn f_n'(x)$
\end{proof}

\begin{remark}
    Аналогично можно доказать теорему для функциональной последовательности
    многих переменных:

    Если $\forall n \in \N \quad f_n(x), \quad x \in A \subset \R^m$ определены на
    выпуклом множестве $A$, имеют на $A$ частные производные 
    $\frac{d f_n}{d x_k} (x)$ при фиксированном $k$ и
    $\{ f_n(x) \}$ и $\{ \frac{d f_n}{d x_k} (x) \}$ равномерно сходятся на
    $A$, то функция $f(x) = \dslimn f_n(x)$ имеет на $A \quad$ 
    $\frac{df}{dx_k} (x)$ 
    и $\frac{df}{dx_k} (x) = \dslimn \frac{df_n}{dx_k} (x)$ 
\end{remark}

\begin{theorem}[о почленном интегрировании функциональной последовательности]
    Если $\forall n \in \N \quad f_n(x)$ интегрируется на $[a; b]$ и $\{ f_n(x) \}$
    равномерно сходится на $[a; b]$ к некоторой $f(x)$, то $f(x)$ интегрируется
    на $[a; b]$ и
    \[ \int_a^b f(x) dx = \dslimn \int_a^b f_n(x) dx \]
    \[ \int_a^b \dslimn f_n(x) dx = \dslimn \int_a^b f_n(x) dx \]
\end{theorem}
\begin{proof}
    $\{ f_n(x) \}$ равномерно сходится на $[a; b]$ к 
    $f(x) \bydef \forall \varepsilon > 0 \quad \exists n_\varepsilon \quad
    \forall n \geq n_\varepsilon \quad \forall x \in [a; b]$ выполняется
    $| f_n(x) - f(x) | < \varepsilon$. Зафиксируем $n$, удовлетворяющий
    $n \geq n_\varepsilon$. Так как $f_n(x)$ интегрирума на $[a; b] \implies$
    по критерию интегрируемости для нашего 
    $\varepsilon > 0 \quad \exists \delta_\varepsilon > 0: \:$
    $\forall$ разбиения $R$ отрезка такого, что $\Delta < \delta_\varepsilon$ --
    диаметр разбиения, выполняется
    \[ \dssum_{i = 0}^{n - 1} \omega_{i,n} \cdot \Delta x_i < \varepsilon \]
    \[ \omega_{i,n} = \sup |f_n(x') - f_n(x'')|, \quad x', x'' \in [x_i; x_{i + 1}] \]

    Колебание $f_n(x)$ на $i$-м отрезке разбиения $\Delta x_i = x_{i + 1} - x_i$

    Обозначим $f \omega_i = \sup |f(x') - f(x'')|, \quad x', x'' \in [x_i; x_{i + 1}]$.
    Оценим $\dssum_{i = 0}^{n - 1} \omega_{i,n} \cdot \Delta x_i$:

    Пусть $x', x'' \in [a; b]$

    \[ 
        |f(x') - f(x'')| = 
        |(f(x') - f_n(x')) + (f_n(x') + f_n(x'')) + (f_n(x'') - f(x'')) <
    \]
    \[
        < |f_n(x') - f(x')| + |f_n(x') - f_n(x'')| + |f_n(x'') - f(x'')| <
        |f_n(x') - f_n(x'')| < 2\varepsilon 
    \]

    Перейдём к верхней грани при всех $x', x'' \in [a; b] \implies$
    \[ \omega_i \leq \omega_{i,n} + 2\varepsilon \; \Big| \cdot \Delta x_i \]
    \[ \omega_i \Delta x_i \leq \omega_{i,n} \Delta x_i + 2\varepsilon \Delta x_i \; \Big| \dssum_{i=0}^{n-1}\]
    \[ 
        \dssum_{i=0}^{n-1} \omega_i \Delta x_i \leq 
        \dssum_{i=0}^{n-1} \omega_{i,n} \Delta x_i + 
        2\varepsilon (b - a) < \varepsilon (1 + 2(b - a)) \dn \varepsilon^*
    \]

    $\implies \forall \varepsilon^* > 0 \quad \exists \delta_{\varepsilon^*} =
    \delta_\varepsilon > 0 \quad \forall R: \Delta < \delta_{\varepsilon^*}$
    выполняется $\dssum_{i=0}^{n-1} \omega_i \Delta x_i < \varepsilon^*$
    $\implies$ по критерию интегрируемости, $f(x)$ интегрируется на $[a; b]$ и
    $\forall n \geq n_\varepsilon$
    \[ 
        \left| \int_a^b f_n(x) dx - \int_a^b f(x) dx \right| =
        \left| \int_a^b (f_n(x) - f(x)) dx \right| \leq
        \int_a^b |f_n(x) - f(x)| dx < \varepsilon (b - a) 
    \]
    $\bydef \exists \dslimn \int_a^b f_n(x) dx = \int_a^b f(x) dx$
\end{proof}

\begin{definition}
    Говорят, что функциональная последовательность $\{ f_n(x) \}$ равномерно
    ограничена на $[a; b]$, если $\exists C > 0 \quad \forall n \in \N \quad
    \forall x \in [a; b]$ выполняется $|f_n(x)| \leq C$
\end{definition}

\begin{definition}
    Говорят, что функциональная последовательность $\{ f_n(x) \}$ является
    равностепенно непрерывной на $[a; b]$, если $\forall \varepsilon > 0 \quad
    \exists \delta_\varepsilon > 0: \quad \forall x', x'' \in [a; b]: \quad
    |x' - x''| < \delta_\varepsilon \quad \forall n \in \N \quad
    |f_n(x') - f_n(x'')| < \varepsilon$
\end{definition}

\begin{theorem}[Арцела]
    Из всякой равномерно ограниченной и равностепенной непрерывной
    функциональной последовательности на $[a; b]$ можно выделить
    равномерно сходящуюся подпоследовательность на $[a; b]$
\end{theorem}
\begin{proof}
    $\{ f_n(x) \}$ равномерно ограничена на $[a; b] \implies$ графики всех
    функций последовательности заключены на $[a; b]$ в прямоугольник, 
    ограниченный $y = -c, \, y = c, \, x = a, \, x = b$.
    
    $\{ f_n(x) \}$ равностепенно непрерывна на $[a; b] \implies 
    \forall \varepsilon > 0$ (т.е. и для $\varepsilon_0 > 0: \: \frac{c}{\varepsilon_0} \in \N$)
    $\exists \delta_0 = \delta (\varepsilon_0) > 0: \: \forall x', x'' \in [a; b]: \:
    |x' - x''| < \delta_0 \quad \forall n \in \N \quad |f_n(x') - f_n(x'')| < \varepsilon_0$

    Разделим $[a; b]$ на равные части длиной $< \delta_0$, а $[-c; c]$ на
    равные части длиной $\varepsilon_0$.

    \textbf{*рисунок*}

    Тогда график каждой функции в пределах полосы $x_i \leq x \leq x_{i + 1}$
    может проходить не более, чем по двум соседним прямоугольникам.

    Рассмотрим $\varepsilon_k = \frac{\varepsilon_0}{2^k}, \, k = 0,1,2,\dots$

    (для $\varepsilon_k = 0 \quad \exists \delta_k > 0 \quad 
    \forall x', x'' \in [a; b]: \: |x' - x''| < \delta_k$ выполняется
    $|f_n(x') - f_n(x'')| < \varepsilon_k$)

    Тогда при $k = 0$ в первой слева полосе $x_0 \leq x \leq x_1$ 
    ($x_0 = a, |x_1 - x_0| < \delta_0$) по крайней мере в одной паре соседних
    прямоугольников проходит бесконечное количество графиков $f_n(x)$.
    Выделим из $f_n(x)$ все такие функции. Тогда во второй слева полосе
    $x_1 \leq x \leq x_2$ с помощью аналогичных рассуждений выделим из уже
    выделенных функций последовательность. График, который проходит по двум
    соседним прямоугольникам второй полосы (и двум соседним первой полосы)
    переходит от одной полосы к другой продолжая выделять последовательность.
    И в последней полосе мы получим последовательность $\{ f_{k_n} (x) \}$,
    график которой проходит по двум соседним прямоугольникам каждой полосы.

    Обозначим
    \[ 
        f_{k_n} = f_{n,0}(x) \implies 
        \forall m,n,x \quad |f_{n,0}(x) - f_{m,0}(x)| \leq 2\varepsilon_0 
    \]

    Теперь берём $k = 1$ и аналогично рассуждая из последовательности
    $\{ f_{n, 0}(x) \}$ выделим $\{ f_{n,1}(x) \}$ так, что
    \[ |f_{n,1}(x) - f_{m,1}(x)| \leq 2 \varepsilon_1 \]

    Повторяя процесс при $k = 2,3,\dots$ мы будем выделять 
    $\{ f_{n,2}(x) \}, \{ f_{n,3}(x) \}$ и так далее. Причём каждая из них
    является подпоследовательностью предыдущй, причём 
    \[ 
        \forall l \geq k \quad \forall p \geq k \quad \forall m,n,k \quad 
        |f_{n,l}(x) - f_{m,p}(x)| < 2\varepsilon_k
    \]

    Рассмотрим $\{ f_{n,n}(x) \}$. Тогда $\forall \varepsilon > 0$ выберем
    $N_\varepsilon: \: 2\varepsilon_{N_\varepsilon} < \varepsilon$ и
    $\forall n \geq N_\varepsilon \quad \forall m \geq N_\varepsilon \quad
    \forall x \in [a; b] \quad |f_{n,n}(x) - f_{m,m}(x)| < \varepsilon$ 
    $\implies$ по критерию Коши равномерной сходимости функциональной
    последовательности, $\{ f_{n,n}(x) \}$ сходится равномерно на $[a; b]$.
\end{proof}