\section{Функциональные последовательности и ряды}

\begin{definition}
    Если любому $n \in \N$ поставить в соответствие функцию $f_n(x)$
    определённую на $A$ ($A \subset \R$ или $A \subset \R^m$), то говорится,
    что на множестве $A$ задана функциональная последовательность $\{ f_n(x) \}$
\end{definition}

\begin{definition}
    Зафиксируем $x_0 \in A$. Если сходится числовая последовательность
    $\{ f_n(x_0) \}$, то говорят, что функциональная последовательность
    $\{ f_n(x) \}$ сходится в точке $x_0$.

    А если функциональная последовательность $\{ f_n(x) \}$ сходится в
    каждой точке множества $A$, то говорят, что $\{ f_n(x) \}$ сходится на
    множестве $A$ (поточечно). И это значит, что существует предельная
    функция $f(x) = \dslimn f_n(x)$

    (То есть $\forall x \in A \; \forall \varepsilon > 0 \; 
    \exists n_{\varepsilon,x} \; \forall n \geq n_{\varepsilon,x}$
    выполняется $|f_n(x) - f(x)| < \varepsilon$)
\end{definition}

\begin{definition}
    Говорят, что функциональная последовательность $\{ f_n(x) \}$ сходится
    равномерно на множестве $A$ к $f(x)$, если
    $\forall \varepsilon > 0 \; \exists n_\varepsilon \;
    \forall n \geq n_\varepsilon \; \forall x \in A$ выполняется
    $|f_n(x) - f(x)| < \varepsilon$
\end{definition}

\begin{example}
    Равномерная сходимость -- условие более жёсткое, чем поточечная. Из того,
    что функциональная последовательность сходится равномерно $\implies$
    она сходится и поточечно. Обратное неверно.

    \[ f_n(x) = x^n \quad A = (0; 1) \]
    Имеем $\exists \dslimn x^n = 0 \implies f(x) = 0 \quad (x \in A)$.
    Но $x^n$ сходится к 0 на интервале $(0; 1)$ только поточечно, но не
    равномерно.

    $\exists \varepsilon_0 = \frac{1}{4}$ ($\varepsilon_0$ достаточно взять
    конкретным положительным $< \frac{1}{2}$).
    $\forall n \; \exists x_n = \sqrt[n]{0.5} \in (0; 1): \; f_n(x_n) - f(x) =
    \sqrt[n]{0.5} - 0 = 0.5 > \varepsilon_0 = 0.25$
\end{example}

\begin{theorem}[Критерий Коши для равномерной сходимости функциональной последовательности]
    Для того, чтобы функциональная последовательность $\{ f_n(x) \}$ равномерно
    сходилась на множестве $A$ $\iff$ 
    $\forall \varepsilon > 0 \; \exists n_\varepsilon: \;
    \forall m \geq n_\varepsilon \; \forall n \geq n_\varepsilon \;
    \forall x \in a$ выполняется $|f_n(x) - f_m(x)| < \varepsilon$
\end{theorem}
\begin{proof}
    \begin{enumerate}
        \item 
            Необходимость

            Пусть $\{ f_n(x) \}$ равномерно сходится на множестве $A$ к $f(x)$
            $\bydef \forall \varepsilon_1 > 0 \; \exists n_{\varepsilon_1}: \;
            \forall n \geq n_{\varepsilon_1} \; \forall x \in A$ выполняется
            $|f_n(x) - f(x)| < \varepsilon_1$

        \item 
            Достаточность

            Пусть $\forall \varepsilon_1 > 0 \; \exists n_{\varepsilon_1}: \;
            \forall n \geq n_{\varepsilon_1} \;
            \forall m \geq n_{\varepsilon_1} \; \forall x \in A$ выполняется
            $|f_n(x) - f_m(x)| < \varepsilon_1 \; (*)$

            Зафиксируем $x \in A$. По критерию Коши сходимости числовой 
            последовательности $\implies \{ f_n(x) \}$ в фиксированной точке
            $x$ сходится. И это справедливо $\forall x \in A$
            $\implies \{ f_n(x) \}$ сходится поточечно на $A$ к 
            некоторой $f(x)$, то есть $\exists \dslimn f_n(x) = f(x)$

            Перейдём в $(*)$ к пределу при $m \approach{} \infty$.
            Получим $|f_n(x) - f(x)| \leq \varepsilon_1$. Тогда
            $\forall \varepsilon > 0$ возьмём $0 < \varepsilon_1 < \varepsilon$.
            Получим $\exists n_\varepsilon = n_{\varepsilon_1}: \;
            \forall n \geq n_\varepsilon \; \forall x \in A$ выполняется
            $|f_n(x) - f(x)| < \varepsilon$ $\bydef \{ f_n(x) \}$ равномерно
            сходится на $A$ к $f(x)$.
    \end{enumerate}
\end{proof}

\begin{theorem}[об изменении порядка предельного перехода в функциональной последовательности]
    Пусть функциональная последовательность $f_n(x)$ сходится равномерно
    на $A$ к $f(x)$ и пусть $\exists \dslim_{x \to a} f_n (x) = b_n$.
    Тогда $\exists \dslim_{x \to a} f(x) = b$, где $b = \dslimn b_n$
    (то есть $\dslim_{x \to a} (\dslimn f(x)) = \dslimn (\dslim_{x \to a} f(x))$)
\end{theorem}
\begin{proof}
    $\{ f_n(x) \}$ равномерно сходится к $f(x)$ на $A$ $\implies$ по критерию
    Коши равномерной сходимости функциональной последовательности
    $\forall \varepsilon_1 > 0 \; \exists n_{\varepsilon_1} \;
    \forall n \geq n_{\varepsilon_1} \; \forall m \geq n_{\varepsilon_1} \;
    \forall x \in A$ выполняется $|f_n(x) - f_m(x)| < \varepsilon_1 \quad (*)$

    Перейдём в $(*)$ к пределу при $x \approach{} a$ 
    $\implies |b_n - b_m| \leq \varepsilon_1 \quad (**)$. 
    Тогда $\forall \varepsilon > 0$ выбрав $0 < \varepsilon_1 < \varepsilon$
    $\implies \exists n_\varepsilon = n_{\varepsilon_1}: \;
    \forall n \geq n_\varepsilon \; \forall m \geq n_\varepsilon$ выполняется
    $|b_n - b_m| < \varepsilon \implies$ по критерию Коши сходимости
    числовой последовательности $\{ b_n \}$ сходится 
    $\implies \exists \dslimn b_n \dn b$

    Перейдём в $(**)$ к пределу при $m \approach{} \infty$
    $\implies |b_n - b| \leq \varepsilon_1 \; \forall n \geq n_{\varepsilon_1}$.
    И перейдём в $(*)$ к пределу при $m \approach{} \infty$
    $\implies |f_n(x) - f(x)| \leq \varepsilon_1 \; 
    \forall n \geq n_\varepsilon \; \forall x \in A$

    Рассмотрим $|f(x) - b| 
    = |f(x) - f_n(x) + f_n(x) - b_n + b_n - b| 
    \leq |f(x) - f_n(x)| + |f_n(x) - b_n| + |b_n - b|
    \leq 2 \varepsilon_1 + |f_n(x) - b_n| 
    \quad \forall x \in a \; \forall n \geq n_\varepsilon$

    Зафиксируем $n \geq n_{\varepsilon_1}$. Так как по условию
    $\dslim_{x \to a} f_n(x) = b_n \bydef$ для нашего 
    $\varepsilon_1 > 0 \; \exists \delta_{\varepsilon_1} > 0: \;
    \forall x \in a: \; 0 < |x - a| < \delta_{\varepsilon_1}$ выполняется
    $|f_n(x) - b_n| < \varepsilon_1$.

    Тогда получаем $|f(x) - b| < 3 \varepsilon_1 \dn \varepsilon
    \implies \forall \varepsilon > 0 \; 
    \exists \delta_\varepsilon = \delta_{\varepsilon_1} > 0: \;
    \forall x \in A: \; 0 < |x - a| < \delta_\varepsilon$ выполняется
    $|f(x) - b| < \varepsilon$ $\implies$ по определению Коши предела
    функции в точке $\exists \dslim_{x \to a} f(x) = b$
\end{proof}

\begin{corollary}
    Если $\forall n \in \N \; f_n(x)$ непрерывна в $x_0 \in A$ и
    функциональная последовательность $\{ f_n(x) \}$ сходится равномерно на $A$
    к $f(x)$ $\implies$ $f(x)$ непрерывна в точке $x_0$
\end{corollary}
\begin{proof}
    Если $x_0$ -- предельная точка $A$, то по предыдущей теореме $\implies$
    $\dslim_{x \to x_0} f(x) 
    = \dslim_{x \to x_0} (\dslimn f_n(x))
    = \dslimn (\dslim_{x \to x_0} f_n(x))
    = \dslimn f_n (x_0) = f(x_0)$ 
    $\bydef$ функция $f(x)$ непрерывна в точке $x_0$

    Если $x_0$ -- изолированная точка множества $A$, то $f(x)$ непрерывна в
    точке $x_0$ по определению.
\end{proof}