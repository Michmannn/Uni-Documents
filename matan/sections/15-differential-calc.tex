\section{Дифференциальное исчисление}

\subsection{Производная и дифференциал}

\begin{definition}
    Производной функции $f(x)$ в точке $x_0$ называют следующий предел, если он существует
    \[ \dslim_{x \to x_0} \frac{f(x) - f(x_0)}{x - x_0} = f'(x_0) \]
\end{definition}

\begin{remark}
    Обозначают $f(x) - f(x_0) = \Delta f(x_0)$ и называют приращение функции 
    в точке $x_0$ соответсвует приращению аргумента $x - x_0 = \Delta x \implies$
    \[ f'(x_0) = \dslim_{\Delta x \to 0} \frac{\Delta f(x_0)}{\Delta x} \]
\end{remark}

\begin{remark}
    \[ 
        \dslim_{x \to x_0 - 0} \frac{\Delta f(x_0)}{\Delta x} = f_-'(x_0) 
        \quad (\textnormal{левая производная в точке } x_0)
    \]
    \[ 
        \dslim_{x \to x_0 + 0} \frac{\Delta f(x_0)}{\Delta x} = f_+'(x_0) 
        \quad (\textnormal{правая производная в точке } x_0)
    \]

    Очевидно, $\exists f'(x_0) \iff \exists f_-'(x_0) = f_+'(x_0)$
\end{remark}

\begin{definition}
    Говорят, что функция $f(x)$ дифференцируема в точке $x_0$, если её приращение
    $\Delta f(x_0)$ соответствует приращению аргумента $\Delta x$ можно представить в виде
    \[ \Delta f(x_0) = A * \Delta x + O(\Delta x) \quad \textnormal{при } \Delta x \approach{} 0, \, A - const \]
\end{definition}

\begin{example}
    \[ f(x) = x^2 \]
    \[ 
        \Delta f(x_0) = f(x_0 + \Delta x) - f(x_0) = (x_0 - \Delta x)^2 - x_0^2 =
        x_0^2 + 2 x_0 * \Delta x + (\Delta x) ^ 2 - x_0^2 = 2 x_0 * \Delta x + (\Delta x)^2
    \]
    \[ \Delta f(x_0) = 2 x_0 * \Delta x + (\Delta x)^2 \]

    Так как $\dslim_{\Delta x \to 0} \frac{(\Delta x)^2}{\Delta x} = 0$
    $\implies (\Delta x)^2 = o(\Delta x)$ при $\Delta x \approach{} 0$ по определению.

    $\Delta f(x_0) = 2 x_0 * \Delta x + o(\Delta x) \implies f(x) = x^2$
    дифференцируема в точке $x_0$, $df(x_0) = 2 x_0 * \Delta x$
\end{example}

\subsection{Геометрический смысл производной и дифференциала}

Проведём через точки $P_0(x_0; f(x_0))$, $P(x_0 + \Delta x; f(x_0 + \Delta x))$ секущую.
Получим прямоугольный треугольник.

\[ \frac{f(x_0 + \Delta x) - f(x_0)}{\Delta x} = \tan \alpha \]
\[ y = y_0 + \tan \alpha * (x - x_0) \textnormal{ -- уравнение секущей} \]

При этом $\Delta x \approach{} 0 \implies P \approach{} P_0$.

При этом в пределе секущая займёт положение, называемое касательной к графику функции $y = f(x)$ в точке $x_0$
и $\tan \alpha \approach{} \tan \alpha_0$, где $\alpha_0$ -- угол наклона касательной к положительной оси $Ox$.

То есть $\dslim_{\Delta x \to 0} \frac{f(x_0 + \Delta x) - f(x_0)}{\Delta x} = \tan \alpha_0 = f'(x_0)$

При этом уравнение касательной $y = y_0 + \tan \alpha_0 * (x - x_0)$, то есть $y = f(x_0) + f'(x_0) (x - x_0)$

Дифференциал $df(x_0) = A * \Delta x = f'(x_0) \Delta x = \tan \alpha_0 * \Delta x$

*рисунок*

То есть $df(x)$ -- то приращение, которое получила бы функция, если бы после точки $x_0$ функция стала бы
изменяться по графику своей касательной. Если $\Delta x << 1$ (много меньше), то $\Delta f(x_0) \approx df(x_0)$

\subsection{Дифференцируемость}

\begin{theorem}[Критерий дифференцируемости функции в точке]
    Для того, чтобы $f(x)$ была дифференцируема в точке $x_0 \iff $ чтобы в точке $x_0$ существовала $f'(x_0)$.
\end{theorem}
\begin{proof}
    \begin{enumerate}[label=\alph*)]
        \item 
            Необходимость

            $f(x)$ дифференцируема в точке $x_0 \implies$ по определению.
            \[ \Delta f(x_0) = A * \Delta x + O(\Delta x), \, \Delta x \approach{} 0 \quad \frac{1}{\Delta x} \]
            \[ 
                \dslim_{\Delta x \to 0} \frac{\Delta f(x_0)}{\Delta x} =
                \dslim_{\Delta x \to 0} \left( A + \frac{o(\Delta x)}{\Delta x} \right) = A
            \]

            $\bydef \exists f'(x_0) = A$
        
        \item
            Достаточность

            \[ 
                \exists f'(x_0) = \dslim_{\Delta x \to 0} \frac{\Delta f(x_0)}{\Delta x} \implies
                \dslim_{\Delta x \to 0} \left( \frac{\Delta f(x_0)}{\Delta x} - f'(x_0) \right) = 0
            \]
            \[ \textnormal{То есть: } \frac{\Delta f(x_0)}{\Delta x} - f(x_0) = o(1) \quad \textnormal{при } \Delta x \approach{} 0 \]
            \[ \implies \Delta f(x_0) = f'(x_0) \Delta x + o(1) \Delta x, \quad \Delta x \approach{} 0 \]
            \[ \implies \Delta f(x_0) = f'(x_0) \Delta x + o(\Delta x), \quad \Delta x \approach{} 0 \]

            $\bydef$ $f(x)$ дифференцируема в точке $x_0$.
    \end{enumerate}
\end{proof}

\begin{theorem}[Связь непрерывности и дифференцируемости]
    Если $f(x)$ дифференцируема в точке $x_0 \implies f(x)$ непрерывна в точке $x_0$.
\end{theorem}
\begin{proof}
    $f(x)$ -- дифференцируема в точке $x_0 \implies $

    \[ \Delta f(x_0) = f'(x_0) * \Delta x + o(\Delta x), \quad \Delta x \approach{} 0 \]
    \[ \implies \dslim_{\Delta x \to 0} \Delta f(x_0) = \dslim{\Delta x \to 0} (f'(x_0) \Delta x + \frac{o(\Delta x)}{\Delta x} * \Delta x) = 0 \]
    \[ \dslim_{x - x_0 \to 0} (f(x) - f(x_0)) = 0 \implies \dslim_{x \to x_0} f(x) = f(x_0) \]

    $\bydef$ $f(x)$ -- непрерывна в точке $x_0$
\end{proof}

\begin{example}
    \textbf{Обратное неверно} *рисунок*

    $y = |x|$

    $y = \begin{cases}
        -x, &\textnormal{если } x < 0 \\
         x, &\textnormal{если } x \ge 0
    \end{cases}$

    $y$ -- непрерывна в точке $x = 0$, но не дифференцируема.

    $f_-(0) = -1,\, f_+(0) = 1 \implies \not\exists f'(0)$
\end{example}

\begin{theorem}[О дифференцируемости линейной комбинации, произведения, частного]
    Пусть функции $u(x)$ и $v(x)$ дифференцируемы в точке $x_0$.
    Тогда в точке $x_0$ дифференцируемы

    \[ c_1 u(x) + c_2 v(x)\]
    \[ u(x) * v(x) \]
    \[ \frac{u(x)}{v(x)} \quad (v(x) \ne 0) \]

    При этом 
    \[(c_1 u(x) + c_2 v(x))' |_{x = x_0} = c_1 u'(x_0) + c_2 v'(x_0)\]
    \[ (u(x) * v(x))' |_{x = x_0} = u'(x_0) * v(x_0) + v'(x_0) * u(x_0)\]
    \[ \left( \frac{u(x)}{v(x)} \right)' |_{x = x_0} = \frac{u'(x_0) * v(x_0) - u(x_0) * v'(x_0)}{v^2(x_0)} \]
\end{theorem}
\begin{proof}
    По условию $\implies \begin{cases}
        \exists \dslim_{\Delta x \to 0} \frac{u(x_0 + \Delta x) - u(x_0)}{\Delta x} = u'(x_0) \\
        \exists \dslim_{\Delta x \to 0} \frac{v(x_0 + \Delta x) - v(x_0)}{\Delta x} = v'(x_0)
    \end{cases}$

    \begin{enumerate}[label=\alph*)]
        \item
            \[ \dslim_{\Delta x \to 0} \frac{(c_1 u(x_0 + \Delta x_0) + c_2 v(x_0 + \Delta x)) - (c_1 u(x_0) + c_2 v(x_0))}{\Delta x} = \]
            \[ 
                = \dslim_{\Delta x \to 0} \left( c_1 \frac{u(x_0 + \Delta x_0) - u(x_0)}{\Delta x} + c_2 \frac{v(x_0 + \Delta x) - v(x_0)}{\Delta x} \right) 
                = c_1 u'(x_0) + c_2 v'(x_0) 
            \]

        \item 
            \[ \dslim_{\Delta x \to 0} \frac{u(x_0 + \Delta x) v(x_0 + \Delta x) - u(x_0) v(x_0)}{\Delta x} = \]
            \[ = \dslim_{\Delta x \to 0} \frac{(u(x_0 + \Delta x) - u(x_0)) v(x_0 + \Delta x) - u(x_0) (v(x_0 + \Delta x) - v(x_0))}{\Delta x} = \]
            \[ = \dslim_{\Delta x \to 0} \left( \frac{(u(x_0 + \Delta x) - u(x_0))}{\Delta x} v(x_0 + \Delta x) + u(x_0) \frac{(v(x_0 + \Delta x) - v(x_0))}{\Delta x} \right) = \]
            \[ = u'(x_0) v(x_0) + u(x_0) v'(x_0) \]

        \item
            \[
                \dslim_{\Delta x \to 0} \frac{ \frac{u(x_0 + \Delta x)}{v(x_0 + \Delta x)} - \frac{u(x_0)}{v(x_0)} }{\Delta x}
                = \dslim_{\Delta x \to 0} \frac{ u(x_0 + \Delta x) v(x_0) - u(x_0) v(x_0 + \Delta x) }{ v(x_0 + \Delta x) v(x_0) \Delta x } =
            \]
            \[ = \dslim_{\Delta x \to 0} \frac{(u(x_0 + \Delta x) - u(x_0)) v(x_0 + \Delta x) - u(x_0) (v(x_0 + \Delta x) - v(x_0))}{ v(x_0 + \Delta x) v(x_0) \Delta x } = \]
            \[ = \dslim_{\Delta x \to 0} \frac{ \frac{(u(x_0 + \Delta x) - u(x_0))}{\Delta x} v(x_0) - u(x_0) \frac{(v(x_0 + \Delta x) - v(x_0))}{\Delta x} }{v(x_0 + \Delta x) v(x_0)} = \]
            \[ = \frac{u'(x_0) v(x_0) - u(x_0) v'(x_0)}{v^2(x_0)} \]
    \end{enumerate}
\end{proof}

\begin{theorem}[О дифференцируемости сложной функции]
    Если функция $\varphi (x)$ дифференцируема в точке $x_0$, а $f(t)$ 
    дифференцируема в точке $t_0 = \varphi (x_0)$ $\implies$
    $f(\varphi (x))$ дифференцируема в точке $x_0$ и
    $(f(\varphi (x)))' |_{x = x_0} = f'(t_0) * \varphi' (x_0)$
\end{theorem}
\begin{proof}
    $\varphi (x)$ дифференцируема в точке $x_0 \implies \Delta \varphi (x_0) = \varphi' (x_0) \Delta x + o(\Delta x), \quad \Delta x \approach{} 0$

    $f(t)$ дифференцируема в точке $t_0 \implies \Delta f(t_0) = f'(t_0) \Delta t + o(\Delta t), \quad \Delta t \approach{} 0$

    Рассмотрим $\Delta f(\varphi (x_0)) = f(\underbrace{\varphi (x)}_t) - f(\underbrace{\varphi (x_0)}_{t_0}) =$

    \[ 
        = f'(t_0) * \Delta \varphi (x_0) + o(\Delta \varphi (x_0))
        = f'(t_0) (\varphi' (x_0) \Delta x + o(\Delta x)) + o(\Delta \varphi (x_0)) =
    \]
    \[ = f'(t_0) \varphi'(x_0) \Delta x + f'(t_0) * o(\Delta x) + o(\Delta \varphi (x_0)) \]

    Осталось доказать, что $f'(t_0) * o(\Delta x) + o(\Delta \varphi (x_0)) = o(\Delta x), \quad \Delta x \approach{} 0$

    Для этого рассмотрим 
    \[
        \dslim_{\Delta x \to 0} \frac{f'(t_0) * o(\Delta x) + o(\Delta \varphi (x_0))}{\Delta x} =
        \dslim_{\Delta x \to 0} \left( f'(t_0) \frac{o(\Delta x)}{\Delta x} + \frac{o(\Delta \varphi (x_0))}{\Delta \varphi (x_0)} * \frac{\Delta \varphi (x_0)}{\Delta x} \right) =
    \]
    \[ \dslim_{\Delta x \to 0} \left( f'(t_0) \frac{o(\Delta x)}{\Delta x} + \frac{o(\Delta \varphi (x_0))}{\Delta \varphi (x_0)} * \left( \varphi' (x_0) + \frac{o(\Delta x)}{\Delta x} \right) \right) = 0\]

    То есть получили, что $\Delta f(\varphi (x_0)) = f'(t_0) * \varphi' (x_0) \Delta x + o(\Delta x), \quad \Delta x \approach{} 0$

    $\bydef f(\varphi (x))$ -- дифференцируема в точке $x_0$ и $(f(\varphi (x)))' |_{x = x_0} = f'(t_0) * \varphi' (x_0)$

\end{proof}

\begin{theorem}[О производной обратной функции]
    Пусть $y = f(x)$ непрерывна, строго монотонна в $(a; b)$ и
    $\exists f'(x_0) \ne 0$, где $x_0 \in (a; b)$.

    Тогда обратная функция $x = \varphi (x)$ имеет в точке $y_0 = f(x_0)$ производную
    \[ \varphi' (y_0) = \frac{1}{f'(x_0)} \]
\end{theorem}
\begin{proof}
    По теореме об обратной функции: $x= \varphi (x)$ существует, непрерывна и
    $\Delta x = \Delta \varphi (y_0) \approach{} 0$ при $\Delta y = \Delta f(x_0) \approach{} 0$

    Тогда рассмотрим $\dslim_{\Delta y \to 0} \frac{\Delta \varphi (y_0)}{\Delta y} =
    \dslim_{\Delta x \to 0} \frac{1}{\frac{\Delta f(x_0)}{\Delta x}} = \frac{1}{f'(x_0)}
    \implies \varphi'(y_0) = \frac{1}{f'(x_0)}$
\end{proof}




\textbf{Свойства дифференциалов имеют вид:}
\begin{enumerate}
    \item $df(x_0) = f'(x_0)*\Delta x = f'(x_0)dx$
    \item $d(\cos x) = -\sin x * dx$
    \item $d(c_1*f(x) + c_2*g(x)) = c_1*df(x) + c_2*dg(x)$
    \item $d(f(x)*g(x)) = g(x)df(x) + f(x)dg(x)$
    \item $d(\frac{f(x)}{g(x)}) = \frac{df(x)g(x) - f(x)dg(x)}{g^2(x)}$
    \item $df(\varphi(x)) = f'(\varphi(x)) * \frac{\varphi'(x)dx}{d\varphi(x)} = f'(\varphi)d\varphi$
\end{enumerate}


\subsection{Производные и дифференциалы высших порядков для функции одной переменной}
\begin{definition}
    Пусть $f(x)$ дифференцируема в точке $\forall x \in A$. Таким образом, каждой точке $x$ из множества $A$ можно
    сопоставить $f'(x)$ и таким образом сопоставить функцию $y = f'(x)$, которая называется производной функцией от $f(x)$.
    Если эта функция дифференцируема в точке $x$, то её производную мы назовём второй производной функции $f(x)$ и обозначим $f''(x)$
    
    Аналогично, можно определить n-ую производную $f^{(n)}(x) = (f^{(n-1)}(x))'$
    \[(x^\alpha)^{(n)} = \alpha(\alpha-1)...(\alpha-n+1)x^{\alpha-n}\]
    \[(e^x)^{(n)} = e^x, \, (a^x)^{(n)} = a^x \ln^n a\]
    \[(\sin x)^{(n)} = \sin(x+\frac{\pi n}{2}), \, (\cos x)^{(n)} = \cos (x + \frac{\pi n}{2})\]
    \[(c_1*f(x) + c_2*g(x))^{(n)} = c_1*f^{(n)}(x) + c_2*g^{(n)}(x)\]
\end{definition}



\begin{theorem}[Формула Лейбница]
    Пусть $u(x)$ и $v(x)$ $n$ раз дифференцируемы в точке x. Тогда $$(u(x)*v(x))^{(n)} = \sum_{k=0}^n C^k_n \cdot U^{(k)}(x) \cdot V^{(n-k)}(x)$$ 
\end{theorem}
\begin{proof}
    Докажем по ММИ:

    База индукции: $n=1$
    \[(u(x) \cdot v(x))' =  \sum_{k=0}^1 C^k_1 \cdot u^{(k)}(x) \cdot v^{(1-k)}(x)\]
    \[u'(x)\cdot v(x)+u(x)\cdot v'(x) = C^0_1 \cdot u^0(x) \cdot v'(x) + C^1_1 \cdot u'(x) \cdot v^0(x)\]
    Имеем $C^0_1 = 1$, $C^1_1 = 1$, $u^{(0)} = u$, $v^{(0)} = v$ $\implies$ База индукции доказана. Перейдём к шагу индукции:
    

    Пусть $$(u(x) \cdot v(x))^{(m)} = \sum_{k=0}^m C^k_m \cdot u^{(k)}(x) \cdot v^{(m-k)}(x)$$ - верно. Надо доказать, что
    $$(u(x) \cdot v(x))^{(m+1)} = \sum_{k=0}^{m+1} C^k_{m+1} \cdot u^{(k)}(x) \cdot v^{(m+1-k)}(x)$$

    Имеем $$(u(x) \cdot v(x))^{(m+1)} = ((u(x) \cdot v(x))^m)' = (\sum_{k=0}^{m}C^k_m \cdot u^{(k)}(x) \cdot v^{(m-k)}(x))' = \sum_{k=0}^{m}C^k_m (u^{(k)}(x) \cdot v^{(m-k)}(x))' =$$ 
    $$\sum_{k=0}^{m}C^k_m \cdot u^{(k+1)}(x) \cdot v^{(m-k)}(x) + \sum_{k=0}^{m}C^k_m \cdot u^{(k)}(x) \cdot v^{(m-k+1)}(x)$$
    В первой сумме заменим $k$ на $k-1$. Получим следующее:
    $$\sum_{k=1}^{m+1}C^{k-1}_m \cdot u^{(k)}(x) \cdot v^{(m-k+1)}(x) + \sum_{k=0}^{m}C^k_m \cdot u^{(k)}(x) \cdot v^{(m-k+1)}(x) = $$
    $$\sum_{k=1}^{m}(C^{k-1}_m+C^k_m) u^{(k)}(x)\cdot v^{(m+1-k)}(x) + C^m_m \cdot u^{(m+1)}(x) \cdot v^{(0)}(x) + C^0_m \cdot u^{(0)}(x) \cdot v^{(m+1)}(x) =$$
    При замене биномиальных коэффициентов мы получаем:
    $$\sum_{k=0}^{m+1}C^k_{m+1} \cdot u^{(k)}(x) \cdot v^{(m+1-k)}(x)$$
\end{proof}

    $d^{(n)}f(x) = d(d^{(n-1)}f(x)) = f^{(n)}(x) \cdot (dx)^n$

\subsection{Применение производных для изучения поведения функции одной переменной}

\begin{definition}
        Функция $f(x)$ возрастает/убывает в точке $x_0$, если $\exists\delta >0$, $0<|x-x_0|<\delta$, то выполняется $\frac{\Delta f(x_0)}{\Delta x} = \frac{F(x)-F(x_0)}{x-x_0}>0 (<0)$
\end{definition}
\begin{definition}
        Говорят, что функция $f(x)$ имеет в точке $x_0$ локальный максимум (минимум), если $\exists \delta >0$, $0<|x-x_0|<\delta$ выполняется $ f(x)<f(x_0) $ или $(f(x)>f(x_0))$.
        У этих точек есть общее название - локальный экстремум.
\end{definition}


\begin{theorem}[Признак возрастающей (убывающей) функции в точке (Лемма Дарбу)]
        Если в точке $x_0$ $\exists f'(x_0)>0 (<0) \implies f(x)$ возрастает (убывает) в точке $x_0$.
\end{theorem}
\begin{proof}
        Если $f'(x_0)>0$, то $\implies$ (по теореме о сохранении знака)
        \[\lim_{x\to x_0} \frac{f(x)-f(x_0)}{x-x_0}>0\] 
        $\implies \exists$ окрестность точки $x_0$, в которой $\frac{f(x)-f(x_0)}{x-x_0}>0$ $\implies f(x)$ возрастает в точке $x_0$ (для убывания аналогично).
\end{proof}

\begin{theorem}[необходимое условие локального экстремума (Теорема Ферма)]
        Если у $f(x)$ в точке $x_0$ существует локальный экстремум и если $\exists f'(x_0)$, то $f'(x_0)=0$
\end{theorem}
\begin{proof}
        Допустим от противного, что $f'(x_0) \neq 0$ $\implies$ $f'(x_0)>0$ или $f'(x_0)<0$ в точке $x_0$ $\implies f(x)$ возрастает или убывает в точке $x_0$ по Дарбу $\implies$ противоречие.
\end{proof}
\begin{theorem}[Теорема Ролля]
        Пусть $y = f(x)$ непрерывна на $[a,b]$ и дифференцируема в $(a, b)$ и $f(a) = f(b) \implies \exists c \in (a, b)$, где $f'(c)=0$ 
\end{theorem}
\begin{proof}
        *рисунок*
        Так как $f(x)$ непрерывна на отрезке $[a, b]$, то она достигает на $[a, b]$ своего наибольшего и наименьшего значения, то есть $\exists c_1,c_2 \in [a, b]:f(c_1)=min f(x)$ и $f(c_2)=max f(x)$, где $x \in [a, b]$

        1 случай:

        $c_1, c_2$ совпадает с концами отрезка $\implies f(x) = c$, где $c = const \implies f'(x)=0$, $\forall x \in (a, b)$
        
        2 случай:

        $c_1$ или $c_2$ $\in (a, b) \implies$ является точкой локального экстремума, значит по теореме Ферма $\implies f'(c_1) = 0$ (или $f'(c_2) = 0$)   
\end{proof}

    Замечание:

    Если $f(a) = f(b) = 0$, то Теорема Ролля говорит о том, что между любыми двумя нулями функции находится нуль производной.
\begin{theorem}[Теорема Коши о среднем приращении]
        Пусть $\varphi(x)$ и $\psi(x)$ непрерывны на $[a, b]$ и дифференцируемы на $(a, b)$ и $\forall x \in (a, b)$, $(\psi'(x))^2+(\varphi'(x))^2 \neq 0$, и $\varphi(b) \neq \varphi(a)$.
        Тогда $\exists c \in (a, b)$, где $\frac{\psi(b) - \psi(a)}{\varphi(b) - \varphi(a)} = \frac{\psi'(c)}{\varphi'(c)}$.
\end{theorem}
\begin{proof}
        Рассмотрим функцию $f(x) = \frac{\psi(b) - \psi(a)}{\varphi(b) - \varphi(a)} \cdot \varphi(x) - \psi(x)$, которая непрерывна на отрезке $[a, b]$ и дифференцируема на $(a, b)$.
        \[f(a)=\frac{\psi(b) - \psi(a)}{\varphi(b) - \varphi(a)}\varphi(a)-\psi(a) = \frac{\psi(b)\varphi(a)-\psi(a)\varphi(a)-\psi(a)\varphi(b)+\varphi(a)\psi(a)}{\varphi(b) - \varphi(a)}\] *возможно какая-то ошибка при умножении*
        \[f(b)=\frac{\psi(b) - \psi(a)}{\varphi(b) - \varphi(a)}\varphi(b)-\psi(b) = \frac{\psi(b)\varphi(b)-\psi(a)\varphi(b)-\psi(b)\varphi(b)+\varphi(a)\psi(b)}{\varphi(b) - \varphi(a)}\]
        $\implies f(a)=f(b)$ по теореме Ролля $\implies \exists c \in (a, b): f'(c) = 0$
        \[f'(c) = \frac{\psi(b)-\psi(a)}{\varphi(b)-\varphi(a)} \cdot \varphi'(c)-\psi'(c) = 0 \implies \frac{\psi(b)-\psi(a)}{\varphi(b)-\varphi(a)} = \frac{\psi'(c)}{\varphi'(c)}\]
\end{proof}

\begin{theorem}[Теорема Лагранжа]
        Если $f(x)$ непрерывна на $[a, b]$ и дифференцируема на $(a, b) \implies \exists c \in (a, b): f(a)-f(b)=f'(c)(a-b)$ *рисунок*
\end{theorem}

\begin{theorem}[О возрастании/убывании функции в промежутке]
    Если $f(x)$ непрерывна на $[a, b]$ и $f'(x)>0 (\geq 0, <0, \leq 0): \forall x \in (a, b) \implies f(x)$ - возрастает (неубывает, убывает, невозрастает) на $(a, b)$.
\end{theorem}
\begin{proof}
    Возьмём $\forall x_1, x_2: a < x_1 < x_2 < b$. По теореме Лагранжа $\exists c \in (x_1, x_2): f(x_2) - f(x_1) = f'(c)(x_2-x_1) \implies f(x_2) > f(x_1)$, что по определению означает, что $f(x)$ возрастает на $(a, b)$. Остальное - аналогично. 
\end{proof}

\begin{theorem}[О константе]
    Если $\forall x \in (a, b)$  $\exists f'(x) = 0 \implies \forall x \in (a, b)$ $f(x) \equiv$ c, где $c = const$.
\end{theorem}
\begin{proof}
    Возьмём $\forall x_1, x_2: a < x_1 < x_2 < b$. По теореме Лагранжа $f(x_2) - f(x_1) = f'(c)(x_2-x_1)$, где $c \in (x_1, x_2) \implies f(x_2) = f(x_1) \equiv c \implies \forall x \in (a, b)$ $f(x) \equiv c$
\end{proof}

\begin{theorem}[Первое достаточное условение локального экстремума]
    Пусть $f(x)$ непрерывна в некоторой окрестности точки $x_0$, и в этой окрестности $f'(x) \geq 0$ $(\leq 0)$ при $x < x_0$ и $f'(x) \leq 0$ $(\geq 0)$ при $x > x_0$. Тогда в точке $x_0$ функция $f(x)$ имеет локальный максимум (минимум).
\end{theorem}
\begin{proof}
    Возьмём произвольную точку $x$ из заданной окрестности. Тогда по теореме Лагранжа $f(x)-f(x_0) = f'(c)(x - x_0)$, где $c$ между $x$ и $x_0$. Тогда $f'(c)(x - x_0) \leq 0$ для $\forall x$ из этой окрестности $\implies f(x) \leq f(x_0)$, что по определению $\implies x_0$ - точка локального максимума.
\end{proof}

\begin{theorem}[Второе достаточное условие локального экстремума]
    Пусть $f(x)$ имеет $f'(x)$ в некоторой окрестности точки $x_0$ и $\exists f''(x_0)$. Если $f'(x_0) = 0$, а $f''(x_0) > 0 (< 0)$, то $x_0$ - точка локального максимума (минимума).
\end{theorem}
\begin{proof}
    $\exists f'(x)$ в некоторой окрестности точки $x_0 \implies f(x)$ непрерывна в этой окрестности. $f''(x_0) > 0$, по теореме Дарбу $\implies f'(x_0)$ возрастает в точке $x_0$. Но $f'(x_0) = 0 \implies f'(x) < 0$ при $x < x_0$ и $f'(x) > 0$ при $x > x_0 \implies$ по теореме о первом условии, точка $x_0$ - точка локального минимума $f(x)$.
\end{proof}


\subsection{Производные и дифференциалы для функций многих переменных}
\begin{definition}
    Пусть $u = f(x)$, $x, x_0 \in \R^m$ и тогда $$\exists \lim_{\Delta x_k \to 0} \frac{f(x^0_1, x^0_2, ..., x^0_{k-1}, x^0_k+\Delta x_k, x^0_{k+1}, ..., x^0_m)-f(x^0_1, ..., x^0_m)}{\Delta x_k}=\frac{df}{dx_k}(x_0)$$

    Пример: $u(x, y) = x^3 \cdot y^2$. 
    
    $\frac{du}{dx} = y^2 \cdot 3x^2$, $\frac{du}{dy} = x^3 \cdot 2y$
\end{definition}
\begin{definition}
    Если $\omega \in \R^m: |\omega| = 1$, то производной функции $f(x)$ в точке $x_0$ по направлению вектора $\omega$ называется $$\lim_{t \to 0+} \frac{f(x_0+t\omega)-f(x_0)}{t} = \frac{df}{d\omega}(x_0)$$
*рисунок*
\end{definition}

\begin{definition}
    Пусть $f(x)$ определена в некоторой окрестности точки $x_0$, $(x_0 \in \R^m)$, и $\Delta f(x_0) = f(x_0+\Delta x) - f(x_0)$, ($\Delta x \in \R^m$, $\Delta x(\Delta x_1, ..., \Delta x_m)$) полное приращение можно представить в виде $$\Delta f(x_0) = \sum_{k=1}^m A_k \cdot \Delta x_k + o(|\Delta x|)$$ при $\Delta x \to 0$.
    $(|a| = \sqrt{a^2_1+...+a^2_m})$, то $f(x)$ называют дифференцируемой в точке $x_0$. Сумму $$\sum_{k=1}^m A_k \cdot \Delta x_k$$ называют полным дифференциалом функции $f(x)$ в точке $x_0$ и обозначают $df(x_0)$.
\end{definition}

\begin{theorem}[Связь дифференцируемости и существования частных производных]
    Если $f(x)$ дифференцируема в точке $x_0$, $(x, x_0 \in \R^m) \implies \exists \frac{df}{dx_k}(x_0)$, $\forall k: 1 \leq k \leq m$
\end{theorem}
\begin{proof}
    $f(x)$ - дифференцируема в точке $x_0$ по определению $\implies$ $$\Delta f(x_0) = \sum_{k=1}^m A_k \cdot \Delta x_k + o(|\Delta x|)$$ при $\Delta x \to 0$.
    Возьмём $\Delta x = (0,..., 0, \Delta x_k, 0,...,0) \implies f(x^0_1, x^0_2, ..., x^0_{k-1}, x^0_k+\Delta x_k, x^0_{k+1}, ..., x^0_m)-f(x^0_1, ..., x^0_m) = A_k \cdot \Delta x_k + o(|\Delta x|)$ при $\Delta x_k \to 0 \implies$ 
    $$\exists \lim_{\Delta x_k \to 0} \frac{f(x^0_1, x^0_2, ..., x^0_{k-1}, x^0_k+\Delta x_k, x^0_{k+1}, ..., x^0_m)-f(x^0_1, ..., x^0_m)}{\Delta x_k} = \lim_{\Delta x_k \to 0} (A_k + \frac{o(|\Delta x_k|)}{\Delta x_k}) \implies \exists \frac{df}{dx_k}(x_0) = A_k$$, где $\forall k: 1 \leq k \leq m$
\end{proof}

Замечание: это условие является необходимым, но не достаточным.

\begin{theorem}[Связь дифференцируемости и непрерывности]
    Если $f(x)$ дифференцируема в точке $x_0$, то она непрерывна в точке $x_0$.
\end{theorem}
\begin{proof}
    $f(x)$ дифференцируема в точке $x_0$ $\implies$ $$\Delta f(x_0) = \sum_{k=1}^m \frac{df}{dx_k}(x_0) \cdot \Delta x_k + o(|\Delta x|), \Delta x \to 0$$
    Рассмотрим $$\lim_{\Delta x \to 0} \Delta f(x_0) = \lim_{\Delta x \to 0} (\sum_{k=1}^m \frac{df}{dx_k}(x_0) \cdot \Delta x_k + \frac{o(|\Delta x|)}{|\Delta x|} \cdot |\Delta x|) = 0$$
    Значит, $f(x)$ непрерывна в точке $x_0$.
\end{proof}

\begin{theorem}[Достаточные условия дифференцируемости фукнций многих переменных]
    Если в некоторой окрестности точки $x_0$ существует $\frac{df}{dx_k}$, где $\forall k = \overline{1, m}$, и они непрерывны в точке $x_0 \implies f(x)$ дифференцируема в точке $x_0$. 
\end{theorem}
\begin{proof}
    $\Delta f(x_0) = f(x^0_1+\Delta x_1, x^0_2+\Delta x_2,..., x^0_m+\Delta x_m)-f(x^0_1,..., x^0_m)$
    Обозначим $$f_k(x_0) = f(x^0_1+\Delta x_1;...; x^0_{k-1}+\Delta x_{k-1}; x^0_k; x^0_{k+1};...; x^0_m)$$
    $f_1(x_0) = f(x^0_1;...;x^0_m) ; f_{m+1}(x_0) = f(x^0_1+\Delta x_1;...; x^0_1+\Delta x^0_m) ; f_2(x_0) = f(x^0_1+\Delta x_1; x^0_2;...; x^0_m)$
    $$= f_{m+2}(x_0) - f_1(x_0) = (f_{m+1}(x_0) - f_m(x_0)) + (f_m(x_0) - f_{m-1}(x_0)) + ... +(f_3(x_0) - f_2(x_0)) + (f_2(x_0) - f_1(x_0)) =$$ 
    $$\sum_{k=1}^m(f_{k+1}(x_0) - f_k(x_0))$$
    Применим к каждому слагаемому теорему Лагранжа:
    $$f_{k+1}(x_0) - f_k(x_0) = f(x^0_1+\Delta x_1;...; x^0_k+\Delta x_k; x^0_{k+1};...; x^0_m) - f(x^0_1+\Delta x_1;...; x^0_{k-1}+\Delta x_{k-1}; x^0_k;...; x^0_m) = $$ $$\frac{df}{dx_k}(x^0_1+\Delta x_1;...; x^0_{k-1}+\Delta x_{k-1}; \overline{x_k}; x^0_{k+1};...; x^0_m)\Delta x_k$$, где
    $\overline{x_k}$ между $x^0_k$ и $x^0_k+\Delta x_k$.
    Так как $\frac{df}{dx_k}$ непрерывна в точке $x_0 \implies$ $$\frac{df}{dx_k}(x^0_1+\Delta x_1;...; x^0_{k-1}+\Delta x_{k-1}; \overline{x_k}; x^0_{k+1};...; x^0_m)$$
    $\implies \frac{df}{dx_k}(x_0)$ при $|\Delta x| \to 0 =$
    $$(\frac{df}{dx_k}(x_0) + o(1))\Delta x_k = \frac{df}{dx_k}(x_0)\Delta x_k + o(\Delta x_k)$$ при $|\Delta x| \to 0 \implies$
    $$\sum_{k=1}^m(\frac{df}{dx_k}(x_0)\Delta x_k + o(\Delta x_k)) = \sum_{k=1}^m\frac{df}{dx_k}(x_0)\Delta x_k + o(\Delta x_k)$$ при $|\Delta x| \to 0$, что по определению означает, что $f(x)$ дифференцируема в точке $x_0$.
\end{proof}

\begin{definition}
    Вектор $\frac{df}{dx_1}(x_0), \frac{df}{dx_2}(x_0), ..., \frac{df}{dx_m}(x_0)$ называется вектором градиента функции $f(x)$ в точке $x_0$ и обозначается
    $$grad f(x_0) = (\frac{df}{dx_1}(x_0), \frac{df}{dx_2}(x_0), ..., \frac{df}{dx_m}(x_0))$$
    Замечание:

    Так как $$df(x_0) = \sum_{k=1}^m\frac{df}{dx_k}(x_0)\Delta x_k = grad f(x_0) \Delta x \implies$$
    $$\frac{df}{d\omega}(x_0) = \lim_{t \to +0}\frac{f(x_0+t\omega)-f(x_0)}{t} = \lim_{t \to +0}\frac{\sum_{k=1}^m\frac{df}{dx_k}(x_0)\cdot t\omega_k + o(|t\omega|)}{t} =$$
    $$\lim_{t \to t_0}(\sum_{k=1}^m\frac{df}{dx_k}(x_0)\omega_k+\frac{o(t)}{t}) = \sum_{k=1}^m\frac{df}{dx_k}(x_0)\omega_k = grad f(x_0)\cdot\omega$$
\end{definition}

\begin{theorem}[О направлении наибольшего роста фукнции]
    $|\frac{df}{d\omega}(x)| \leq |grad f(x)|$
\end{theorem}
\begin{proof}
    $|\frac{df}{d\omega}(x)| = |grad f(x)\cdot\omega| \leq |grad f(x)|\cdot|\omega|$, где $|\omega| = 1$.

    И $|\frac{df}{d\omega}(x)| = |grad f(x)|$, если $\omega = \frac{gradf(x)}{|gradf(x)|}$, то есть функция растет быстрее всего в направлении своего градиента.
\end{proof}

\begin{theorem}[О дифференцируемости сложной функции]
    Пусть дана функция $\varphi_k(x)$, $k = \overline{1,l}$ и $f(t)$ дифференцируема в точке $x = a = (a_1,...,a_m)$ и $t = \varphi(a) = b = (b_1,...,b_l)$ соответственно.
    
    Имеем $(x\in\R^m, t\in\R^l; \varphi(a) = (\varphi_1(a),...,\varphi_l(a))) = (\varphi_1(a_1,...,a_m),...,\varphi_l(a_1,...,a_m)) \implies f(x)=f(\varphi(x))$ - дифференцируема в точке $a$.
\end{theorem}
\begin{proof}
    $\varphi_k(x)$ дифференцируема в точке $a \implies$ \[\Delta \varphi_k(a) = \varphi_k(x) - \varphi_k(a) = \sum_{p=1}^m \frac{d\varphi_k}{dx_p}(a) \cdot \Delta x_p + o(|\Delta x|)\]
    $f(t)$ дифференцируема в точке $b \implies$ \[\Delta f(b) = f(t) - f(b) = \sum_{k=1}^l \frac{df}{dt_k}(b) \cdot \Delta t_k + o(|\Delta t|)\]

    Тогда \[\Delta F(a) = F(x) - F(a) = f(\varphi(x)) - f(\varphi(a)) = \sum_{k=1}^l \frac{df}{dt_k}(b) \cdot \Delta \varphi_k(a) + o(|\Delta \varphi_k(a|) =\]
    \[= \sum_{k=1}^l \frac{df}{dt_k}(b) \cdot (\sum_{p=1}^m \frac{d\varphi_k}{dx_p}(a) \cdot \Delta x_p + o(|\Delta x|)) + o(\sum_{p=1}^m \frac{d\varphi_k}{dx_p}(a) \cdot \Delta x_p + o(|\Delta x|)) = \]
    \[\sum_{k=1}^l \sum_{p=1}^m \frac{df}{dt_k}(b) \cdot \frac{d \varphi_k}{dx_p}(a) \cdot \Delta x_p + o(|\Delta x|)\] (при $|\Delta x| \to 0$) $\implies F(x) = f(\varphi(x))$ дифференцируема в точке $a$.
\end{proof}

Замечание:
Получили формулы дифференцирования сложной функции:
\[\frac{dF}{dx_p}(a) = \sum_{k=1}^l \frac{df}{dt_k}(b) \cdot \frac{d \varphi_k}{dx_p}(a)\]

\begin{definition}
    Множество A называется выпуклым если $\forall x_1, x_2 \in A$ точки $x(t) = x_1 + t(x_2 - x_1)$ при $(0 \leq x \leq 1) \in A$
\end{definition}

\begin{theorem}[Обобщение формулы Лагранжа]
    Пусть $f(x)$ дифференцируема на выпуклом множестве $A$, тогда $\forall x_1, x_2 \in A$ \[f(x_2) - f(x_1) = \sum_{k=1}^m \frac{dt}{dx_k}(x(t)) + \Delta x_k\] где $x(t) = x_1 + t(x_2 - x_1), 0 < t < 1, \Delta x_k = x^2_k - x^1_k$
\end{theorem}
\begin{proof}
    Рассмотрим выпуклое множество $A \implies \forall x_1, x_2 \in A$ $x(t) = x_1 + t(x_2 - x_1) \in A, 0 \leq t \leq 1$  

    Рассмотрим $\varphi(t) = f(x(t))$ - функция одного переменного, тогда $\varphi(0) = f(x_1)$, $\varphi(1) = f(x_2)$, $\varphi(t)$ дифференцируема в $[0, 1]$, то есть на множестве $A$, а $x_k(t)$ - линейная функция, дифференцируема на $[0, 1]$, и \[\varphi'(t) = \sum_{k=1}^m \frac{df}{dx_k}(x(t)) \cdot x'_k(t) = \sum_{k=1}^m \frac{df}{dx_k}(x(t)) \cdot \Delta x_k\]
    Тогда по теореме Лагранжа для фукнции одной переменной $\varphi(1) - \varphi(0) = \varphi'(t)(1 - 0)$, $0 < t < 1$

    Следовательно, $\varphi(1) = f(x_2)$, $\varphi(0) = -f(x_1) \implies$ \[f(x_2) - f(x_1) = \sum_{k=1}^m \frac{df}{dx_k}(x(t)) \cdot \Delta x_k\]
\end{proof}

\begin{definition}
    Если каждой точке $x$ из множества $A$ поставить в соответствие $\frac{df}{dx_k}(x)$, то мы получим функцию переменных $(x_1, ..., x_n)$.

    Тогда $\frac{d^2f}{dx_pdx_k} = \frac{d}{dx_p}(\frac{df}{dx_k})$ и т.д.
    
    Если $\frac{d}{dx_k}(\frac{df}{dx_k}) = \frac{d^2f}{dx^2_k} = f''x_kx_k$ - двойная производная

    Если $\frac{d}{dx_l}(\frac{df}{dx_k}) = \frac{d^2f}{dx_ldx_k} = f''x_kx_l$ - смешанная производная
\end{definition}

\begin{theorem}[О равенстве смешанных производных 2-го порядка для функций двух переменных]
    Пусть в некоторой окрестности точки $(x_0; y_0)$ функция $z = f(x,y)$ имеет частные производные $\frac{d^2f}{dydx}$ и $\frac{d^2f}{dxdy}$ и они непрерывны в $(x_0; y_0) \implies \frac{d^2f}{dxdy}(x_0; y_0) = \frac{d^2f}{dydx}(x_0; y_0)$
\end{theorem}
\begin{proof}
    Рассмотрим квадрат: $x_0 \leq x \leq x_0+h; y_0 \leq y \leq y_0+h$, где возьмём такое $h$, чтобы квадрат находился в этой окрестности.
    
    *рисунок*

    Пусть даны функции $\varphi(x) = f(x; y_0+h) - f(x; y_0)$ и $\psi(y) = f(x_0+h; y) - f(x_0; y)$

    Имеем $\varphi(x_0+h) - \varphi(x_0) = f(x_0+h; y_0+h) - f(x_0+h; y_0) - f(x_0; y_0+h) + f(x_0; y_0) = \psi(y_0+h) - \psi(y_0)$

    Итак, $\varphi(x_0+h) - \varphi(x_0) = \psi(y_0+h) - \psi(y_0)$. По теореме Лагранжа: $\exists c_1$ между $x_0$ и $x_0+h$ и $\exists c_2$ между $y_0$ и $y_0+h$ такие что:
    \[\varphi'(c_1) \cdot h = \psi'(c_2) \cdot h \implies \varphi(c_1) = \psi(c_2)\]
    Тогда $\frac{df}{dx}(c_1; y_0+h) - \frac{df}{dx}(c_1; y_0) = \frac{df}{dy}(x_0+h; c_2) - \frac{df}{dy}(x_0; c_2)$. Применим теорему Лагранжа ещё раз, получим

    $\exists \widetilde{c_1}$ между $x_0$ и $x_0+h$; $\exists \widetilde{c_2}$ между $y_0$ и $y_0+h$.
    \[\frac{d^2f}{dydx}(c_1; \widetilde{c_2}) = \frac{d^2f}{dxdy}(\widetilde{c_1} ; c_2) | \lim_{x \to 0}\]
    И так как $\frac{d^2f}{dydx}, \frac{d^2f}{dxdy}$ непрерывны в $(x_0; y_0) \implies \frac{d^2f}{dydx} (x_0; y_0) = \frac{d^2f}{dxdy}(x_0; y_0)$
\end{proof}

Замечание: имет место и более общая теорема. Если все частные производные n-го порядка существуют и непрерывны в окрестности точки $x_0 (x_0 \in \R^m)$, то значение любой смешанной производной в точке $x_0$ не изменится при изменении порядка дифференцирования.

\begin{definition}
    Дифференциалом 2-го порядка функции $f(x), (x \in \R^m)$ называется \[d^2f(x) = d(df(x)) = d(\sum_{k=1}^m \frac{df}{dx_k}(x) \cdot \Delta x_k) =\]
    \[= \sum_{k=1}^m d(\frac{df}{dx_k}(x)) dx_k = \sum_{k=1}^m \sum_{l=1}^m \frac{d^2f}{dx_ldx_k}(x) dx_k dx_l\]
    \[d^kf(x) = d(d^{n-1}f(x)) = \sum_{k_1,...,k_n = 1}^m \frac{d^nf}{d{x_k}_n,...,d{x_k}_1}(x) \cdot d{x_k}_1 \cdot ... \cdot d{x_k}_n\]
\end{definition}

\begin{theorem}[Функция Тейлора]
    Рассмотрим многочлен $P(x) = d_n x^n + ... + d_0$, запишем его по степеням $(x-a)$:
    \[x^n = ((x-a)+a)^n = \sum_{p=0}^k C^p_k (x-a)^p a^{k-p}, k = \overline{1, n}\]
    Подставим в $P(x)$ и приведём подобные:
    \[P(x) = \sum_{k=0}^n \beta_k (x-a)^k \]
    Так как $P^{(l)}(a) = \sum_{k=0}^n \beta_k ((x-a)^k)^{(l)} |(x=a) = \beta_l \cdot l! \implies \beta_l = \frac{P^{(l)}(a)}{l!} \implies$
    \[\implies P(x) = \sum_{k=0}^n \frac{P^{(k)}(a)}{k!} - (x-a)^k\] - многочлен Тейлора.

    Пусть функция $f(x)$ отлична от многочлена. Если в некоторой окрестности точки $a$ её значение мало отличается от её многочлена Тейлора, то говорят, что имеет место формула Тейлора, т.е.
    \[f(x) = \sum_{k=0}^n \frac{f^{(k)}(a)}{k!}(x-a)^k + R_n(x;a)\] где $R_n(x;a)$ - остаточный член
\end{theorem}

 \begin{theorem}[Теорема Тейлора (для функции одной переменной)]
     Пусть функция $f(x)$ дифференцируема в окрестностях точки $x = a$ до $(n+1)$-го порядка включительно, тогда в этой окрестности имеет место формула Тейлора
     \[f(x) = \sum_{k=0}^n \frac{f^{(k)} (a)}{k!} (x-a)^k + Rn(x)\] и $Rn(x) = \frac{(x-c)^{n-p+1}}{n!p} \cdot (x-a)^p f^{(n+1)} (c)$, где $p>0$; $c$ между $a$ и $x$.
 \end{theorem}
 \begin{proof}
     Рассмотрим вспомогательную функцию: \[\Phi(t) = \sum_{k=0}^n \frac{f^{(k)} (t)}{k!} (x_0 - t)^k + (\frac{x_0 - t}{x_0 - a})^p \cdot Rn(x_0)\] где $x_0$ - фиксированная точка указанной окрестности точки $a$.
     Имеем  $\Phi(a) = f(x_0)$; $\Phi(x_0) = f(x_0)$, значит $\Phi(t)$ непрерывна и дифференцируема на отрезке $[a ; x_0]$ (или $[x_0 ; a]$), значит по теореме Ролля $\implies \exists c \in (a ; x_0)$ (или $(x_0 ; a)$): $\Phi'(c) = 0$
     \[\Phi'(t) = \sum_{k=0}^n \frac{1}{k!} (f^{(k+1)} (t) (x_0 - t)^k - f^{(k)} (t) k (x_0 - t)^{k-1}) + Rn(x_0) \frac{1}{(x_0 - a)^p}(-p(x_0 - t)^{p-1}) =\]
     \[= (f'(t) + f''(t)(x_0 - t) + ... + \frac{f^{(n+1)} (t)}{n!}(x_0 - t)^n) - (f'(t) + f''(t)(x_0 - t) + ... + \] 
     \[\frac{f^{(n)} (t)}{(n-1)!}(x_0 - t)^{n-1}) - \frac{p}{(x_0 - a)} (\frac{x_0 - t}{x_0 - a})^{p-1} Rn(x_0) =\]
     \[= \frac{f^{(n+1)} (t)}{n!}(x_0 - t)^n - \frac{p}{(x_0 - a)} (\frac{x_0 - t}{x_0 - a})^{p-1} Rn(x_0)\]
     Отсюда получаем, что \[\Phi'(c) = 0 \Leftrightarrow \frac{f^{(n+1)} (c)}{n!}(x_0 - c)^n = \frac{p}{(x_0 - a)} (\frac{x_0 - c}{x_0 - a})^{p-1} Rn(x_0)\]
     Следовательно, $Rn(x_0) = \frac{f^{(n+1)} (c)}{n!p}(x_0 - c)^{n - p + 1}(x_0 - a)^p$
 \end{proof}
 Замечание 1 (Форма Коши ($c$ между $a$ и $x$)): если положить $p = 1$, то получим остаточный член в форме Коши: $Rn(x) = \frac{f^{(n+1)} (c)}{n!} (x - c)^n (x - a)$

 Замечание 2 (Форма Лагранжа): если положить $p = n + 1$, то получим остаточный член в форме Лагранжа: $Rn(x) = \frac{f^{(n+1)} (c)}{(n+1)!} (x - a)^{n+1}$

 Замечание 3: если потребовать условие, более слабое, чем условие теоремы, то есть чтобы функция $f(x)$ имела в точке $a$ непрерывную производную $n$-го порядка, то получим формулу Тейлора с остаточным членом в форме Пеано:
 \[f(x) = \sum_{k=0}^{n-1} \frac{f^{(k)} (a)}{k!} (x-a)^k + \frac{f^{(n)} (c)}{n!} (x-a)^n\] где $f^{(n)} (x)$ - непрерывна в окрестностях точки $a \implies \exists \lim_{x \to a} f^{(n)} (c) = f^{(n)} (a)$, то есть $f^{(n)} (c) = f^{(n)} (a) + o(1)$ при $x \to a \implies$
 \[\frac{f^{(n)} (c)}{n!}(x-a)^n = \frac{f^{(n)} (a)}{n!}(x-a)^n + o((x-a)^n) =\]
 \[= \sum_{k=0}^n \frac{f^{(k)} (a)}{k!} (x-a)^k + o((x-a)^n)\] при $x \to a$, где $o((x-a)^n)$ - остаточный член в форме Пеано.

 \begin{theorem}
     Пусть функция $u = f(M) = f(x_1, x_2, ..., x_m)$ задана в некоторой $\eps$-окрестности точки $M_0(x'_1, x'_2, ..., x'_m)$ и $n+1$ раз дифференцируема в указанной $\eps$-окрестности. Тогда полное приращение $\Delta u = f(M) - f(M_0)$ этой функции в точке $M_0$ для любой точки $M$ из указанной $\eps$-окрестности может быть представлено в следующей форме:
     \[\Delta u = du\]
 \end{theorem}