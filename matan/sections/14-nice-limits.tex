\section{Некоторые замечательные пределы}

\subsection{I замечательный предел}

*рисунок*

Сектор единичного радиуса с углом $x \in (0; \frac{\pi}{2})$

\[ S_{\triangle OMA} < S_{\textnormal{сект.} OMA} < S_{\triangle OBA} \]
\[ \frac{1}{2} OM * OA * \sin \angle MOA < \frac{x}{2 \pi} * \pi * OA^2 < \frac{1}{2} OA * OB = OA * \tan \angle BOA \]

Получаем
\[ \sin x < x < \tan x \implies \frac{1}{\tan x} < \frac{1}{x} < \frac{1}{\sin x}\]

Домножим на $\sin x$:

\[ \cos x < \frac{\sin x}{x} < 1 \implies \dslim_{x \to +0} \frac{\sin x}{x} = 1 \]

Но $\frac{\sin x}{x}$ -- чётная $\implies \dslim_{x \to -0} \frac{\sin x}{x} = 1$
$\implies \dslim_{x \to 0} \frac{\sin x}{x} = 1$

\subsection{II замечательный предел}

$[x]$ -- целая часть числа $x$ (наибольшее целое, не превосходящее $x$)

\[ [x] \le x \le [x] + 1 \]
\[
    \left( 1 + \frac{1}{[x] + 1} \right)^{[x]} \le
    \left( 1 + \frac{1}{x} \right)^{x} \le
    \left( 1 + \frac{1}{[x]} \right)^{[x] + 1}
\]

Возьмём такие $[x] = n,\, n \in \N$

\[
    \left( 1 + \frac{1}{n + 1} \right)^{n} \le
    \left( 1 + \frac{1}{x} \right)^{x} \le
    \left( 1 + \frac{1}{n} \right)^{n + 1}
\]

Если $[x] \approach{} +\infty \implies x \approach{} +\infty \implies \dslim_{x \to +\infty} \left( 1 + \frac{1}{x} \right)^x = e$

Пусть $x \approach{} -\infty$. Возьмём $x = -(1 + t)$. Тогда
\[
    \dslim_{x \to -\infty} \left( 1 + \frac{1}{x} \right)^x =
    \dslim_{t \to +\infty} \left( 1 + \frac{1}{t} \right)^t \left( 1 + \frac{1}{t} \right) = e
    \implies \dslim_{x \to \infty} \left( 1 + \frac{1}{x} \right)^x = e
\]


