\section{Числовые последовательности}

\subsection{Числовые последовательности}

\begin{definition}
    Пусть $\forall n \in \mathbb{N}$ поставлено в соответствие единственное $x_n \in \mathbb{R}$,
    тогда говорят, что задана \textbf{числовая последовательность}, которую обозначают
    $\{x_n\}$, число $x_n$ называют $n$-ным членом последовательности.
\end{definition}

\begin{definition}
    Суммой, разностью, произведением, частным, линейной комбинацией последовательностей 
    $\{x_n\}$ и $\{y_n\}$ называется последовательность члены которой образуются 
    по следующим правилам соответственно: 
    $x_n + y_n; x_n - y_n; x_n * y_n; \frac{x_n}{y_n} (y_n \ne 0); \alpha x_n + \beta y_n (\alpha, \beta \in \mathbb{R})$.
\end{definition}

\begin{definition}
    $\{x_n\}$ называют ограниченной, если множество её элементов ограничено.
\end{definition}

\begin{definition}
    $\{x_n\}$ называют возрастающей (неубывающей, убывающей, невозрастающей), если
    $\forall n \in \mathbb{N} \quad x_n < x_{n + 1} (x_n \le x_{n + 1}; x_n > x_{n + 1}; x_n \ge x_{n + 1})$.
\end{definition}

\begin{definition}
    Последовательность $\{x_n\}$ называют сходящейся к числу $a$, а число $a$ называют
    \textbf{пределом} последовательности $\{x_n\}$ и обозначают 
    $\displaystyle\lim_{n \rightarrow \infty} x_n = a$, если 
    $\forall \varepsilon > 0 \quad \exists n_\varepsilon \in \mathbb{N}: \forall n \ge n_\varepsilon$
    выполняется неравенство $|x_n - a| < \varepsilon$.
\end{definition}

\underline{\textbf{Замечание}}.

$|x_n - a| < \varepsilon \Leftrightarrow -\varepsilon < x_n - a < \varepsilon 
\Leftrightarrow a - \varepsilon < x_n < a + \varepsilon
\Leftrightarrow x_n \in (a - \varepsilon; a + \varepsilon)$

\parspace

\underline{\textbf{Замечание}}.

$\displaystyle\lim_{x \rightarrow \infty} x_n = a$ означает, что
$\forall \varepsilon$-окрестностей точки $a$ найдется такой номер, 
начиная с которого все члены последовательности содержатся в этой
$\varepsilon$-окрестности точки $a$.

\begin{definition}
    $\{x_n\}$ называются бесконечно большой, если
    $\forall \varepsilon > 0 \quad \exists n_\varepsilon \in \mathbb{N}:
    \forall n \ge n_\varepsilon$ выполняется $|x_n| > \varepsilon$ и
    обозначается $\dslimn x_n = \infty$.
\end{definition}

\parspace

\underline{\textbf{Пример}}.

Докажем по определению 
$\dslimn \frac{1}{\sqrt[3]{n^2 + 1}} = 0$,
т.е. докажем, что $\forall \varepsilon > 0 \quad \exists n_\varepsilon \in \mathbb{N}:
\forall n \ge n_\varepsilon$ выполняется $\left|\frac{1}{\sqrt[3]{n^2 + 1}} - 0\right| < \varepsilon$
т.к. $\left|\frac{1}{\sqrt[3]{n^2 + 1}}\right| < \frac{1}{\sqrt[3]{n^2}}$,
а $\frac{1}{\sqrt[3]{n^2}} < \varepsilon \Leftrightarrow n > \frac{1}{\sqrt{\varepsilon^3}}
\implies$ если $\forall \varepsilon > 0$ выбрать 
$n_\varepsilon \in \mathbb{N}: n_\varepsilon > \frac{1}{\sqrt{\varepsilon^3}}$,
то $\forall n \ge n_\varepsilon$ выполняется $\left|\frac{1}{\sqrt[3]{n^2 + 1}} - 0\right| < \varepsilon$.

\begin{definition}
    Пусть дана $\{x_n\}$ и $\{k_n\}$-возрастающая последовательность натуральных чисел.
    Тогда $\{y_n\}: y_n = x_{k_n}$ называют подпоследовательностью последовательности $\{x_n\}$.
\end{definition}

\begin{definition}
    Если у $\{x_n\}$ существует подпоследовательность $\{y_n\}$ сходящаяся к числу $a^*$,
    то $a^*$ называют частичным пределом последовательности $\{x_n\}$.
    
    Наибольшим (наименьшим) из частичных пределов $\{x_n\}$ называют верхним (нижним) пределом
    и обозначают $\displaystyle\varlimsup_{n \to \infty} x_n$
    ($\displaystyle\varliminf_{n \to \infty} x_n$).
\end{definition}

\begin{theorem}[Больцано-Вейерштрасса для последовательностей]
    Всякая ограниченная последовательность имеет хотя бы один частичный предел.    
\end{theorem}
\begin{proof}
    $\{x_n\}$ -- ограниченная $\bydef$ множество $A$ её
    элементов ограничено.
    
    \underline{1 случай.} $A$ -- конечно.
    
    $\implies$ некоторому $a \in A$ соответствует бесконечное число элементов последовательности 
    \[x_{k_1} = 1, x_{k_2} = a, \dots, x_{k_n} = a, \dots\]
    \[(k_1 < k_2 < k_3 < \dots < k_n < \cdots)\]
    
    Получаем подпоследовательность $\{y_n\}: y_n = a (y_n = x_{k_n})$ и
    $\dslimn y_n = a \implies a$ -- частичный предел $\{x_n\}$.
    
    \underline{2 случай.} $A$ -- бесконечно (и ограничено).
    
    $\implies$ по \textit{Теореме Больцано-Вейерштрасса для множеств} у множества $A$
    существует предельная точка $x$.
    
    Возьмём числа $\varepsilon_n = \frac{1}{10^n}$ тогда по определению предельной точки
    в $\varepsilon_1$-окрестности точки $x$ найдется бесконечное количество элементов последовательности.
    
    Выберем $x_{k_1}$. В $\varepsilon_2$-окрестности точки $x$ найдется бесконечное
    количество элементов последовательности.
    
    Выберем $x_{k_2}: k_2 > k_1$.
    
    Повторим процесс.
    
    $\{x_{k_n}\}: \left| x_{k_n} - x \right| < \varepsilon_N \quad n \ge N$
    
    Тогда $\forall \varepsilon > 0$ выберем $\varepsilon_N < \varepsilon$
    (возьмём $N > \lg \frac{1}{\varepsilon}$) и $\forall n \ge N$ будем иметь 
    $\left| x_{k_n} - x \right| < \varepsilon$ 
    $\bydef \dslimn x_{k_n} = x$
    $\bydef x$ -- частичный предел последовательности $\{x_n\}$.    
\end{proof}

\underline{\textbf{Замечания}}.
\begin{enumerate}
    \item Тот факт, что $\{x_n\}$ не ограничено сверху (снизу) обозначается как 
    $\displaystyle\varlimsup_{n \to \infty} x_n = +\infty$ 
    ($\displaystyle\varliminf_{n \to \infty} x_n = -\infty$).

    \item Методом, предложенным в Теореме можно доказать, что определение точки
    эквивалентно другому определению: \textit{``Точка $x$ называется предельной точкой
    множества $A$, если существует последовательность, состоящая из элементов
    множества, сходящаяся к этой точке''}.
\end{enumerate}

\begin{theorem}[Свойства сходящихся последовательностей]
    \begin{enumerate}
        \item Если последовательность $\{x_n\} \underset{n \to \infty}{\to} a$,
        то любая её подпоследовательность $\underset{n \to \infty}{\to} a$.
    
        \item Сходящаяся последовательность может иметь только один предел.
        \item Если $\exists \dslimn x_n = a \ne 0 \implies
        \exists n_0 \in \mathbb{N} \quad \forall n \ge n_0$ $x_n$ имеет знак числа $a$.
    
        \item Сходящаяся последовательность ограничена.
        \item Если $\exists \dslimn x_n = x$ и
        $\exists \dslimn y_n = y \implies$
    
        $\exists \dslimn (\alpha x_n + \beta y_n) = \alpha x + \beta y$
        ($\alpha, \beta$ -- const)
    
        $\exists \dslimn (x_n y_n) = x y$
    
        $\exists \dslimn \frac{x_n}{y_n} = \frac{x}{y}$
        ($y_n \ne 0, y \ne 0$)
    
        \item Если $\forall n \ge n_0 \quad x_n \le y_n$ и 
        $\exists \dslimn x_n = x$ и
        $\exists \dslimn y_n = y$ $\implies x \le y$.
    
        \item Если $\forall n \ge n_0 \quad x_n \le y_n \le z_n$ и
        $\exists \dslimn x_n = 
        \dslimn z_n = a$
        $\implies \exists \dslimn y_n = a$
    \end{enumerate}
\end{theorem}
\begin{proof}
    \begin{enumerate}
        %--------------------------------------------------------------------------------------
        \item 
            По условию $\exists \dslimn x_n = a$
            $\bydef \underline{\forall \varepsilon > 0}
            \quad \underline{\exists n_\varepsilon \in \mathbb{N}}: 
            \forall n \ge \underline{n_\varepsilon}$ выполняется 
            $\left| x_n - a \right| < \varepsilon$. Но $\underline{k_n \ge n}$
            ($\{k_n\}$ -- возрастающая последовательность натуральных чисел)
            $\implies$ \underline{выполняется $\left| x_{k_n} - a \right| < \varepsilon$}
            $\implies$ $\dslimn x_{k_n} = a$.
    
        %--------------------------------------------------------------------------------------
        \item 
            \textit{От противного.} Пусть $\dslimn x_n = a$ и
            $\dslimn x_n = b \bydef$
            $\forall \varepsilon > 0 \begin{cases}
                \exists n_1 \in \mathbb{N}: \forall n \ge n_1 \quad \left| x_n - a \right| < \varepsilon \\
                \exists n_2 \in \mathbb{N}: \forall n \ge n_2 \quad \left| x_n - b \right| < \varepsilon
            \end{cases}$
        
            Выберем $n_3 = \max \left( n_1, n_2 \right)$, тогда начиная с номера $n_3$ будут выполняться оба
            неравенства одновременно.
        
            Рассмотрим $\left| a - b \right| = \left| a - x_n + x_n - b \right| = 
            \left| (a - x_n) + (x_n - b) \right| \le \left| x_n - a \right| + 
            \left| x_n - b \right| \underset{\forall n \ge n_3}{<}
            \varepsilon + \varepsilon = 2 \varepsilon$, т.е. $\left| a - b \right| < 2 \varepsilon$,
            при том, что $a - b$ -- неотрицательное число.
        
            Получаем, что неотрицательное число $<$ любого положительного 
            $\implies a - b = 0 \implies a = b$. \textbf{Противоречие.}
    
        %--------------------------------------------------------------------------------------
        \item 
            Имеем: $\exists \dslimn x_n = a \ne 0$
            $\bydef \forall \varepsilon > 0 \quad \exists n_\varepsilon \in \mathbb{N} 
            \forall n \ge n_\varepsilon$ выполняется $\left| x_n - a \right| < \varepsilon$
            или $a - \varepsilon < x_n < a + \varepsilon$.
            
            Возьмём $\varepsilon = \frac{\left| a \right|}{2} > 0 \implies 
            \exists n_0 \in \mathbb{N} \quad \forall n \ge n_0 \quad
            a - \frac{\left| a \right|}{2} < x_n < a + \frac{\left| a \right|}{2}$
        
            Отсюда получаем, что $x_n$ начиная с номера $n_0$ имеют тот же знак, что и число $a$.
    
        %--------------------------------------------------------------------------------------
        \item 
            Дано: $\dslimn x_n = a$
            $\bydef \forall \varepsilon > 0 \quad
            \exists n_\varepsilon \in \mathbb{N}: \forall n \ge n_\varepsilon \quad
            \left| x_n - a \right| < \varepsilon$.
        
            Возьмём $\varepsilon = 1 \implies \exists n_1 \in \mathbb{N}: \forall n \ge n_1 \quad
            \left| x_n - a \right| < 1$ или $a - 1 < x_n < a + 1$.
        
            Возьмём $m = \min \left( x_1, x_2, \dots, x_{n_1 - 1}, a - 1 \right)$.
        
            Возьмём $M = \min \left( x_1, x_2, \dots, x_{n_1 - 1}, a + 1 \right)$.
        
            $\implies \forall n \in \mathbb{N} \quad m \le x_n \le M$ $\implies \{x_n\}$ -- ограничена.
    
        %--------------------------------------------------------------------------------------
        \item 
            Дано:
            
            $\dslimn x_n = x$
            $\bydef \forall \varepsilon > 0 \quad
            \exists n_1 \in \mathbb{N}: \forall n \ge n_1 \quad
            \left| x_n - x \right| < \varepsilon$.
        
            $\dslimn y_n = y$
            $\bydef \forall \varepsilon > 0 \quad
            \exists n_2 \in \mathbb{N}: \forall n \ge n_2 \quad
            \left| y_n - y \right| < \varepsilon$.
        
            $\implies \forall n \ge n_3 = \max \left( n_1, n_2 \right)$ 
            выполняются оба неравенства одновременно.
        
            \begin{enumerate}[label*=\arabic*.]
                \item 
                    $|(\alpha x_n + \beta y_n) - (\alpha x + \beta y)| = 
                    |\alpha (x_n - x) + \beta (y_n - y)| \le 
                    |\alpha| |x_n - x| + |\beta| |y_n - y| 
                    \underset{\forall n \ge n_3}{<}
                    \underbrace{(|\alpha| + |\beta|) \varepsilon}_{\text{обозначим } \varepsilon^*} \\
                    \implies \forall \varepsilon^* > 0 \exists n_{\varepsilon^*} = n_3 \in \mathbb{N}:
                    \forall n \ge n_{\varepsilon^*}$ выполняется 
                    $|(\alpha x_n + \beta y_n) - (\alpha x + \beta y)| < \varepsilon^* \\
                    \bydef \dslimn (\alpha x_n + \beta y_n) = \alpha x + \beta y$
        
                \item 
                    Рассмотрим $|x_n y_n - x y| = |x_n y_n - x y_n + x y_n - x y| = 
                    |(x_n y_n - x y_n) + (x y_n - x y)| = 
                    |(x_n - x) y_n + x (y_n - y)| \le |x_n - x| |y_n| + |x| |y_n - y| \quad (1)$
            
                    Но $\{y_n\}$ сходится $\stackrel{\text{по св 4}}{\implies} \{y_n\}$ -- ограничена
                    $\implies \exists C > 0 \quad \forall n \in \mathbb{N} \quad |y_n| \le C$.
            
                    $\implies (1) \underset{\forall n \ge n_3}{<} 
                    \underbrace{\varepsilon (C + |x|)}_{\varepsilon^*} \implies
                    \forall \varepsilon^* > 0 \quad \exists n_{\varepsilon^*} = n \in \mathbb{N}:
                    \forall n \ge n_{\varepsilon^*} \quad |x_n y_n - x y| < \varepsilon^*$.
        
                \item 
                    $\left| \frac{x_n}{y_n} - \frac{x}{y} \right| = 
                    \left| \frac{x_n y - x y_n}{y_n y} \right| =
                    \left| \frac{x_n y - x y + x y - x y_n}{y_n y} \right| =
                    \left| \frac{y (x_n - x) + x (y - y_n)}{y_n y} \right| \stackrel{\triangle}{\le}
                    \frac{|x_n - x}{y_n} + \frac{|x|}{|y_n| |y|} |y_n - y| \quad (2)$
            
                    Так как $\{y_n\}$ сходится, то для 
                    $\varepsilon = \frac{|y|}{2} \quad \exists n_4 \in \mathbb{N} \quad
                    \forall n \ge n_4$ выполняется $|y_n - y| < \frac{|y|}{2} \implies
                    \left| |y_n| - |y| \right| < \frac{|y|}{2} \implies
                    |y_n| > |y| - \frac{|y|}{2} \implies \frac{1}{|y_n|} < \frac{2}{|y|}$.
            
                    $\implies (2) < \underbrace{\varepsilon ( \frac{2}{|y|} + \frac{|x|}{ (|y| - 1) |y| } )}_{\varepsilon^*}
                    \implies \forall \varepsilon^* > 0 \quad 
                    \exists n_{\varepsilon^*} = \max (n_3, n_4) \in \mathbb{N}:
                    \forall n \ge n_4$ выполняется 
                    $\left| \frac{x_n}{y_n} - \frac{x}{y} \right| < \varepsilon^* 
                    \implies \dslimn \frac{x_n}{y_n} = \frac{x}{y}$.
            \end{enumerate}
    
        %--------------------------------------------------------------------------------------
        \item 
            Имеем $\forall n \ge n_0 \quad x_n \le y_n \quad 
            \exists \dslimn x_n = x 
            \bydef \forall \varepsilon > 0 \quad
            \exists n_1 \in \mathbb{N} \quad \forall n \ge n_1 \quad |x_n - x| < \varepsilon$
            и $\exists \dslimn y_n = y 
            \bydef \forall \varepsilon > 0 \quad
            \exists n_2 \in \mathbb{N} \quad \forall n \ge n_2 \quad |y_n - y| < \varepsilon$
        
            Возьмём $n_3 = \max (n_0, n_1, n_2) \implies \forall n \ge n_3$ выполняются
            все три неравенства одновременно.
        
            Пусть \textit{от противного} $x > y$. 
            $x_n - y_n = (x_n - x) - (y_n - y) + (x - y) > 
            -\varepsilon - \varepsilon + (x - y) = (x - y) - 2 \varepsilon$, где $(x - y) > 0$.
        
            Тогда неравенство будет верным, если $2 \varepsilon < (x - y) \implies x_n > y_n$,
            а \textbf{это противоречит условию} $\implies x \le y$.
    
        %--------------------------------------------------------------------------------------
        \item 
            Дано: $\forall n \ge n_0 \quad x_n \le y_n \le z_n, 
            \dslimn x_n = a \bydef
            \forall \varepsilon > 0 \quad \exists n_1 \in \mathbb{N}:
            \forall n \ge n_1 \quad |x_n - a| < \varepsilon$ или
            $-\varepsilon < x_n - a < \varepsilon$.
            
            $\dslimn z_n = a \bydef
            \forall \varepsilon > 0 \quad \exists n_2 \in \mathbb{N}:
            \forall n \ge n_2 \quad |z_n - a| < \varepsilon$ или
            $-\varepsilon < z_n - a < \varepsilon$.
        
            Тогда $\forall n \ge n_3 = \max (n_1, n_2, n_0)$ оба неравенства
            будут выполняться одновременно.
        
            $\underline{-\varepsilon} < x_n - a \le \underline{y_n - a} \le
            z_n - a < \underline{\varepsilon} \Leftrightarrow |y_n - a| < \varepsilon
            \implies \forall \varepsilon > 0 \quad \exists n_\varepsilon = n_3 \in \mathbb{N}:
            \forall n \ge n_3 \quad |y_n - a| < \varepsilon \bydef 
            \dslimn y_n = a$.
    \end{enumerate}
\end{proof}

\begin{theorem}[Свойства верхнего (и нижнего) предела последовательности]
    Если $\exists \displaystyle\varlimsup_{n \to \infty} x_n = a$ 

    или ($\exists \displaystyle\varliminf_{n \to \infty} x_n = a$)
    $\implies \forall \varepsilon > 0$ существует лишь конечное число элементов
    последовательности превосходящие $(a + \varepsilon)$ (меньше $(a - \varepsilon)$).
\end{theorem}
\begin{proof}
    \textbf{ДОКАЗАТЬ ДЛЯ СЛУЧАЯ В СКОБКАХ!!!}

    Пусть $\exists \displaystyle\varlimsup_{n \to \infty} x_n = a$ и $\varepsilon > 0$ и
    пусть \textit{от противного} найдется бесконечное число элементов больших, чем
    $(a + \varepsilon)$. Составим из них новую последовательность. Тогда по 
    \textit{свойству 6 теоремы (свойства сходящихся последовательностей)} любой
    частичный предел этой последовательности будет $\ge (a + \varepsilon)$.
    
    Но он является и частичным пределом исходной последовательности, причём $> a$, но
    это противоречит тому, что $a$ -- верхний предел $\implies$ таких элементов может
    быть только конечное число.
\end{proof}

\begin{definition}
    Последовательность $\{x_n\}$ называется фундаментальной, если 
    $\forall \varepsilon > 0 \quad \exists n_\varepsilon \in \mathbb{N}:
    \forall n \ge n_\varepsilon \quad \forall m \ge n_\varepsilon$
    выполняется $|x_n - x_m| < \varepsilon$.
\end{definition}

\begin{theorem}[Критерий Коши сходимости последовательности]
    Пусть $\{x_n\}$ сходится $\Leftrightarrow$ $\{x_n\}$ фундаментальна.    
\end{theorem}
\begin{proof}
    \begin{enumerate}
        \item 
            Сходится $\implies$ фундаментальна.
            
            По условию $\{x_n\}$ сходится $\implies \exists \dslimn x_n = a
            \implies \underline{\forall \varepsilon > 0 \quad \exists n_\varepsilon \in \mathbb{N} \quad \forall n \ge n_\varepsilon}$
            выполняется $|x_n - a| < \frac{\varepsilon}{2}$ и для $\underline{\forall m \ge n_\varepsilon}$ 
            выполняется $|x_m - a| < \frac{\varepsilon}{2}$.
        
            Рассмотрим $\underline{|x_n - x_m|} = |(x_n - a) + (a - x_m)| 
            \le |x_n - a| + |x_m - a| \le \frac{\varepsilon}{2} + \frac{\varepsilon}{2} = \varepsilon
            \bydef \{x_n\}$ -- фундаментальна.
    
        \item
            Фундаментальна $\implies$ сходится.
            
            Имеем $\{x_n\}$ -- фундаментальна 
            $\implies \forall \varepsilon > 0 \quad \exists n_\varepsilon \in \mathbb{N}:
            \forall n \ge n_\varepsilon \quad \forall m \ge n_\varepsilon$ выполняется
            $|x_n - x_m| < \varepsilon$.
        
            Тогда для $\varepsilon = 1 \quad \exists n_1 \in \mathbb{N} \quad 
            \forall n \ge n_1 \quad \forall m \ge n_1$ выполняется $|x_n - x_m| < 1$
        
            Зафиксируем $m: m \ge n_1$.
        
            $|x_n| = |(x_n - x_m) + x_m| \le |x_n - x_m| + |x_m| < 
            \underbrace{1 + |x_m|}_{C \text{ -- const}} \quad \forall n \ge n_1$
            $\implies \{x_n\}$ -- ограничена $\implies$ по 
            \textit{Теореме Больцано-Вейерштрасса для последовательностей} у $\{x_n\}$
            существует хотя бы один частичный предел, т.е. 
            $\exists \{x_{k_n}\} \underset{n \to \infty}{\to} x$
            $\bydef$ для нашего $\varepsilon > 0 \quad \exists k_{n_\varepsilon} \in \mathbb{N}:
            \forall k_n \ge k_{n_\varepsilon}$ выполняется $|x_{k_n} - x| < \varepsilon$.
        
            Рассмотрим $\forall n \ge k_{n_\varepsilon} \ge n_\varepsilon:
            |x_n - x| = |(x_n - x_{k_n}) + (x_{k_n} - x)| \le |x_n - x_{k_n}| + |x_{k_n} - x|
            < \underbrace{2 \varepsilon}_{\varepsilon^*}$.
        
            Получим:
            $\forall \varepsilon^* \quad \exists n_{\varepsilon^*} = n_\varepsilon \in \mathbb{N}:
            \forall n \ge n_{\varepsilon^*}$ имеем $|x_n - x| < \varepsilon^*$
            $\bydef \exists \dslimn x_n = x$
    \end{enumerate}
        
\end{proof}

\begin{theorem}[Признак сходимости монотонной последовательности]
    Ограниченная сверху (снизу) неубывающая (возрастающая) последовательность сходится.
\end{theorem}

\begin{proof}
    \textbf{ДОКАЗАТЬ ДЛЯ СЛУЧАЯ В СКОБКАХ!!!}

    $\{x_n\}$ -- неубывающая и ограниченная сверху $\implies$ множество элементов
    $\{x_n\}$ ограничено сверху $\implies$ по
    \textit{Теореме О существовании верхней грани} 
    $\exists \sup x_n \stackrel{\text{об.}}{=} a$ $\implies$ по признаку:
    \begin{enumerate}
        \item $\forall n \in \mathbb{N} \quad x_n \le a$
        \item $\forall \varepsilon > 0 \quad 
        \exists x_{n_\varepsilon} : x_{n_\varepsilon} > a - \varepsilon$
    \end{enumerate}
    
    Возьмём $n \ge n_\varepsilon$: 
    $a - \varepsilon < x_{n_\varepsilon} \le x_n \le a < a + \varepsilon
    \implies \forall \varepsilon > 0 \quad \exists n_\varepsilon \in \mathbb{N}:
    \forall n \ge n_\varepsilon$ выполняется $a - \varepsilon < x_n < a + \varepsilon$
    или $|x_n - a| < \varepsilon \bydef \exists \dslimn x_n = a$.
\end{proof}

\subsection{Замечательные пределы}

\begin{enumerate}
    \item 
        $\displaystyle \lim_{n \to \infty} \left( 1 + \frac{1}{n} \right)^n = e$ (Второй замечательный предел)
        
        Рассмотрим $x_n = \left( 1 + \frac{1}{n} \right)^{n + 1}$. 
        Имеем $x_n > 1 \quad \forall n \in \mathbb{N}$ $\implies \{x_n\}$ -- ограничена снизу.

        Рассмотрим $\frac{x_n}{x_{n-1}} = 
        \frac{\left( 1 + \frac{1}{n} \right)^{n - 1}}{\left( 1 + \frac{1}{n - 1} \right)^{n}}
        = \frac{(n + 1)^{n+1} * (n - 1)^n}{n^{n+1} * n^n}
        = \left( \frac{(n + 1) * (n - 1)}{n^2} \right)^n * \frac{n + 1}{n}
        = \left( 1 - \frac{1}{n^2} \right)^{n} * \left( 1 + \frac{1}{n} \right)
        \le \left( 1 - \frac{1}{n^2} \right)^{n} * \left( 1 + \frac{1}{n^2} \right)^n
        \le \left( 1 - \frac{1}{n^4} \right)^{n} < 1 \implies x_n < x_{n-1}$

        Т.е. $\{x_n\}$ -- убывающая последовательность $\implies$
        по \textit{Теореме О сходимости монотонной последовательности}
        $\exists \dslimn x_n = 
        \dslimn \left( 1 + \frac{1}{n} \right)^{n + 1} = e$

        $\dslimn \left( 1 - \frac{1}{n} \right)^{n}
        = \dslimn \frac{x_n}{1 + \frac{1}{n}} = e \approx 2.71828$

    %%%%%%%%%%%%%%%%%%%%%%%%%%%%%%%%%%%%%%%%%%%%

    \item 
        Докажем, что $\dslimn \frac{n^k}{a^n} = 0,
        \quad k \in \mathbb{N}, \quad a > 1$

        $\dslimn \frac{n^k}{a^n} = 
        \dslimn \left( \frac{n}{(\sqrt[k]{a})^n} \right)^k$

        Рассмотрим $(\sqrt[k]{a})^n = (1 + (\sqrt[k]{a} - 1)) ^ n = 
        1 + n * (\sqrt[k]{a} - 1) + \frac{n (n - 1)}{2} (\sqrt[k]{a} - 1)^2
        + \dots + (\sqrt[k]{a} - 1)^n > \frac{n (n - 1)}{2} (\sqrt[k]{a} - 1)^2$

        Получаем $(\sqrt[k]{a})^n > \frac{n (n - 1)}{2} (\sqrt[k]{a} - 1)^2$

        $\implies$ по \textit{$7^{\text{о}}$ свойству сходящихся последовательностей}
        $\dslimn \frac{n}{(\sqrt[k]{a})^n} = 0
        \implies \dslimn \frac{n^k}{a^n} = 0$

    %%%%%%%%%%%%%%%%%%%%%%%%%%%%%%%%%%%%%%%%%%%%

    \item 
        $\dslimn \frac{a^n}{n!} = 0, \quad a > 0$

        $0 < \underbrace{\frac{a}{1} * \frac{a}{2} * \frac{a}{3} * \dots * \frac{a}{k_0 - 1}}_A
        * \underbrace{\frac{a}{k_0}}_{\frac{a}{k_0} = q < 1}
        * \frac{a}{k_0 + 1} * \dots *
        \frac{a}{n} < A * q^{n - k_0 + 1}$

        $\implies$ по \textit{$7^{\text{о}}$ свойству сходящихся последовательностей}
        $\dslimn \frac{a^n}{n!} = 0$

    %%%%%%%%%%%%%%%%%%%%%%%%%%%%%%%%%%%%%%%%%%%%
    
    \item 
        $\dslimn \sqrt[n]{n} = 1$
        
        При $n > 1: \sqrt[n]{n} > 1 \implies \sqrt[n]{n} = 1 + \beta_n \implies
        n = (1 + \beta_n)^n = 1 + n \beta_n + \frac{n (n - 1)}{2} (\beta_n)^2 + \dots +
        (\beta_n)^n > \frac{n (n - 1)}{2} (\beta_n) ^ 2 \implies
        0 < \beta_n < \sqrt{\frac{2}{n-1}} \implies
        \dslimn \beta_n = 0 \implies
        \dslimn \sqrt[n]{n} = 1$
\end{enumerate}

