\section{Предельные точки числового множества}

\subsection{Определение предельных точек}

\begin{definition}
    Множество $x \in \mathbb{R}$, удовлетворяющих неравенству $|x - a| < \varepsilon$
    называется $\varepsilon$-окрестностью точки $a$.
    
    $|x| < c \Leftrightarrow -c < x < c$
    
    $|x - a| < \varepsilon \Leftrightarrow -\varepsilon < x - a < \varepsilon \Leftrightarrow
    a - \varepsilon < x < a + \varepsilon \Leftrightarrow x \in (a - \varepsilon, a + \varepsilon)$.    
\end{definition}

\begin{definition}
    Точка $x$ называется \textbf{предельной} точкой множества $A$, если в любой $\varepsilon$-окрестности
    точки $x$ находятся точки множества $A$, отличные от $x$.        
\end{definition}

\underline{\textbf{Замечание}}

Можно показать, что если точка $x$ -- предельная точка множества $A$, то в любой её
$\varepsilon$-окрестности находится бесконечное число элементов $A$.

Допустим, что это не так и у точки $x$ в некоторой окрестности содержится конечное
число элементов множества $A$ ($x_1,x_2,\dots,x_n \in A$).

Выберем такой элемент, что $|x - x_j|$ был $\underset{1 \le k \le n}{\min} |x - x_k|$,
тогда в $\varepsilon_1$-окрестности, где $\varepsilon_1 = \frac{|x - x_j|}{2}$ нет ни
одного элемента множества, отличного от $x$ $\implies$ $x$ не предельная точка множества
$A$, а это противоречит условию.


\begin{definition}
    Точка $x \in A$ называется \textbf{изолированной} точкой множества $A$, если найдется такая
    окрестность точки $x$, в которой не найдется ни одного элемента множества $A$,
    кроме неё самой.        
\end{definition}

\begin{definition}
    Точка $x$ называется \textbf{внутренней} точкой множества $A$, если найдется такая
    окрестность точки $x$, которая целиком содержится во множестве $A$.
\end{definition}

\begin{definition}
    Множество, все точки которого внутренние называется \textbf{открытым} множеством.    
\end{definition}

\begin{definition}
    Множество, которое содержит все свои предельные точки называется замкнуным множеством.    
\end{definition}

\begin{theorem}[О существовании предельных точек (Теорема Больцано-Вейерштрасса для множеств)]
    У всякого бесконечного ограниченного числового множества существует по крайней мере
    одна предельная точка.    
\end{theorem}

\begin{proof}
    Докажем, что существует самая правая предельная точка.

    $A$ -- ограниченное $\bydef \exists C > 0 \quad
    \forall x \in A: |x| \le C$.
    $C$ без ограничения общности можно выбрать целым.
    Рассмотрим отрезок вида $[k;k+1]$, удовлетворяющий $-c \le k \le c - 1$.
    Так как $A$ -- бесконечное, то по крайней мере в одном из этих отрезков содержится
    бесконечное количество элементов множества $A$.
    
    Выберем самый правый из них и обозначим его $[y_0;\overline{y_0}]$.
    Разобьём его на 10 равных частей $\implies$ по крайней мере в одном из
    полученных отрезков содержится бесконечное число элементов множества $A$.
    
    Выберем самый правый из них и обозначим его $[y_1;\overline{y_1}]$.
    Продолжим процесс деления отрезок на 10 частей и выберем самый правый отрезок
    на котором содержится бесконечное число элементов множества $A$.
    
    Получим отрезок $[y_m;\overline{y_m}]$, где $m = 0,1,\dots$, причём неравенству
    $x > \overline{y_m}$ может удовлетворять только конечное число элементов $A$,
    длина каждого отрезка $= \frac{1}{10^m}$ и концы отрезков удовлетворяют
    следующим неравенствам:
    
    \[y_{m+1} \ge y_m\]
    \[\overline{y_{m+1}} \le \overline{y_m}\]
    \[y_{m+1} - y_m < \frac{1}{10^m}\]
    \[\overline{y_m} - y_m = \frac{1}{10^m}\]
    
    Тогда они являются нижним и верхним $m$-значным приближением некоторого числа $y$.
    
    Покажем, что оно является предельной точкой $A$. Возьмём любую $\varepsilon$-окрестность
    точки $y$ и выберем такое $m$, чтобы: 
    $\frac{1}{10^m} < \varepsilon \implies [y_m;\overline{y_m}] \subset 
    (y - \varepsilon; y + \varepsilon)$, но он содержит бесконечное число элементов
    множества $A$ $\bydef$ $y$ являлся предельной точкой
    множества $A$, а так как правее $\overline{y_m}$ находится только конечное число
    элементов $A$, то $y$ -- самая правая предельная точка $A$.
\end{proof}

